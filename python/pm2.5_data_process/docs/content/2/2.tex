\subsection{概述}
本节分为5部分, 分别介绍数据预处理, 以及对数据进行的五项数据处理, 并在部分章节中对数据进行了数据
可视化. 所有的代码都在src/data\_process.py中的类DataProcess中.
\begin{enumerate}
    \item 对HUMI, PRES, TEMP三列, 进行线性插值处理. 并对其中超过3倍标准差的高度异常
          数据, 修改为3倍标准差的数值.
    \item 假设PM指数最高为500, 对PM\_Dongsi, PM\_Dongsihuan, PM\_Nongzhanguan三列中
          超过500的数据, 修改为500PM指数进行异常值的处理.
    \item 修改cbwd列中值为"cv"的单元格, 其值用后项数据填充.
    \item 对DEWP和TEMP两列, 分别进行了0-1归一化和Z-Score归一化处理, 并将结果使用散点图的形式表示.
    \item 将北京的空气质量数据进行离散化, 按照空气质量分级标准, 计算出每个级别(或颜
          色值)对应的天数各有多少, 并将结果以饼图的形式进行可视化.
\end{enumerate}

\subsection{输入数据格式}
输入数据为csv表格形式存储的, 按日期排列的北京空气质量状况数据.
\begin{lstlisting}
    No,year,month,day,hour,season,PM_Dongsi,PM_Dongsihuan,PM_Nongzhanguan,PM_US Post,DEWP,HUMI,PRES,TEMP,cbwd,Iws,precipitation,Iprec
    1,2010,1,1,0,4,NA,NA,NA,NA,-21,43,1021,-11,NW,1.79,0,0
    2,2010,1,1,1,4,NA,NA,NA,NA,-21,47,1020,-12,NW,4.92,0,0
    3,2010,1,1,2,4,NA,NA,NA,NA,-21,43,1019,-11,NW,6.71,0,0
    4,2010,1,1,3,4,NA,NA,NA,NA,-21,55,1019,-14,NW,9.84,0,0
    5,2010,1,1,4,4,NA,NA,NA,NA,-20,51,1018,-12,NW,12.97,0,0
    6,2010,1,1,5,4,NA,NA,NA,NA,-19,47,1017,-10,NW,16.1,0,0
    7,2010,1,1,6,4,NA,NA,NA,NA,-19,44,1017,-9,NW,19.23,0,0
    8,2010,1,1,7,4,NA,NA,NA,NA,-19,44,1017,-9,NW,21.02,0,0
    9,2010,1,1,8,4,NA,NA,NA,NA,-19,44,1017,-9,NW,24.15,0,0
    
    ...
\end{lstlisting}

\subsection{数据预处理}
下面介绍数据集读入和预处理. 数据集读入和预处理在类DataProcessor中的函数
\_load\_data()中. 由于pandas的read\_csv()可以方便的对读入数据的日期进
行处理, 所以使用如下代码进行处理, 将数据的日期作为索引, 并让pandas自主推断日期格
式. 此外, 对数据中的NaN值进行了规定, 让Pandas可以正确读取NaN值, 而不是读成字符串.
\begin{lstlisting}
    self.raw_df = dataset
    # Read csv using pandas and store data in self.raw_dat and store data in
    # self.raw_data.
    dataset = pd.read_csv(
        in_file,
        parse_dates={"Date": ["year", "month", "day", "hour"]},
        date_parser=lambda x: datetime.strptime(x, "%Y %m %d %H"),
        infer_datetime_format=True,
        index_col="Date",
        na_values=["NaN", "?"],
    )
    self.raw_df = dataset
\end{lstlisting}

输入的数据格式如下:
\begin{lstlisting}
                        No  season  PM_Dongsi  PM_Dongsihuan  ...  cbwd    Iws  precipitation  Iprec
    Date                                                       ...                               
    2010-01-01 00:00:00   1       4        NaN            NaN  ...    NW   1.79            0.0    0.0
    2010-01-01 01:00:00   2       4        NaN            NaN  ...    NW   4.92            0.0    0.0
    2010-01-01 02:00:00   3       4        NaN            NaN  ...    NW   6.71            0.0    0.0
    2010-01-01 03:00:00   4       4        NaN            NaN  ...    NW   9.84            0.0    0.0
    2010-01-01 04:00:00   5       4        NaN            NaN  ...    NW  12.97            0.0    0.0
    2010-01-01 05:00:00   6       4        NaN            NaN  ...    NW  16.10            0.0    0.0
    2010-01-01 06:00:00   7       4        NaN            NaN  ...    NW  19.23            0.0    0.0
    2010-01-01 07:00:00   8       4        NaN            NaN  ...    NW  21.02            0.0    0.0
    2010-01-01 08:00:00   9       4        NaN            NaN  ...    NW  24.15            0.0    0.0
    2010-01-01 09:00:00  10       4        NaN            NaN  ...    NW  27.28            0.0    0.0
\end{lstlisting}

输入数据的统计信息如下:
\begin{lstlisting}
    --- Input data statistics:
<class 'pandas.core.frame.DataFrame'>
DatetimeIndex: 52584 entries, 2010-01-01 00:00:00 to 2015-12-31 23:00:00
Data columns (total 14 columns):
 #   Column           Non-Null Count  Dtype
---  ------           --------------  -----
 0   No               52584 non-null  int64
 1   season           52584 non-null  int64
 2   PM_Dongsi        25052 non-null  float64
 3   PM_Dongsihuan    20508 non-null  float64
 4   PM_Nongzhanguan  24931 non-null  float64
 5   PM_US Post       50387 non-null  float64
 6   DEWP             52579 non-null  float64
 7   HUMI             52245 non-null  float64
 8   PRES             52245 non-null  float64
 9   TEMP             52579 non-null  float64
 10  cbwd             52579 non-null  object
 11  Iws              52579 non-null  float64
 12  precipitation    52100 non-null  float64
 13  Iprec            52100 non-null  float64
dtypes: float64(11), int64(2), object(1)
\end{lstlisting}

\subsection{处理线性插值和高度异常数据}
本节介绍对HUMI, PRES和TEMP三列进行线性插值处理, 并截断其中超出三倍标准差的数据.\par

对线性插值和高度异常数据(超过3倍标准差)的数据的处理, 在类DataProcess中的函数
linear\_interpolate()中.\par

\subsubsection{实现介绍}
使用pandas的函数interpolate()对数据进行线性插值处理如下:
\begin{python}
    # Linear interpolate
    new_col = column.interpolate(
        method="linear", limit_direction="forward")
\end{python}

由于pandas是numpy的扩展, 所以很好的继承了numpy的特性, 例如下面用到的Boolean
Array.
对数据中超出三倍标准差的数据, 通过如下方式截断为三倍标准差:
\begin{python}
    # Process highly anomalous data that exceeds 3 standard deviations.
    std = column.std()
    mean = column.mean()
    new_col[column > mean + 3 * std] = 3 * std + mean
    new_col[column < mean - 3 * std] = 3 * std - mean
\end{python}

\subsubsection{处理结果}
使用样例数据进行测试. 测试结果中省略了不相关的列:
\begin{lstlisting}
    # Raw data before process.
                         No  season  ...  DEWP   HUMI  PRES  TEMP ...
    Date
    2010-01-01 00:00:00   1       4  ...   -21   43.0  1021   -11 ...
    2010-01-01 01:00:00   2       4  ...   -21   47.0  1020   -12 ...
    2010-01-01 02:00:00   3       4  ...   -21   43.0  1019   -11 ...
    2010-01-01 03:00:00   4       4  ...   -21   55.0  1019   -14 ...
    2010-01-01 04:00:00   5       4  ...   -20    NaN  1018   -12 ...
    2010-01-01 05:00:00   6       4  ...   -19   47.0  1017   -10 ...
    2010-01-01 06:00:00   7       4  ...   -19   44.0  1017    -9 ...
    2010-01-01 07:00:00   8       4  ...   -19  440.0  1017    -9 ...
    2010-01-01 08:00:00   9       4  ...   -19   44.0  1017    -9 ...
    2010-01-01 09:00:00  10       4  ...   -20   37.0  1017    -8 ...

    # Run log.
    Linear interpolate and process highly anomalous data in HUMI.
    Modified 1 cells.
    Linear interpolate and process highly anomalous data in PRES.
    Modified 0 cells.
    Linear interpolate and process highly anomalous data in PRES.
    Modified 0 cells.

    # Processed data.
                         No  season  ...  DEWP   HUMI  PRES  TEMP ...
    Date
    2010-01-01 00:00:00   1       4  ...   -21   43.0  1021   -11 ...
    2010-01-01 01:00:00   2       4  ...   -21   47.0  1020   -12 ...
    2010-01-01 02:00:00   3       4  ...   -21   43.0  1019   -11 ...
    2010-01-01 03:00:00   4       4  ...   -21   55.0  1019   -14 ...
    2010-01-01 04:00:00   5       4  ...   -20   51.0  1018   -12 ...
    2010-01-01 05:00:00   6       4  ...   -19   47.0  1017   -10 ...
    2010-01-01 06:00:00   7       4  ...   -19   44.0  1017    -9 ...
    2010-01-01 07:00:00   8       4  ...   -19  440.0  1017    -9 ...
    2010-01-01 08:00:00   9       4  ...   -19   44.0  1017    -9 ...
    2010-01-01 09:00:00  10       4  ...   -20   37.0  1017    -8 ...
\end{lstlisting}
可以看到HUMI中的NaN被插值为51.0.

对完整数据集进行测试, 根据输出日志可以看到, 三列中均有数据被进行过插值或修改到三倍标准差内.
\begin{lstlisting}
Linear interpolate and process highly anomalous data in HUMI.
Modified 339 cells.
Linear interpolate and process highly anomalous data in PRES.
Modified 339 cells.
Linear interpolate and process highly anomalous data in PRES.
Modified 5 cells.
\end{lstlisting}
对线性插值和高度异常数据(超过3倍标准差)的数据的处理, 在类DataProcess中, 