\documentclass[12pt, twoside]{article}
\setlength{\headheight}{14.49998pt}
\usepackage{xeCJK}

% Pseudo-code
\usepackage{algorithm}
\usepackage{algpseudocode}

% Wrap-figure
\usepackage{wrapfig}

\usepackage{listings}
\usepackage{lstautogobble} % Fix relative indenting
\usepackage{xcolor}
\usepackage{multirow}

\usepackage[a4paper, left=3.17cm, right=3.17cm, top=2.54cm, bottom=2.54cm]{geometry}
\usepackage{fancyhdr} % header and footer
\usepackage[T1]{fontenc}
\usepackage{mathptmx}
\usepackage{amsmath}
\usepackage{amsfonts}
\usepackage{chemformula}
\usepackage{cite}
\usepackage[colorlinks, linkcolor=black, anchorcolor=black, citecolor=black]{hyperref}
\usepackage{graphicx}
\usepackage{hyperref}

\usepackage[utf8]{inputenc}

\setCJKmainfont{Songti SC}
\setCJKmonofont{Songti SC}

% Default fixed font does not support bold face
\DeclareFixedFont{\ttb}{T1}{txtt}{bx}{n}{12} % for bold
\DeclareFixedFont{\ttm}{T1}{txtt}{m}{n}{12}  % for normal

% Custom colors
\usepackage{color}
\definecolor{deepblue}{rgb}{0,0,0.5}
\definecolor{deepred}{rgb}{0.6,0,0}
\definecolor{deepgreen}{rgb}{0,0.5,0}

% Python style for highlighting
\newcommand\pythonstyle{\lstset{
language=Python,
basicstyle=\ttm,
morekeywords={self},              % Add keywords here
keywordstyle=\ttb\color{deepblue},
emph={MyClass,__init__},          % Custom highlighting
emphstyle=\ttb\color{deepred},    % Custom highlighting style
stringstyle=\color{deepgreen},
frame=tb,                         % Any extra options here
showstringspaces=false
}}


% Python environment
\lstnewenvironment{python}[1][]
{
\pythonstyle
\lstset{#1}
}
{}

% Python for external files
\newcommand\pythonexternal[2][]{{
\pythonstyle
\lstinputlisting[#1]{#2}}}

% Python for inline
\newcommand\pythoninline[1]{{\pythonstyle\lstinline!#1!}}

% Define colors for code listing
\definecolor{bluekeywords}{rgb}{0.13,0.13,1}
\definecolor{greencomments}{rgb}{0,0.5,0}
\definecolor{redstrings}{rgb}{0.9,0,0}
\definecolor{graynumbers}{rgb}{0.5,0.5,0.5}

\lstset{
  autogobble,
  columns=fullflexible,
  showspaces=false,
  showtabs=false,
  breaklines=true,
  showstringspaces=false,
  breakatwhitespace=true,
  escapeinside={(*@}{@*)},
  commentstyle=\color{greencomments},
  keywordstyle=\color{bluekeywords},
  stringstyle=\color{redstrings},
  numberstyle=\color{graynumbers},
  basicstyle=\ttfamily\footnotesize,
  frame=l,
  framesep=12pt,
  xleftmargin=12pt,
  tabsize=4,
  captionpos=b
  language=C++
  % backgroundcolor=\color{black!5},
}
% show paragraphs in table of contentes
\setcounter{tocdepth}{4}
\setcounter{secnumdepth}{4}

\graphicspath{{figures/}}

\setlength{\parskip}{0.5em}
\title{Python链家五大城市租房数据爬取与分析实验报告}
\author{\textup{吴清柳}}
\begin{document}
\begin{titlepage}
    \newcommand{\HRule}{\rule{\linewidth}{0.5mm}}
    \includegraphics[width=8cm]{title/logo_bupt.png}\\[1cm]
    \center
    \quad\\[1.5cm]
    \textsl{\Large 北京邮电大学}\\[0.5cm]
    \textsl{\large  计算机学院}\\[0.5cm]
    \makeatletter
    \HRule \\[0.4cm]
    {\huge \bfseries \@title}\\[0.4cm]
    \HRule \\[1.5cm]
    % \begin{minipage}{0.4\textwidth}
    %     \begin{flushleft} \large
    %         \emph{Author:}\\
    %         \@author
    %     \end{flushleft}
    % \end{minipage}
    % \large {(C++版和FLEX版均实现)}\\[2cm]
    ~\\[2cm] % Tips: if use previous line, comment or delete this line

    \makeatother
    {\large 姓名: 吴清柳}\\[0.5cm]
    {\large 学号: 2020211597}\\[0.5cm]
    {\large 班级: 2020211323}\\[0.5cm]
    {\large 指导老师: 邵蓥侠}\\[0.5cm]
    {\large \emph{课程名称: 算法设计与分析}}\\[0.5cm]
    {\large \today}\\[2cm]
    \vfill
\end{titlepage}

\tableofcontents
\newpage

% set page stype to fancy then decorate it
\pagestyle{fancy}
\fancyhead{} % clear all header fields
\fancyhead[LE, RO]{吴清柳 2020211597}
\fancyhead[LO, RE]{Python链家五大城市租房数据爬取与分析实验报告}
\fancyfoot{} % clear all footer fileds
\fancyfoot[CE, CO]{\thepage}


\section{概述}
\subsection{题目描述}
对n个整数使用归并排序进行升序排列.
\subsection{输入格式}
输入文件名为mergesort.in, 输入共两行.
\begin{itemize}
	\item 第一行是一个正整数n.
	\item 第二行包含n个整数$n_i$, 每两个整数之间用空格隔开.
\end{itemize}
\subsection{输出格式}
输出文件名为mergesort.out, 输出共一行.
第一行包含n个整数, 为排序后的升序序列, 每两个整数之间用空格隔开.
\subsection{输入输出样例}
\begin{table}[h!]
	\centering
	\begin{tabular}{|l|l|}
		\hline
		mergesort.in & mergesort.out \\
		\hline
		5            & 5 9 11 12 22  \\
		9 11 5 22 12 & ~             \\
		\hline
	\end{tabular}
\end{table}
\subsection{数据范围}
$ 0 < n \leq 10^6 $,
$ | n_i | < 10^8 $

\section{爬虫设计和数据爬取}
在本实验中, 选取北京, 上海, 广州, 深圳和潍坊五个城市, 借助Scrapy框架设计爬虫, 部
署在Scrapyd上运行爬取; 爬取数据以定义的item结构保存, 通过pipeline传输到使用
SQLite3建立的数据库进行去重, 断点续爬和持久化.

爬虫设置方面, 由于部署在Scrapyd上进行爬取, 因此爬取时间不成为限制. 为了避免可能
出现的反爬措施, 设置下载延迟为6\~10秒, 每个域名并行请求数为2, 每个IP并行请求数为
2, 并使用了轮换的User Agent.

由于链家一次只显示100页, 只有3000个房源, 远小于房源总数, 直接对租房房源进行爬取将遗漏大量房
源信息. 因此为了尽可能多的爬取房源, 制作了三个爬虫, 分别爬取板块(商圈, business
area), 小区(community)和租房房源(rental).在爬取板块的同时, 爬虫可以\textbf{自动
    收集行政区划分}, 小区指北京邮电大学家属小区这一级, 租房房源指具体的出租房源信息.
由于一个小区内不会有超过3000各租房房源,因此通过这一划分方式, 可以尽可能的避免这
一限制造成的爬取不充分. 对三个爬虫分别介绍如下.

\subsection{行政区划和板块爬虫}
这一爬虫是首先运行的爬虫, 定义在 ``business\_area\_spider.py'' 中的
\pythoninline{class BusinessAreaSpider} 中.

其部署到scrapyd或者本地运行后, 将以start\_requests中urls中的url为起点, 对每个城
市的行政区, 以及板块进行爬取.

\subsubsection{爬取结果的持久化}
爬取到的板块将保存在定义在``items.py''中的 \pythoninline{class BusinessAreaItem}
中, 并传输给定义在``pipeline.py''中的 \pythoninline{class
    DownBusinessAreaUrlPipeline} , 由该pipeline判断此BusinessAreaItem所在的城市, 行
政区, 以及此板块是否分别存在与相应数据库表中. 如果不存在, 则插入.

BusinessAreaItem定义如下.
\begin{python}
    class BusinessAreaItem(scrapy.Item):
    # Business area
    business_area_url = scrapy.Field()
    business_area_name = scrapy.Field()
    business_area_region = scrapy.Field()
    business_area_city = scrapy.Field()
\end{python}

分别存储了板块的url, 名称, 行政区和所在城市. 而其保存到的SQL中的板块表还有一个
accessbit字段, 用于判断该板块是否已经被下面介绍的小区爬虫爬取过. 借助这一字段,可
以方便的实现\textbf{断点续爬}: 不会对已经爬取过的小区重复爬取, 而是对因为意外中
断而没有来得及爬取的板块中小区进行爬取.

\subsection{小区爬虫}
这一爬虫在行政区划和板块爬虫爬取完毕后运行, 定义在``community\_spider.py''中的
\pythoninline{class ComemunitySpider(scrapy.Spider)}, 在本地或者scrapyd上启动后,
将首先访问SQLite数据库, 获取没有访问的板块url和对应城市url, 并对其开始并行爬取.
由于小区数量很多, 一页不能完全展示, 所以需要处理\textbf{翻页}问题; 此外, 为了避免对一个区
块的重复爬取, 需要设置\textbf{已爬标记}.

\subsubsection{已爬标记和翻页}
每爬取完一页小区后, 需要借助当前页码和总页数进行判断. 当前页和总页数均存储在
\pythoninline{response.xpath("//div[@class='page-box
house-lst-page-box']/@pagedata")} 中. 如果已经是最后一页小区, 则设置该小区所在的
business\_area的accessbit为1, 表示这一区块已经被爬取过. 如果不是最后一页小区, 则
对response.url进行处理, 提取出当前板块的主url, 并链接
\pythoninline{f"pg{next_page}/"}字段, 作为下一页url进行爬取.

\subsubsection{爬取结果的持久化}
爬取结果保存在\pythoninline{class CommunityItem}中, 传输给
\pythoninline{class DownCommunityInfoPipeline}, 由其根据小区链接
\pythoninline{item["community_url"]} 是否已在数据库中来判断该结果是否重复. 如果
不重复, 则插入数据库. CommunityItem定义如下.
\begin{python}
    class CommunityItem(scrapy.Item):
    # Community
    community_url = scrapy.Field()
    community_name = scrapy.Field()
    community_region = scrapy.Field()
    community_city = scrapy.Field()
    community_business_area = scrapy.Field()
    community_rent_url = scrapy.Field()
\end{python}
分别存储小区链接, 小区名, 小区所在行政区, 小区所在城市, 小区所在板块和小区租房链
接. 其中如果小区内没有房屋出租, 则租房链接为`None', 避免数据库中出现null值.

\subsection{租房房源爬虫}
小区信息爬取完毕后, 在小区信息的基础上进行租房房源信息的爬取. 租房房源爬虫为
``rent\_spider.py''中的 \pythoninline{class RentSpider(scrapy.Spider)}, 首先从数
据库中读取租房链接community\_rent\_url非`None', 且访问位community\_accessbit为0
的小区列表, 再对这些小区进行租房房源的并行爬取. 由于单小区内房源也会有一页不能完
全展示的问题, 因此也需要处理\textbf{翻页}问题. 链家的租房房源不同页的url设计比较
奇怪, 给翻页问题的处理带来了一些困扰.

\subsubsection{已爬标记和翻页}
与爬取板块内小区不同的是, 小区内会出现没有房源正在出租的情况. 通过仔细观察, 注意
到字段 \pythoninline{response.xpath("//span[@class='q']/text())} 记录了当前小区
内房源数量. 如果当前小区内租房房源数量为0, 则设置当前小区的访问位
community\_accessbit为2, 跳过当前小区.

否则, 每爬取完小区内的一页租房房源后, 需要借助当前页码和总页数进行判断. 当前页和
总页数均存储在\pythoninline{response.xpath("//div[@class='content__pg']")} 中.
如果已经是最后一页小区,则设置该页所在小区的community\_accessbit为1, 结束对该小区
的访问.

\subsubsection{爬取结果的持久化}
爬取结果保存在\pythoninline{class RentalItem} 中, 传输给
\pythoninline{class DownRentalInfoPipeline}, 由其根据房源url判断是否已经
在SQLite数据库中的rental表中. 如果不存在, 则插入. RentalItem定义如下:
\begin{python}
    class RentalItem(scrapy.Item):
    # Rental info
    rental_name = scrapy.Field()
    rental_url = scrapy.Field()
    rental_city=scrapy.Field()
    rental_region = scrapy.Field()
    rental_business_area = scrapy.Field()
    rental_community_url=scrapy.Field()
    rental_community = scrapy.Field()
    rental_area = scrapy.Field()
    rental_lighting = scrapy.Field()
    rental_rooms = scrapy.Field()
    rental_liverooms = scrapy.Field()
    rental_bathrooms = scrapy.Field()
    rental_price = scrapy.Field()
    rental_timestamp = scrapy.Field()
\end{python}

\subsection{部署在scrapyd上进行爬取}
完成上述过程后, 为了让爬虫能够稳定运行, 在服务器上安装scrapyd, 部署在端口6800并
转发端口. 在本地使用scrapyd-client将爬虫部署到服务器的scrapyd如图~\ref{fig:运行
在Scrapyd上的爬虫}, 并依次运行BusinessAreaSpider, CommunitySpider和RentSpider进
行爬取.
\begin{figure}[ht!]
    \centering
    \includegraphics[width=0.95\textwidth]{scrapyd.png}
    \caption{运行在Scrapyd上的爬虫}
    \label{fig:运行在Scrapyd上的爬虫}
\end{figure}
\section{房租租金分析}
\subsection{huffman}
\subsubsection{题目描述}
对给出的字符设计Huffman编码, 计算期望:
\begin{equation}
	W = \sum_1^n{P_i \times L_i}
	\label{eq:huf-exp}
\end{equation}
\subsubsection{输入格式}
\begin{itemize}
	\item 输入文件名为huffman.in, 输入共两行.
	\item 第一行一个正整数n, 代表字符个数.
	\item 第二行包含n个三位小数$P_i$, 代表第i个字符的出现概率,
	      两个数字之间用空格隔开.
\end{itemize}

\subsubsection{输出格式}
\begin{itemize}
	\item 输出文件名为huffman.out, 输出共一行.
	\item 第一行包含一个\textbf{三位}小数W, 为最后的期望.
\end{itemize}

\subsubsection{输入输出样例}
\label{sec:iosample}
\begin{table}[h!]
	\centering
	\begin{tabular}{|l|l|}
		\hline
		huffman.in              & huffman.out \\
		\hline
		4                       & 1.600       \\
		0.100 0.100 0.200 0.600 & ~           \\
		\hline
	\end{tabular}
\end{table}

\subsubsection{数据范围}
$0 < n \leq 10^6$,
$0<P_i \leq 1$.

\subsubsection{解法}
使用两种方法实现. C++版使用最小堆算法实现建树, Rust版使用不稳定排序算法实现建树,
用以对比不同算法的速度, 以及验证正确性.


\subsection{Dijkstra}
\subsubsection{算法类别}
Dijkstra单源最短路径算法属于贪心算法.

\paragraph{问题描述}
给定带权有向图$G=(V,E)$, 其中每条边的权是\textbf{非负实数}. 另外, 还给定V中的一个顶点,
称为源. 现在要计算从源到所有其他各顶点的最短路长度.
此处的最短路长度指路上各边权之和. 这个问题通常称为\textbf{单源最短路径问题}.

\paragraph{Dijkstra算法的基本思路}
求解的基本思路是, 设置顶点集合S并不断地进行贪心选择来扩充这个集合.
一个顶点属于集合S当且仅当从源到该顶点的最短路径长度已知.\par

初始的时候, S中仅含有源. 设u是G的某一个顶点,
把从源到u且中间只经过S中顶点的路称为从源到u的特殊路径,
并记录当前每个顶点所对应的最短特殊路径长度.
Dijkstra算法每次从V-S中取出具有最短特殊路径长度的顶点u, 将u添加到S中,
同时更新影响到的顶点所对应的最短特殊路径长度. 一旦S中包含了所有V中的顶点,
就已经完成了从源到其他顶点之间的最短路径长度.

\paragraph{0-1背包问题的最优子结构性质}


\subsubsection{关键函数及代码段的描述}
其伪代码描述如下:
\begin{algorithm}
	\caption{Dijkstra's algorithm}\label{alg:dijkstra}
	\begin{algorithmic}{1}
		\Require $(G, w, s)$
		\State $S = \emptyset$
		\State $Q = G.V$
		\While{$Q \neq \emptyset$}
		\State $u = Extract-Min(Q)$
		\State S = S \cup \{u\}
		\For{$each vertex v \in G.Adj[u]$}
		\State Relax(u, v, w)
		\EndFor
		\EndWhile
	\end{algorithmic}

	\begin{algorithmic}{1}
		Require $(u, v, w)$
		\If{$v.d > u.d + w(u, v)$}
		\State $v.d = u.d + w(u, v)$
		\EndIf
	\end{algorithmic}
\end{algorithm}

其核心代码实现如下:

\begin{lstlisting}[language=c++]
void BackPack01::backPackDP() {
    maxVal = -1;

    // Initialize
    for (int j = 0; j <= volume; j++) {
        dp[n - 1][j] = (j >= w[n - 1]) ? val[n - 1] : 0;
    }

    int i = n - 2, j = volume;
    for (; i >= 0; i--) {
        j = volume;
        for (; j >= 0; j--) {

            // Compare total value of items in the backpack between put
            // item[i] in and not put it in.
            dp[i][j] = (j >= w[i]) ? std::max(dp[i + 1][j],
                                              dp[i + 1][j - w[i]] + val[i])
                                   : dp[i + 1][j];
        }
    }

    maxVal = dp[0][volume];
}
\end{lstlisting}

\subsubsection{算法时间及空间复杂性分析}
\paragraph{空间复杂度分析}
0-1背包问题的动态规划实现需要一个大小为N*C的数组进行辅助,
所以空间复杂度为$O(NC)$.

\paragraph{时间复杂度分析}
由于动态规划的本质是上述递推式\ref{eq:01bp-sub-problem},
且实现的核心代码使用了两层循环, 所以可以很方便的计算出时间复杂度为

\begin{equation}
	C(n) = O(N)\times O(C) = O(NC)
\end{equation}

% \begin{gather}
% 	C'(0) = 0            \nonumber \\
% 	C'(k) = 2C'(k-1)+2^k \nonumber
% \end{gather}
综上, 0-1背包问题的一般动态规划解法的时间复杂度为O(NC).

\subsection{应用跳跃点方法优化的0-1背包问题动态规划解法}
\subsubsection{算法类别}
应用了跳跃点方法优化的0-1背包问题解法也是一种动态规划方法.

\subsubsection{算法思路}
\label{sec:jumpPointThink}
\paragraph{概述}
由$m(i, j)$的递归式容易证明, 在一般的情况下, 对每一个确定的$i(1\leq i\leq n)$,
函数$m(i, j)$是关于变量j的阶梯状单调不减函数. 跳跃点是这一类函数的描述特征.
在一般情况下, 函数$m(i, j)$由其全部跳跃点唯一确定. 如图所示.

\begin{figure}[ht!]
	\centering
	\includegraphics[width=0.8\textwidth]{figures/JumpPoint.png}
	\caption{跳跃点}
	\label{fig:JumpPoint}
\end{figure}

对每一个确定的$i,(1\leq i\leq n)$, 用一个链表p[i]存储函数$m(i,j)$的全部跳跃点.
链表p[i]可以计算$m(i,j)$的递归式递归地由表p[i+1]计算,
初始的时候p[n+1]=\{(0,0)\}.

\paragraph{算法改进的描述}
函数$m(i,j)$是由函数$m(i+1,j)$与函数$m(i+1, j-w_i)+v_i$作max运算得到的. 因此,
函数$m(i,j)$的全部跳跃点包含于函数$m(i+1,j)$的跳跃点集p[i+1]与函数
$m(i+1,j-w_i)+v_i$的跳跃点集q[i+1]的并集中.\par

易知, $(s,t)\in q[i+1]$当且仅当$w_i\leq s\leq c$且$(s-w_i,t-v{i})\in p[i+1]$. \par

因此, 容易由p[i+1]确定跳跃点集q[i+1]如下:
\begin{equation}
	q[i+1]=p[i+1]\oplus (w_i,v_i)=\{(j+w_i,m(i,j)+v_i)|(j,m(i,j))\in p[i+1]\}
	\label{eq:jumpSet}
\end{equation}

另一方面, 设(a,b)和(c,d)是$p[i+1]\cup q[i+1]$中的2个跳跃点, 则当$c\geq
	a$且$d<b$的时候, (c,d)受控于(a, b), 从而(c, d)不是p[i]中的跳跃点.
除了受控跳跃点之外, $p[i+1]\cup q[i+1]$中的其他跳跃点都是p[i]中的跳跃点.\par

由此可见, 在递归地由表p[i+1]计算表p[i]的时候, 可以先由p[i+1]计算出q[i+1],
然后合并表p[i+1]和表q[i+1], 并清除其中的受控跳跃点得到表p[i].

\subsubsection{关键函数及代码段的描述}
使用跳跃点优化的动态规划方法求解0-1背包问题的关键在于根据上述推导过程动态
根据上一状态的跳跃点集维护当前状态的跳跃点. 具体代码如下:\par
对函数均使用类似C++的Pseudo-Code描述.

\begin{lstlisting}[language=c++]
void BackPackJumpPoint::JumpPointBackPackDP() {
    int *head = new int[n + 2]; // Track jump point start position.
    head[n] = 0;
    jp[0][0] = 0; // Store item weight
    jp[0][1] = 0; // Store item value

    // Left points to first jump point of p[i+1], right points to last jump
    // point of p[i+1]. Next is position where next jump point will store.
    int left = 0, right = 0, next = 1;
    head[n - 1] = 1; // Points to the position of first jump point of item[n-1].

    for (int i = n - 1; i >= 0; i--) {
        int k = left; // k points to jump points of p[], move k to
                      // evaluate controlled points in p[] and
                      // p[]+(w,v)
        for (int j = left; j <= right; j++) {

            if (jp[j][0] + w[i] > volume) {
                // No enough backpack space to fit item[i] in, exit loop.
                break;
            }

            // Compute new jump point as jp[]+(w,v).
            int x = jp[j][0] + w[i];
            int y = jp[j][1] + val[i];

            // If jp[k][0] < x, then it must be a jump point of current
            // item.
            while (k <= right && jp[k][0] < x) {
                jp[next][0] = jp[k][0];
                jp[next++][1] = jp[k++][1];
            }

            // Clear controlled jump point.
            if (k <= right && jp[k][0] == x) {
                if (y < jp[k][1]) {
                    y = jp[k][1];
                }
                k++;
            }

            if (y > jp[next - 1][1]) {
                jp[next][0] = x;
                jp[next++][1] = y;
            }

            while (k <= right && jp[k][1] <= jp[next - 1][1]) {
                k++;
            }
        }

        // Add remaining jump points.
        while (k <= right) {
            jp[next][0] = jp[k][0];
            jp[next++][1] = jp[k++][1];
        }

        left = right + 1;
        right = next - 1;

        head[i - 1] = next;
    }


    maxVal = jp[next - 1][1];
}
\end{lstlisting}

\subsubsection{算法时间及空间复杂性分析}
\paragraph{空间复杂度分析}
在我的C++版和Rust版实现中, 均采用了邻接表数据结构,
表示具有n个顶点和e条边的带权有向图$G(V, E)$. 在平均情况下, 空间复杂度为$O(V+E)$;
在最坏情况下, 空间复杂度为$O(V^2)$.\par

相比于一般的邻接矩阵表示方式, 邻接表方式在绝大多数情况下所需空间都更少.

\paragraph{时间复杂度分析}


从而, 改进后算法的计算时间复杂性为$O(2^n)$. 当所给物品的重量$w_i(1\leq i\leq n)$
是整数的时候, $|p[i]|\leq c+1, (1\leq i\leq n)$. 在这种情况下,
改进后的算法的计算时间复杂度为$O(min\{nc, 2^n\})$.

% \begin{align}
% 	C(1)       & = 0                        \nonumber \\
% 	C_{min}(n) & = 2C(\frac{n}{2}) + n - 1  \nonumber \\
% 	C_{max}(n) & = C(1) + C(n - 1) + n - 1  \nonumber
% \end{align}

\subsection{Prim}
Prim算法和Kruskal算法都属于最小生成树算法.
\subsubsection{最小生成树}
\paragraph*{生成树} 设一个网络表示为无向连通带权图$G=(V, E)$, E中每条边$(u,
	v)$的权为$w(u, v)$. 如果G的子图G'是一棵包含G的所有顶点的树,
则称G'为G的生成树.\par

\paragraph*{生成树的成本} 生成树上各边权的总和.\par

\paragraph*{G的最小生成树} 在G的所有生成树中, 耗费最小的生成树.

\subsubsection{算法类别}
Dijkstra单源最短路径算法属于贪心算法.

% TODO: Write the following sections.
\subsubsection{问题描述}

\subsubsection{Dijkstra算法的基本思路}

\subsubsection{单源最短路径问题的的最优子结构性质}

\subsubsection{Dijkstra算法的贪心选择性质}

\subsubsection{Dijkstra的最优子结构性质}

\subsubsection{关键函数及代码段的描述}

其核心代码实现如下:

% TODO: Add code here

\subsubsection{算法时间及空间复杂性分析}
\paragraph{空间复杂度分析}

\paragraph{时间复杂度分析}

\subsection{Kruskal}
\subsubsection{算法类别}
Kruskal算法属于贪心算法.

% TODO: Write the following sections.
\subsubsection{问题描述}
设$G=(V,E)$是连通带权图, $V=\{1,2,\dots,n\}$. 求节~\ref{sub:MST}所描述的最小生成树.

\subsubsection{Kruskal算法的基本思路}
\begin{enumerate}
	\item 将G的n个顶点看做n个孤立的联通分支.
	\item 将所有的边按权从小到大排序.
	\item 从权最小的第一条边开始, 依边权递增的顺序查看每一条边(v, w),
	      并按下述方法链接2个不同的联通分支:\par
	      当查看到第k条边(v,w)的时候,
	      \begin{enumerate}
		      \item
		            如果端点v和w分别是当前2个不同的联通分支$T_1$和$T_2$中的顶点的时候,
		            用边$(v,w)$将$T_1$和$T_2$合并成一个联通分支,
		            然后继续查看后续第$k+1$条边.
		      \item 如果端点v和w已经属于当前的同一个联通分支中, 不允许将$(v,w)$加入,
		            否则会产生回路. 此时, 直接查看后续第k$k+1$条边.
		      \item 重复上述过程, 直到只剩下一个联通分支, 即最小生成树.
	      \end{enumerate}
\end{enumerate}

\subsubsection{Kruskal算法的正确性证明}
即证明: 如果T是Kruskal算法从图$G=(V,E)$中选择的生成树, 则T是G的最小生成树.\par
首先, T是一棵生成树.
\begin{description}
	\item[T是森林] 由联通分支的连接方式可知, 在T中没有环生成.
	\item[T是生成的] 假设有一个顶点$v \in G$不在T中的边中.
		则v作为顶点的边一定在算法的某一步被考虑过.
		有且仅有以v为顶点的最小的边应该已经被包括了, 否则有环, 与T的定义冲突.
	\item[T是连通的] 假设T不是联通的, 则T由两个或更多的联通分支. 由于G是联通的,
		过这些联通分支一定会被G中不在T中的边连接.
		这些边中有且仅有最小的一个应该已经被T包含了, 因为T中不能有环,
		否则与T的定义矛盾.
\end{description}
其次, T是最小生成树. 使用归纳法证明. 让$T^*$是一棵最小生成树, 如果$T=T^*$,
则T是最小生成树. 如果$T\neq T^*$, 则$\exists e\in
	T^*$\textbf{有最小权重}且不在T中. 因此, $T\cup e$有一个环$C$使得
\begin{itemize}
	\item 由Kruskal算法建立T的过程知, C中的每条边的全都小于$w(e)$,
	      否则e将在C的建立过程中的某一步被考虑.
	\item $\exists f\in C$使得f不在$T^*$中. (因为$T^*$没有环$C$).
\end{itemize}
考虑树$T_2=T-f+e$:
\begin{enumerate}
	\item $T_2$是生成树.
	\item 相比$T$, $T_2$与$T^*$共有的边更多.
	\item $w(T_2)\geq w(T)$. (我们将T中的边f与$T^*$中的边e进行了交换, 而$w(f) <
		      w(e)$).
\end{enumerate}

重复上面的过程, 每次用相同方法替换一条边, 直到达到T*. 有
\begin{equation}
	w(T)\leq w(T_2)\leq w(T_3)\leq \cdots \leq w(T^*)
\end{equation}

由于$T^*$是最小生成树, 这些不等号必须相等. 因此, T也是最小生成树.
Kruskal算法的正确性得证.

\subsubsection{关键函数及代码段的描述}
使用\textbf{disjoint-union set}数据结构来模拟联通分支对Kruskal算法进行优化. Make-Set(),
Find-Set()和Union()为disjoint-union set数据结构的三个操作.
%TODO: Pseudo-code here.
\begin{algorithm}
	\begin{algorithmic}{1}
		\Require $(G,w)$
		\State $A = \emptyset$
		\For{each vertex $\mathit{v}\in G.V$}
		\State $Make-Set(\mathit(v))$
		sort the edges of G.E into non-decreasing order by weight $\mathit{w}$
		\EndFor
		\For{each edge $(\mathit{u,v}\in G.E$, taken in non-deecreasing order by
			weight}
		\If{$Find-Set(\mathit{u})\neq Find-Set(\mathit{v})$}
		\State $A = A\cup \{(\mathit{u,v})\}$
		\State $Union(\mathit{u,v})$
		\EndIf
		\EndFor
		\return A
	\end{algorithmic}
\end{algorithm}

%TODO: Disjoint-set union.

其核心代码实现如下:

% TODO: Add code here

\subsubsection{算法时间及空间复杂性分析}
\paragraph{空间复杂度分析}
使用邻接表存图, 空间复杂度为$O(V+E)$. 借用二叉最小堆的初始化过程进行排序,
空间复杂度为$O(E)$. 使用Disjoint-set union来维护联通分支, 空间复杂度为$O(E)$.
综上, 空间复杂度为$O(E)$.

\paragraph{时间复杂度分析}
由于使用了二叉最小堆对边进行排序, 堆的初始化开销为$O(|E|lg|E|)$. 在for循环中,
在disjoint-set森林中执行了$O(E)$次Find-Set和Union操作, 还有$|V|$次Make-Set操作,
共计花费了$O((V+E)\alpha (V))$时间, 其中$\alpha$是disjoint-set
union数据结构中一个增长非常缓慢的函数. 由于假设G是联通的, 有$|E|\geq |V|-1$,
因此disjoint-set操作花费$O(E\alpha (V))$时间. 此外, 由于$\alpha(|V|) = O(\lg V)
	= O(\lg E)$, Kruskal算法的总时间复杂度为$O(E\lg E)$. 注意到$|E|\leq |V|^2$,
有$\lg |E| = O(\lg V)$, 因此可以将Kruskal算法的时间复杂度表述为$O(E\lg V)$.





\section{各居室房型比较分析}
\label{ssub:实验心得}
本次实验花费了6小时33分钟在C++排序程序代码编写上, 2小时在makefile
用法的学习和编写上, 9小时20分钟在tex实验报告的编写上.\par

通过这次实验, 我动手实现了MergeSort, QuickSort和
k\'th smallest element选择算法. 动手实践了尾递归调用优化, 以及
各算法的时间复杂度和空间复杂度分析. 此外, 为了综合评价各算法的
正确性, 以及衡量各排序算法的性能, 还动手实现了数据生成器, 用来
方便的生成指定数量的均匀分布在给定范围的, 递增排序, 递减排序或
随机排序的数据作为排序程序的输入.\par

在构建和管理项目的工具方面, 学习并使用了make工具, 方便的实现了
对排序程序的构建和测试; 此外, 对于文档, 也使用了make工具作为管理,
很大程度上方便了实验报告修改后的再次编译. 在实验报告编写方面,
练习了latex的使用, 对latex语法和使用更加熟悉.\par

但是, 由于时间有限, 没有实现QuickSort的多种优化方式, 比如应用
Dijkstra的partiton方法的三路快排, 将递归书写的QuickSort改写为
非递归方式书写的QuickSort等.\par

总的来说, 这次实验在排序算法之内及之外都学到了许多知识, 收获颇丰.
% subsection 实验心得 (end)

\section{各城市不同板块均价比较分析}
比较各个城市不同朝向的单位面积租金分布情况,采用合适的图或表形式进行展示。哪个方
向最高,哪个方向最低?各个城市是否一致?如果不一致,你认为原因是什么?
\section{不同朝向租金比较分析}
\subsection{输出格式概述}
\begin{itemize}
  \item 在Statistic information之前的是输入词法分析程序的源代码文件中解析到的符号, 格式为
L[行号]: <[记号], [属性]>.
  \item Statistic information(统计信息)有如下格式:
  \begin{itemize}
    \item 第一行为行数, 字符数和符号数统计, 格式为[行数] lines, [字符数] characters,
    [符号数] symbols;
    \item 之后跟[符号数]行, 每行是一个符号及其出现次数, 格式为<[记号], [属性]> appeared
     [出现次数] times;
  \end{itemize}
\end{itemize}

\subsection{测试样例1的输出}
测试样例1为较简短的样例, 用于进行输入输出功能和状态转移基本功能的测试;
\subsubsection{C++版词法分析程序测试}
\lstinputlisting{code_lists/sin_1_output}
\subsubsection{FLEX版词法分析程序测试}
\lstinputlisting{code_lists/sin_1_flex}
\subsection{测试样例2的输出}
测试样例2为较长, 测试内容较全面的样例, 用于对词法分析程序实现的各项功能进行测试.
\subsubsection{C++版词法分析程序测试}
\lstinputlisting{code_lists/sin_2_output}
\subsubsection{FLEX版词法分析程序测试}
\lstinputlisting{code_lists/sin_2_flex}
\subsection{分析和总结}
通过C++版和FLEX版分析结果的对照, 说明该C语言词法分析程序有以下功能:
\begin{enumerate}
  \item 可以识别出用C语言编写的源程序中的每个单词符号, 运算符, 数字等,
  并以<记号, 属性>的形式 输出每个单词符号, 包括转义字符;
  \item 可以识别并跳过源程序中的注释;
  \item 可以统计源程序中的语句行数, 各类单词个数, 字符总数, 并输出统计结果;
  \item 可以识别C语言程序源代码中存在的词法错误, 并报告错误出现的位置;
  \item 对源程序中出现的错误进行适当的恢复, 让词法分析可以继续进行;
  \item 对源程序进行一次扫描, 即可检查并报告源程序中存在的所有词法错误, 并输出
  源程序中所有的记号.
\end{enumerate}

\section{人均GDP与单位面积租金的关系}
通过这次词法分析程序的上手实践, 让我对编译原理课程的认识增加了除理论知识之外的内容, 
亲自动手实现词法分析程序也让我对编译器进行词法分析的过程有了更加深入的理解. 

词法分析程序与语法分析程序之间的关系可以有3种, 分别是词法分析程序作为独立的
一遍, 词法分析程序作为语法分析程序的子程序, 和词法分析程序和语法分析程序作为
协同程序. 在本次课程设计中, 将词法分析程序作为了独立的一遍, 可以将词法分析程序
的输出放入到单独的中间文件, 让之后的语法分析程序读取中间文件即可获得词法分析
结果, 有利于提柜编译程序的效率.

此外, 通过本次课程设计, 通过自己的动手亲身体验了将词法分析和语法分析等过程独立
处理的好处: 如可以将各部分需要实现的功能进行良好封装和解耦合, 对外只暴露接口和
提供服务, 各模块的具体实现对外部不可见, 简化了各部分实现的时候需要考虑的内容, 
从而在实现识别并去除空格, 注释等功能的时候思路更加清晰, 还可以让程序可移植性, 
可扩展性更强.

再次, 在本次课程设计中我还尝试了利用FLEX自动生成词法分析程序, 在FLEX自动生成版本
和自己书写的版本的对比中, 体会到了FLEX功能的强大, 灵活和便利, 令我受益匪浅.

总体来说, 在本次课程设计过程中, 我对上学期所学形式语言和自动机知识, 以及本学期
所学的词法分析内容都有了更加深刻的理解, 并掌握了运用方法; 此外, 编程能力, 程序
设计能力等也有了不小的提升.

\section{平均工资与单位面积租金的关系}
\input{content/9/9.tex}
\section{附录}
代码仓库: \href
{https://github.com/Micuks/code/tree/master/python/final/lianjia}
{Lianjia}

% \bibliographystyle{ieeetrans}
% \bibliography{Assignment_Ref}

\end{document}
