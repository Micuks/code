\documentclass[12pt, twoside]{article}
\setlength{\headheight}{14.49998pt}
\usepackage{xeCJK}

% Pseudo-code
\usepackage{algorithm}
\usepackage{algpseudocode}

% Wrap-figure
\usepackage{wrapfig}

\usepackage{listings}
\usepackage{lstautogobble} % Fix relative indenting
\usepackage{xcolor}
\usepackage{multirow}

\usepackage[a4paper, left=1.17cm, right=1.17cm, top=1.54cm, bottom=1.54cm]{geometry}
\usepackage{fancyhdr} % header and footer
\usepackage[T1]{fontenc}
\usepackage{mathptmx}
\usepackage{amsmath}
\usepackage{amsfonts}
\usepackage{chemformula}
\usepackage{cite}
\usepackage[colorlinks, linkcolor=black, anchorcolor=black, citecolor=black]{hyperref}
\usepackage{graphicx}
\usepackage{hyperref}

\usepackage[utf8]{inputenc}

\setCJKmainfont{Songti SC}
\setCJKmonofont{Songti SC}

% Default fixed font does not support bold face
\DeclareFixedFont{\ttb}{T1}{txtt}{bx}{n}{12} % for bold
\DeclareFixedFont{\ttm}{T1}{txtt}{m}{n}{12}  % for normal

% Custom colors
\usepackage{color}
\definecolor{deepblue}{rgb}{0,0,0.5}
\definecolor{deepred}{rgb}{0.6,0,0}
\definecolor{deepgreen}{rgb}{0,0.5,0}

% Python style for highlighting
\newcommand\pythonstyle{\lstset{
language=Python,
basicstyle=\ttm,
morekeywords={self},              % Add keywords here
keywordstyle=\ttb\color{deepblue},
emph={MyClass,__init__},          % Custom highlighting
emphstyle=\ttb\color{deepred},    % Custom highlighting style
stringstyle=\color{deepgreen},
frame=tb,                         % Any extra options here
showstringspaces=false
}}


% Python environment
\lstnewenvironment{python}[1][]
{
\pythonstyle
\lstset{#1}
}
{}

% Python for external files
\newcommand\pythonexternal[2][]{{
\pythonstyle
\lstinputlisting[#1]{#2}}}

% Python for inline
\newcommand\pythoninline[1]{{\pythonstyle\lstinline!#1!}}

% Define colors for code listing
\definecolor{bluekeywords}{rgb}{0.13,0.13,1}
\definecolor{greencomments}{rgb}{0,0.5,0}
\definecolor{redstrings}{rgb}{0.9,0,0}
\definecolor{graynumbers}{rgb}{0.5,0.5,0.5}

\lstset{
  autogobble,
  columns=fullflexible,
  showspaces=false,
  showtabs=false,
  breaklines=true,
  showstringspaces=false,
  breakatwhitespace=true,
  escapeinside={(*@}{@*)},
  commentstyle=\color{greencomments},
  keywordstyle=\color{bluekeywords},
  stringstyle=\color{redstrings},
  numberstyle=\color{graynumbers},
  basicstyle=\ttfamily\footnotesize,
  frame=l,
  framesep=12pt,
  xleftmargin=12pt,
  tabsize=4,
  captionpos=b
  language=C++
  % backgroundcolor=\color{black!5},
}
% show paragraphs in table of contentes
\setcounter{tocdepth}{4}
\setcounter{secnumdepth}{4}

\graphicspath{{figures/}}

\setlength{\parskip}{0.5em}
\title{Python链家五大城市租房数据爬取与分析实验报告}
\author{\textup{吴清柳}}
\begin{document}
\begin{titlepage}
    \newcommand{\HRule}{\rule{\linewidth}{0.5mm}}
    \includegraphics[width=8cm]{title/logo_bupt.png}\\[1cm]
    \center
    \quad\\[1.5cm]
    \textsl{\Large 北京邮电大学}\\[0.5cm]
    \textsl{\large  计算机学院}\\[0.5cm]
    \makeatletter
    \HRule \\[0.4cm]
    {\huge \bfseries \@title}\\[0.4cm]
    \HRule \\[1.5cm]
    % \begin{minipage}{0.4\textwidth}
    %     \begin{flushleft} \large
    %         \emph{Author:}\\
    %         \@author
    %     \end{flushleft}
    % \end{minipage}
    % \large {(C++版和FLEX版均实现)}\\[2cm]
    ~\\[2cm] % Tips: if use previous line, comment or delete this line

    \makeatother
    {\large 姓名: 吴清柳}\\[0.5cm]
    {\large 学号: 2020211597}\\[0.5cm]
    {\large 班级: 2020211323}\\[0.5cm]
    {\large 指导老师: 邵蓥侠}\\[0.5cm]
    {\large \emph{课程名称: 算法设计与分析}}\\[0.5cm]
    {\large \today}\\[2cm]
    \vfill
\end{titlepage}

\tableofcontents
\newpage

% set page stype to fancy then decorate it
\pagestyle{fancy}
\fancyhead{} % clear all header fields
\fancyhead[LE, RO]{吴清柳 2020211597}
\fancyhead[LO, RE]{Python链家五大城市租房数据爬取与分析实验报告}
\fancyfoot{} % clear all footer fileds
\fancyfoot[CE, CO]{\thepage}


\section{概述}
\subsection{题目描述}
对n个整数使用归并排序进行升序排列.
\subsection{输入格式}
输入文件名为mergesort.in, 输入共两行.
\begin{itemize}
	\item 第一行是一个正整数n.
	\item 第二行包含n个整数$n_i$, 每两个整数之间用空格隔开.
\end{itemize}
\subsection{输出格式}
输出文件名为mergesort.out, 输出共一行.
第一行包含n个整数, 为排序后的升序序列, 每两个整数之间用空格隔开.
\subsection{输入输出样例}
\begin{table}[h!]
	\centering
	\begin{tabular}{|l|l|}
		\hline
		mergesort.in & mergesort.out \\
		\hline
		5            & 5 9 11 12 22  \\
		9 11 5 22 12 & ~             \\
		\hline
	\end{tabular}
\end{table}
\subsection{数据范围}
$ 0 < n \leq 10^6 $,
$ | n_i | < 10^8 $

\section{爬虫设计和数据爬取}
在本实验中, 选取北京, 上海, 广州, 深圳和潍坊五个城市, 借助Scrapy框架设计爬虫, 部
署在Scrapyd上运行爬取; 爬取数据以定义的item结构保存, 通过pipeline传输到使用
SQLite3建立的数据库进行去重, 断点续爬和持久化.

爬虫设置方面, 由于部署在Scrapyd上进行爬取, 因此爬取时间不成为限制. 为了避免可能
出现的反爬措施, 设置下载延迟为6\~10秒, 每个域名并行请求数为2, 每个IP并行请求数为
2, 并使用了轮换的User Agent.

由于链家一次只显示100页, 只有3000个房源, 远小于房源总数, 直接对租房房源进行爬取将遗漏大量房
源信息. 因此为了尽可能多的爬取房源, 制作了三个爬虫, 分别爬取板块(商圈, business
area), 小区(community)和租房房源(rental).在爬取板块的同时, 爬虫可以\textbf{自动
    收集行政区划分}, 小区指北京邮电大学家属小区这一级, 租房房源指具体的出租房源信息.
由于一个小区内不会有超过3000各租房房源,因此通过这一划分方式, 可以尽可能的避免这
一限制造成的爬取不充分. 对三个爬虫分别介绍如下.

\subsection{行政区划和板块爬虫}
这一爬虫是首先运行的爬虫, 定义在 ``business\_area\_spider.py'' 中的
\pythoninline{class BusinessAreaSpider} 中.

其部署到scrapyd或者本地运行后, 将以start\_requests中urls中的url为起点, 对每个城
市的行政区, 以及板块进行爬取.

\subsubsection{爬取结果的持久化}
爬取到的板块将保存在定义在``items.py''中的 \pythoninline{class BusinessAreaItem}
中, 并传输给定义在``pipeline.py''中的 \pythoninline{class
    DownBusinessAreaUrlPipeline} , 由该pipeline判断此BusinessAreaItem所在的城市, 行
政区, 以及此板块是否分别存在与相应数据库表中. 如果不存在, 则插入.

BusinessAreaItem定义如下.
\begin{python}
    class BusinessAreaItem(scrapy.Item):
    # Business area
    business_area_url = scrapy.Field()
    business_area_name = scrapy.Field()
    business_area_region = scrapy.Field()
    business_area_city = scrapy.Field()
\end{python}

分别存储了板块的url, 名称, 行政区和所在城市. 而其保存到的SQL中的板块表还有一个
accessbit字段, 用于判断该板块是否已经被下面介绍的小区爬虫爬取过. 借助这一字段,可
以方便的实现\textbf{断点续爬}: 不会对已经爬取过的小区重复爬取, 而是对因为意外中
断而没有来得及爬取的板块中小区进行爬取.

\subsection{小区爬虫}
这一爬虫在行政区划和板块爬虫爬取完毕后运行, 定义在``community\_spider.py''中的
\pythoninline{class ComemunitySpider(scrapy.Spider)}, 在本地或者scrapyd上启动后,
将首先访问SQLite数据库, 获取没有访问的板块url和对应城市url, 并对其开始并行爬取.
由于小区数量很多, 一页不能完全展示, 所以需要处理\textbf{翻页}问题; 此外, 为了避免对一个区
块的重复爬取, 需要设置\textbf{已爬标记}.

\subsubsection{已爬标记和翻页}
每爬取完一页小区后, 需要借助当前页码和总页数进行判断. 当前页和总页数均存储在
\pythoninline{response.xpath("//div[@class='page-box
house-lst-page-box']/@pagedata")} 中. 如果已经是最后一页小区, 则设置该小区所在的
business\_area的accessbit为1, 表示这一区块已经被爬取过. 如果不是最后一页小区, 则
对response.url进行处理, 提取出当前板块的主url, 并链接
\pythoninline{f"pg{next_page}/"}字段, 作为下一页url进行爬取.

\subsubsection{爬取结果的持久化}
爬取结果保存在\pythoninline{class CommunityItem}中, 传输给
\pythoninline{class DownCommunityInfoPipeline}, 由其根据小区链接
\pythoninline{item["community_url"]} 是否已在数据库中来判断该结果是否重复. 如果
不重复, 则插入数据库. CommunityItem定义如下.
\begin{python}
    class CommunityItem(scrapy.Item):
    # Community
    community_url = scrapy.Field()
    community_name = scrapy.Field()
    community_region = scrapy.Field()
    community_city = scrapy.Field()
    community_business_area = scrapy.Field()
    community_rent_url = scrapy.Field()
\end{python}
分别存储小区链接, 小区名, 小区所在行政区, 小区所在城市, 小区所在板块和小区租房链
接. 其中如果小区内没有房屋出租, 则租房链接为`None', 避免数据库中出现null值.

\subsection{租房房源爬虫}
小区信息爬取完毕后, 在小区信息的基础上进行租房房源信息的爬取. 租房房源爬虫为
``rent\_spider.py''中的 \pythoninline{class RentSpider(scrapy.Spider)}, 首先从数
据库中读取租房链接community\_rent\_url非`None', 且访问位community\_accessbit为0
的小区列表, 再对这些小区进行租房房源的并行爬取. 由于单小区内房源也会有一页不能完
全展示的问题, 因此也需要处理\textbf{翻页}问题. 链家的租房房源不同页的url设计比较
奇怪, 给翻页问题的处理带来了一些困扰.

\subsubsection{已爬标记和翻页}
与爬取板块内小区不同的是, 小区内会出现没有房源正在出租的情况. 通过仔细观察, 注意
到字段 \pythoninline{response.xpath("//span[@class='q']/text())} 记录了当前小区
内房源数量. 如果当前小区内租房房源数量为0, 则设置当前小区的访问位
community\_accessbit为2, 跳过当前小区.

否则, 每爬取完小区内的一页租房房源后, 需要借助当前页码和总页数进行判断. 当前页和
总页数均存储在\pythoninline{response.xpath("//div[@class='content__pg']")} 中.
如果已经是最后一页小区,则设置该页所在小区的community\_accessbit为1, 结束对该小区
的访问.

\subsubsection{爬取结果的持久化}
爬取结果保存在\pythoninline{class RentalItem} 中, 传输给
\pythoninline{class DownRentalInfoPipeline}, 由其根据房源url判断是否已经
在SQLite数据库中的rental表中. 如果不存在, 则插入. RentalItem定义如下:
\begin{python}
    class RentalItem(scrapy.Item):
    # Rental info
    rental_name = scrapy.Field()
    rental_url = scrapy.Field()
    rental_city=scrapy.Field()
    rental_region = scrapy.Field()
    rental_business_area = scrapy.Field()
    rental_community_url=scrapy.Field()
    rental_community = scrapy.Field()
    rental_area = scrapy.Field()
    rental_lighting = scrapy.Field()
    rental_rooms = scrapy.Field()
    rental_liverooms = scrapy.Field()
    rental_bathrooms = scrapy.Field()
    rental_price = scrapy.Field()
    rental_timestamp = scrapy.Field()
\end{python}

\subsection{部署在scrapyd上进行爬取}
完成上述过程后, 为了让爬虫能够稳定运行, 在服务器上安装scrapyd, 部署在端口6800并
转发端口. 在本地使用scrapyd-client将爬虫部署到服务器的scrapyd如图~\ref{fig:运行
在Scrapyd上的爬虫}, 并依次运行BusinessAreaSpider, CommunitySpider和RentSpider进
行爬取.
\begin{figure}[ht!]
    \centering
    \includegraphics[width=0.95\textwidth]{scrapyd.png}
    \caption{运行在Scrapyd上的爬虫}
    \label{fig:运行在Scrapyd上的爬虫}
\end{figure}
\section{数据分析和处理}
\newpage
\section{实验总结}
通过这次实验, 我体验了从零开始设计爬虫, 爬取数据到数据分析处理的全部过程, 体会到
了python的方便强大.

在爬虫部署的工具方面, 使用了scrapyd工具, 方便的实现了
爬虫的服务器端自动化爬取.

但是, 由于时间有限, 且感染了奥密克戎, 深受发烧困扰, 没能来得及进行进一步的优化.
例如, 将使用jupyter notebook书写的数据分析部分与前面latex书写的爬虫介绍部分更好
的组合在一起. 也没有实现更多的优化内容, 例如IP池等, 比较遗憾.

总的来说, 这次实验在爬虫, 数据分析和Python之内及之外都学到了许多知识, 收获颇丰.
% subsection 实验心得 (end)

\section{附录}
代码仓库: \href
{https://github.com/Micuks/code/tree/master/python/final/lianjia}
{Lianjia}

% \bibliographystyle{ieeetrans}
% \bibliography{Assignment_Ref}

\end{document}
