\label{ssub:实验心得}
本次实验花费了6小时33分钟在C++排序程序代码编写上, 2小时在makefile
用法的学习和编写上, 9小时20分钟在tex实验报告的编写上.\par

通过这次实验, 我动手实现了MergeSort, QuickSort和
k\'th smallest element选择算法. 动手实践了尾递归调用优化, 以及
各算法的时间复杂度和空间复杂度分析. 此外, 为了综合评价各算法的
正确性, 以及衡量各排序算法的性能, 还动手实现了数据生成器, 用来
方便的生成指定数量的均匀分布在给定范围的, 递增排序, 递减排序或
随机排序的数据作为排序程序的输入.\par

在构建和管理项目的工具方面, 学习并使用了make工具, 方便的实现了
对排序程序的构建和测试; 此外, 对于文档, 也使用了make工具作为管理,
很大程度上方便了实验报告修改后的再次编译. 在实验报告编写方面,
练习了latex的使用, 对latex语法和使用更加熟悉.\par

但是, 由于时间有限, 没有实现QuickSort的多种优化方式, 比如应用
Dijkstra的partiton方法的三路快排, 将递归书写的QuickSort改写为
非递归方式书写的QuickSort等.\par

总的来说, 这次实验在排序算法之内及之外都学到了许多知识, 收获颇丰.
% subsection 实验心得 (end)
