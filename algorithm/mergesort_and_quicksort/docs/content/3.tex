对MergeSort, Quicksort和应用线性选择的QuickSort的程序执行环境和
运行方式均相同, 包括其测试结果都在此一并介绍.

\subsection{程序执行环境}
在以下环境下进行过测试.
\begin{description}
	\item[操作系统] Ubuntu 20.04.5 LTS on Windows 10 x86\_64, \\
		macOS 12.6 21G115 arm64;
	\item[编译器] clang-1400.0.29.102;
	\item[构建和测试工具] GNU Make 3.81;
	\item[文档编译工具] XeTeX 3.141592653-2.6-0.999994 (TeX Live 2022)
\end{description}

\subsection{程序运行方式}
\paragraph{排序程序的编译, 运行和测试方法}
mergesort, 尾递归调用优化的quicksort和应用线性选择优化的quicksort都采用
make进行统一构建和测试, 并使用diff命令和C++的STL库sort结果进行比较,
以确保排序结果正确.\par

程序(及文档)均使用make进行构建, 对于排序程序的构建, 在shell中执行以下命令:
\begin{lstlisting}[language=bash]
make
\end{lstlisting}
二进制文件mergesort, quicksort和quicksort\_kthSmallest将被编译在build/目录下,

如果要运行默认测试, 可以在shell中执行这条命令:
\begin{lstlisting}[language=bash]
make test
\end{lstlisting}

如果要使用数据生成器data\_gen生成指定范围内的均匀分布的随机数进行测试, 可以用
这条命令:
\begin{lstlisting}[language=bash]
make testg
\end{lstlisting}
生成的输入存储在samples/yet\_another\_sample.in中, 程序的输出存储在samples/
${filename} 中, ${filename}为mergesort, quicksort和QuickSort\_kthSmallest.\par

如果要使用数据生成器data\_gen生成指定范围内的均匀分布的随机数进行测试, 可以用
这条命令:
\begin{lstlisting}[language=bash]
make testg [from=<range start>] [to=<range end>] [amount=<number of random numbers>]
# e.g. 
# make testg from=-12345 to=12345 amount=50
\end{lstlisting}
这种参数对下面介绍的生成递增随机数和递减随机数\textbf{同样适用}.\par

如果要使用数据生成器date\_gen生成指定范围内的均匀分布的递增排列的随机数进行测试,
可以用这条命令:
\begin{lstlisting}[language=bash]
make testg_ascend
\end{lstlisting}
生成的输入和程序的输出的存储路径和上述相同.\par

如果要使用数据生成器date\_gen生成指定范围内的均匀分布的递增排列的随机数进行测试,
可以用这条命令:
\begin{lstlisting}[language=bash]
make testg_descend
\end{lstlisting}
生成的输入和程序的输出的存储路径和上述相同.\par

如果要使用自定义文件进行测试, 可以用这条命令:
\begin{lstlisting}[language=bash]
build/mergesort ${infilename} [${outfilename}]
build/quicksort ${infilename} [${outfilename}]
build/quicksort_kthSmallest ${infilename} [${outfilename}]
\end{lstlisting}
其中, \$\{infilename\}是按第一章中的\textbf{输入格式}\ref{sec:iosample}组织的输入数据.
默认情况下将同时输出到标准输出和samples/mergesort.out, 可选自定义输出到
\$\{outfilename\}路径下. 输出格式为第一章中的\textbf{输出格式}\ref{sec:iosample}.

\paragraph{文档编译}
除排序程序使用make管理之外, 本实验的课程报告使用latex进行编写, 因此同样可以
使用make进行管理. 如果要编译文档, 运行这条命令:
\begin{lstlisting}[language=bash]
make docs
\end{lstlisting}
make将调用xelatex进行编译; 为了确保cross-referencing的正确工作, make将执行两遍
xelatex编译.\par
编译生成的中间文件及文档都保存在docs/build下, docs/report.pdf和docs/build/
report.pdf均是生成的实验报告.

\subsection{程序执行示例}
\label{sec:sortBench}
此处演示程序从构建到测试的过程.
\begin{lstlisting}[language=bash]
# user input
make clean

# console output
rm -rf build/*
/Library/Developer/CommandLineTools/usr/bin/make -C docs/ clean
rm -rf build/*

# user input
make sort # Compile sort programs

# console output
clang++ -Wall -std=c++11 -O3 -c -g mergesort/mergesort.cpp -o build/mergesort.o
clang++ -Wall -std=c++11 -O3 build/mergesort.o -o build/mergesort
clang++ -Wall -std=c++11 -O3 -c -g quicksort/quicksort.cpp -o build/quicksort.o
clang++ -Wall -std=c++11 -O3 build/quicksort.o -o build/quicksort
clang++ -Wall -std=c++11 -O3 -c -g quicksort/quicksort_kthSmallest.cpp -o build/quicksort_kthSmallest.o
clang++ -Wall -std=c++11 -O3 build/quicksort_kthSmallest.o -o build/quicksort_kthSmallest
clang++ -Wall -std=c++11 -O3 data_gen/stdsort.cpp -o build/stdsort

# user input
make test # Run default test with sample given with assignment

# samples/mergesort.in:
# 5
# 9 11 5 22 12

# console output
build/mergesort samples/mergesort.in
[MergeSort] Time measured: 1.042e-06 seconds.
build/quicksort samples/quicksort.in
[QuickSort] Time measured: 7.08e-07 seconds.
build/quicksort_kthSmallest samples/quicksort.in
[QuickSort_kthSmallest] Time measured: 3.417e-06 seconds.
build/stdsort samples/quicksort.in
[StdSort] Time measured: 1.541e-06 seconds.
diff samples/stdsort.out samples/mergesort.out
diff samples/stdsort.out samples/quicksort.out
diff samples/stdsort.out samples/quicksort_kthSmallest.out

# user input
# test with random number generator
make testg # Compile data_gen and generate sample for test purpose

# console output
clang++ -Wall -std=c++11 -O3  data_gen/data_gen.cpp -o build/data_gen
echo "[INFO]: generage 1000000 random numbers from -100000000 to 100000000."
[INFO]: generage 1000000 random numbers from -100000000 to 100000000.
build/data_gen -100000000 100000000 1000000
build/mergesort samples/yet_another_sample.in
[MergeSort] Time measured: 0.0442986 seconds.
build/quicksort samples/yet_another_sample.in
[QuickSort] Time measured: 0.0633752 seconds.
build/quicksort_kthSmallest samples/yet_another_sample.in
[QuickSort_kthSmallest] Time measured: 0.260368 seconds.
build/stdsort
[StdSort] Time measured: 0.0522087 seconds.
diff samples/stdsort.out samples/mergesort.out
diff samples/stdsort.out samples/quicksort.out
diff samples/stdsort.out samples/quicksort_kthSmallest.out

# user input
make testg from=-100000000 to=100000000 amount=10000
# It means generate 10000 random numbers ranging from -1e8 to 1e8.

# console output
echo "[INFO]: generage 10000 random numbers from -100000000 to 100000000."
[INFO]: generage 10000 random numbers from -100000000 to 100000000.
build/data_gen -100000000 100000000 10000
build/mergesort samples/yet_another_sample.in
[MergeSort] Time measured: 0.000659459 seconds.
build/quicksort samples/yet_another_sample.in
[QuickSort] Time measured: 0.000819 seconds.
build/quicksort_kthSmallest samples/yet_another_sample.in
[QuickSort_kthSmallest] Time measured: 0.00246375 seconds.
build/stdsort
[StdSort] Time measured: 0.000479917 seconds.
diff samples/stdsort.out samples/mergesort.out
diff samples/stdsort.out samples/quicksort.out
diff samples/stdsort.out samples/quicksort_kthSmallest.out

# user input
make testg_ascend 
# Generate ascending random numbers with date_gen
# to test the performance of sorting algorighms on ascending
# dateset.

# console output
echo "[INFO]: generage 1000000 random numbers from -100000000 to 100000000."
[INFO]: generage 1000000 random numbers from -100000000 to 100000000.
build/data_gen -100000000 100000000 1000000
build/stdsort ascend
cp samples/stdsort.out samples/yet_another_sample.in
build/mergesort samples/yet_another_sample.in
[MergeSort] Time measured: 0.0212245 seconds.
build/quicksort samples/yet_another_sample.in
[QuickSort] Time measured: 0.0133075 seconds.
build/quicksort_kthSmallest samples/yet_another_sample.in
[QuickSort_kthSmallest] Time measured: 0.0807739 seconds.
build/stdsort
[StdSort] Time measured: 0.000937209 seconds.

# user input
make testg_descend
# Generate ascending random numbers with date_gen
# to test the performance of sorting algorighms on ascending
# dateset.

# console output
echo "[INFO]: generage 1000000 random numbers from -100000000 to 100000000."
[INFO]: generage 1000000 random numbers from -100000000 to 100000000.
build/data_gen -100000000 100000000 1000000
build/stdsort descend
cp samples/stdsort.out samples/yet_another_sample.in
build/mergesort samples/yet_another_sample.in
[MergeSort] Time measured: 0.0205804 seconds.
build/quicksort samples/yet_another_sample.in
[QuickSort] Time measured: 0.0150562 seconds.
build/quicksort_kthSmallest samples/yet_another_sample.in
[QuickSort_kthSmallest] Time measured: 0.149183 seconds.
build/stdsort
[StdSort] Time measured: 0.00165979 seconds.

\end{lstlisting}
