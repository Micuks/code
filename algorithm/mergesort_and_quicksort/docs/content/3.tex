\subsection{程序执行环境及运行方式}
\subsubsection{程序执行环境}
在以下环境下进行过测试.
\begin{description}
	\item[操作系统] Ubuntu 20.04.5 LTS on Windows 10 x86\_64, \\
		macOS 12.6 21G115 arm64;
	\item[编译器] clang-1400.0.29.102;
\end{description}
\subsubsection{运行方式}
程序(及文档)均使用make进行构建, 对于程序运行, 在shell中执行以下命令:
\begin{lstlisting}[language=bash]
make
\end{lstlisting}
二进制文件mergesort将被编译在build/目录下, 如果要运行默认测试, 可以
在shell中执行这条命令:
\begin{lstlisting}[language=bash]
make test
\end{lstlisting}
如果要使用其他测试, 可以用这条命令:
\begin{lstlisting}[language=bash]
build/mergesort $(filename)
\end{lstlisting}
其中, \$(filename)是按第一章中的\textbf{输入格式}组织的输入数据. 默认情况下
将同时输出到标准输出和samples/mergesort.out.
\subsubsection{数据生成}
为了方便的生成测试数据, 来测试算法性能和算法正确性, 我设计了一个数据生成器,
可以生成指定数量(如$1e6$)的, 分布在$[-1e8, 1e8]$范围内的如下几种数据:
\begin{itemize}
  \item uniform\_distribution(均匀分布)的, 顺序随机的数据;
  \item 均匀分布的, 正序排列的数据;
  \item 均匀分布的, 倒序排列的数据;
\end{itemize}
\paragraph{设计思路}
常用的C++随机数生成方法均采用rand()函数和模操作获取分布在某一范围内的均匀
分布数据, 但是, 这种方法\href{https://stackoverflow.com/a/52869953}
{可能不会生成均匀分布的数据}
(这取决于生成数据的范围, 以及\textbf{RAND\_MAX}的取值), 因此是不被建议
的方式.\par

因此, 我采用C++11引入的随机数生成操作. C++11标准引入了多种其他的随机数生成
方式, 在这些方式中, \href{http://en.cppreference.com/w/cpp/numeric/random/uniform\_int\_distribution}
{std::uniform\_int\_distribution}
与我的需求吻合, 所以我采用这种方式生成均匀分布的数据.
由于我认为数据的分布方式(均匀分布, 正态分布等)不会影响数据的大小关系分布, 
从而不会影响算法性能, 所以没有生成均匀分布之外的分布的数据.
\paragraph{运行方式}
数据生成器也使用make构建和管理, 对于程序执行, 在shell中执行以下命令:
\begin{lstlisting}[language=bash]
make
\end{lstlisting}
二进制文件data_gen将被编译在build/目录下, 在不传入参数的情况下会默认生成
[-1000, 1000]内的50个整数. 在shell中执行这条命令:
\begin{lstlisting}[language=bash]
make testg
\end{lstlisting}
data_gen可以传入三个参数from, to和amount, 可以生成[from, to]范围内的amount个
整数, 当不传入amount的时候, 将默认生成50个随机数. 在shell中执行这条命令:
\begin{lstlisting}[language=bash]
make testg from=<from> to=<to> [amount=<amount>]
\end{lstlisting}
运行结束后, console会显示排序花费的时间, samples/yet_another_sample.in保存
生成的随机数, samples/mergesort.out保存排序后的结果.
\subsubsection{程序执行示例}
此处演示程序从构建到测试的过程.
\begin{lstlisting}[language=bash]
# user input
make

# console output
clang++ -Wall -std=c++11 -c -g mergesort/mergesort.cpp -o \
build/mergesort.o
clang++ -Wall -std=c++11 build/mergesort.o -o build/mergesort

# user input
build/mergesort
# samples/mergesort.in:
# 5
# 9 11 5 22 12

# console output
Time measured: 2.875e-06 seconds.
# samples/mergesort.out
5 9 11 12 22 

# user input
# test with random number generator
make testg
# samples/yet_another_sample.in
# 50
# 8 35 1 7 62 99 82 95 30 1 7 24 98 38 56 42 96 83 30 ...
# not all displayed

# console output
build/data_gen
build/mergesort samples/yet_another_sample.in
Time measured: 2.416e-06 seconds.
# samples/mergesort.out
0 0 3 4 7 8 10 10 11 11 14 15 15 16 17 20 22 22 26 30 ...
# not all displayed
\end{lstlisting}
