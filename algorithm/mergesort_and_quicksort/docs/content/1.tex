\subsection{归并排序}
\subsubsection{题目描述}
对n个整数使用归并排序进行升序排列.
\subsubsection{输入格式}
输入文件名为mergesort.in, 输入共两行.
\begin{itemize}
	\item 第一行是一个正整数n.
	\item 第二行包含n个整数$n_i$, 每两个整数之间用空格隔开.
\end{itemize}
\subsubsection{输出格式}
输出文件名为mergesort.out, 输出共一行.
第一行包含n个整数, 为排序后的升序序列, 每两个整数之间用空格隔开.
\subsubsection{输入输出样例}
\label{sec:iosample}
\begin{table}[h!]
	\centering
	\begin{tabular}{|l|l|}
		\hline
		mergesort.in & mergesort.out \\
		\hline
		5            & 5 9 11 12 22  \\
		9 11 5 22 12 & ~             \\
		\hline
	\end{tabular}
\end{table}
\subsubsection{数据范围}
$ 0 < n \leq 10^6 $,
$ | n_i | \leq 10^8 $

\subsection{快速排序}
\subsubsection{题目描述}
对n个整数使用快速排序进行升序排列.
\subsubsection{输入格式}
输入文件名为quicksort.in, 输入共两行.
\begin{itemize}
	\item 第一行是一个正整数n.
	\item 第二行包含n个整数$n_i$, 每两个整数之间用空格隔开.
\end{itemize}
\subsubsection{输出格式}
输出文件名为quicksort.out, 输出共一行.
第一行包含n个整数, 为排序后的升序序列, 每两个整数之间用空格隔开.
\subsubsection{输入输出样例}
\begin{table}[h!]
	\centering
	\begin{tabular}{|l|l|}
		\hline
		quicksort.in & quicksort.out \\
		\hline
		5            & 5 9 11 12 22  \\
		9 11 5 22 12 & ~             \\
		\hline
	\end{tabular}
\end{table}
\subsubsection{数据范围}
$ 0 < n \leq 2\times 10^6 $,
$ | n_i | \leq 10^8 $
