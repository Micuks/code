Dijkstra算法分别使用C++与Rust进行编写, 在邻接表具体的实现方式上,
C++方法使用了链表,
而Rust版使用了二维可变长数组Vec<>.以通过对比输出结果来验证算法正确性.

\subsection{程序执行环境}
在以下环境下进行过测试.

\begin{description}
	\item[操作系统] macOS 13.1 22C5044e arm64;
	\item[C++编译器] g++ (Homebrew GCC 12.2.0) 12.2.0;
	\item[Rust编译器] rustc 1.65.0 (897e37553 2022-11-02)
	\item[构建和测试工具] GNU Make 3.81;
	\item[文档编译工具] XeTeX 3.141592653-2.6-0.999994 (TeX Live 2022)
	\item[文本编辑器] nvim v0.8.1
\end{description}

\subsection{程序运行方式}
\subsubsection{Dijkstra编码程序的编译, 运行和测试方法}
C++版本和Rust版本均使用 make进行统一构建和测试, 并使用diff命令将结果进行比较,
以确保排序结果正确.\par

程序(及文档)均使用make进行构建, 对于Huffman编码程序的构建, 在Huffman/文件夹下的shell中执行以下命令:
\begin{lstlisting}[language=bash]
make
\end{lstlisting}
二进制文件 Dijkstra和DijkstraRS 将被编译在../build/目录下, 其中,
Dijkstra为C++版的编译输出, 而DijkstraRS为Rust版的编译输出.
如果要运行样例测试, 可以在shell中执行这条命令:

\begin{lstlisting}[language=bash]
make default_test
\end{lstlisting}
程序输出将存储在data/dijkstra[\_rs].out中.\par

程序同样支持\textbf{debug模式}编译. 在debug模式下, 会输出更多的运行信息.
例如, 在默认测试中, 指定DEBUG变量如下:
\begin{lstlisting}[language=bash]
make default_test DEBUG=1
\end{lstlisting}

如果要自定义生成数据进行测试, 如下
\begin{lstlisting}[language=bash]
make test
\end{lstlisting}
默认情况下会使用数据生成器data\_gen生成范围在0到1000范围内的5000个数字进行测试.\par

自定义数据同样支持\textbf{debug模式}编译. 在debug模式下, 会输出更多的运行信息.
指定DEBUG变量如下:
\begin{lstlisting}[language=bash]
make test DEBUG=1
\end{lstlisting}

其中, dijkstra\_data\_gen生成的输入存储在data/dijkstra\_yasample.in中,
程序的输出存储在data/dijkstra\_yasample.out和data/dijkstra\_rs\_yasample.out中. \par

如果要使用数据生成器dijkstra\_data\_gen生成指定范围内的均匀分布的随机数进行测试, 可以用
这条命令:
\begin{lstlisting}[language=bash]
make test [from=<range start>] [to=<range end>] [amount=<number of random numbers>]
# e.g. 
# make test from=0 to=12345 amount=50 DEBUG=[01]
\end{lstlisting}

由于分别使用c++和rust实现了命令行参数处理工具, 因此如果要使用自定义文件进行测试,
可以用这条命令:
\begin{lstlisting}
# c++版
../build/Dijkstra --in ${infilename} --out [${outfilename}]
# Rust版
../build/DijkstraRS --in ${infilename} --out [${outfilename}]
\end{lstlisting}
其中, \$\{infilename\}是按第一章中的\textbf{输入格式}\ref{sec:iosample}组织的输入数据.
默认情况下将同时输出到标准输出和data/dijkstra_yasample.out, 可选自定义输出到
\$\{outfilename\}路径下. 输出格式为第一章中的\textbf{输出格式}\ref{sec:iosample}.

\subsection{程序执行示例}
\label{sec:dpBench}
此处演示程序从构建到测试的过程.
\begin{lstlisting}[language=bash]
# cwd = Dijkstra/ directory
# user input
make clean

# console output
rm -rf ../build/Dijkstra*

# user input
make # Compile Rust and C++ programs

# console output
g++ -c -Wall -std=c++17 -O3 -DNDEBUG src//Dijkstra.cpp -o ../build/dijkstra.o
g++ -c -Wall -std=c++17 -O3 -DNDEBUG src/utils/CLIParser.cpp -o ../build/CLIParser.o
g++ -Wall -std=c++17 -O3 -DNDEBUG ../build/CLIParser.o ../build/Dijkstra.o -o ../build/Dijkstra
rustc -O --edition 2021 --cfg 'feature=""' -o ../build/DijkstraRS src//main.rs

# user input
# Run default test with sample given in *DEBUG* mode
make clean default_test DEBUG=1
# or use `make clean huffman_default_test` to run in *RELEASE* mode

# data/dijkstra.in:
3 3
1 2 5
2 3 5
3 1 2

# console output
rm -rf ../build/Dijkstra*
rm -rf ../build/CLIParser.o
g++ -c -Wall -Werror -Wextra -std=c++17 -O0 -DDEBUG src/utils/CLIParser.cpp -o ../build/CLIParser.o
g++ -Wall -Werror -Wextra -std=c++17 -O0 -DDEBUG ../build/CLIParser.o ../build/Dijkstra.o -o ../build/Dijkstra
rustc -g --edition 2021 --cfg 'feature="debug"' -o ../build/DijkstraRS src//main.rs
../build/Dijkstra --in data/dijkstra.in --out data/dijkstra.out
data/dijkstra.in
data/dijkstra.out
[dijkstra] Time measured: 1e-06 seconds.
2
--------------------END OF C++ OUTPUT-----------------------
------------------------------------------------------------
-------------------BEGIN OF RUST OUTPUT---------------------
../build/DijkstraRS --in data/dijkstra.in --out data/dijkstra_rs.out
Load from file: data/dijkstra.in
[Dijkstra Rust] Time measured: 0.000010375 seconds.
2
Write to file: data/dijkstra_rs.out
--------------------END OF RUST OUTPUT----------------------
diff data/dijkstra.out data/dijkstra_rs.out

# user input
# test with random number generator
make clean test # Compile data_gen and generate sample to test

# console output
rm -rf ../build/Dijkstra*
rm -rf ../build/CLIParser.o
g++ -Wall -std=c++17 -O3 -DNDEBUG src/data_gen/dijkstra_data_gen.cpp -o ../build/dijkstra_data_gen
g++ -c -Wall -std=c++17 -O3 -DNDEBUG src/utils/CLIParser.cpp -o ../build/CLIParser.o
g++ -Wall -std=c++17 -O3 -DNDEBUG ../build/CLIParser.o ../build/Dijkstra.o -o ../build/Dijkstra
rustc -O --edition 2021 --cfg 'feature=""' -o ../build/DijkstraRS src//main.rs
[INFO]: Generate 5000 nodes.
../build/dijkstra_data_gen 0 1000 5000
../build/Dijkstra --in data/dijkstra_yasample.in --out data/dijkstra_yasample.out
data/dijkstra_yasample.in
data/dijkstra_yasample.out
[dijkstra] Time measured: 0.000709 seconds.
59066
--------------------END OF C++ OUTPUT-----------------------
------------------------------------------------------------
-------------------BEGIN OF RUST OUTPUT---------------------
../build/DijkstraRS --in data/dijkstra_yasample.in --out data/dijkstra_rs_yasample.out
Load from file: data/dijkstra_yasample.in
[Dijkstra Rust] Time measured: 0.000879208 seconds.
59066
Write to file: data/dijkstra_rs_yasample.out
--------------------END OF RUST OUTPUT----------------------
diff data/dijkstra_yasample.out data/dijkstra_rs_yasample.out

# user input
make test from=0 to=1000 amount=500
# It means generate 500 random numbers ranging from 0 to 1000.

# console output
rm -rf ../build/Dijkstra*
rm -rf ../build/CLIParser.o
g++ -c -Wall -std=c++17 -O3 -DNDEBUG src/utils/CLIParser.cpp -o ../build/CLIParser.o
g++ -Wall -std=c++17 -O3 -DNDEBUG ../build/CLIParser.o ../build/Dijkstra.o -o ../build/Dijkstra
rustc -O --edition 2021 --cfg 'feature=""' -o ../build/DijkstraRS src//main.rs
[INFO]: Generate 500 nodes.
../build/dijkstra_data_gen 0 1000 500
../build/Dijkstra --in data/dijkstra_yasample.in --out data/dijkstra_yasample.out
data/dijkstra_yasample.in
data/dijkstra_yasample.out
[dijkstra] Time measured: 4.1e-05 seconds.
4077
--------------------END OF C++ OUTPUT-----------------------
------------------------------------------------------------
-------------------BEGIN OF RUST OUTPUT---------------------
../build/DijkstraRS --in data/dijkstra_yasample.in --out data/dijkstra_rs_yasample.out
Load from file: data/dijkstra_yasample.in
[Dijkstra Rust] Time measured: 0.000005208 seconds.
4077
Write to file: data/dijkstra_rs_yasample.out
--------------------END OF RUST OUTPUT----------------------
diff data/dijkstra_yasample.out data/dijkstra_rs_yasample.out
\end{lstlisting}

