Huffman编码算法分别使用C++与Rust进行编写, 在具体的实现方式上,
C++方法使用了二叉最小堆,
而Rust版在每次插入操作结束后都对数组重新进行排序.以通过对比输出结果来验证算法正确性.

\subsection{程序执行环境}
在以下环境下进行过测试.

\begin{description}
	\item[操作系统] macOS 13.1 22C5044e arm64;
	\item[C++编译器] g++ (Homebrew GCC 12.2.0) 12.2.0;
	\item[Rust编译器] rustc 1.65.0 (897e37553 2022-11-02)
	\item[构建和测试工具] GNU Make 3.81;
	\item[文档编译工具] XeTeX 3.141592653-2.6-0.999994 (TeX Live 2022)
	\item[文本编辑器] nvim v0.8.1
\end{description}

\subsection{程序运行方式}
\subsubsection{Huffman编码程序的编译, 运行和测试方法}
C++版本和Rust版本均使用 make进行统一构建和测试, 并使用diff命令将结果进行比较,
以确保排序结果正确.\par

程序(及文档)均使用make进行构建, 对于Huffman编码程序的构建, 在Huffman/文件夹下的shell中执行以下命令:
\begin{lstlisting}[language=bash]
make
\end{lstlisting}
二进制文件 Huffman和HuffmanRS 将被编译在../build/目录下, 其中,
Huffman为C++版的编译输出, 而HuffmanRS为Rust版的编译输出.
如果要运行样例测试, 可以在shell中执行这条命令:
\begin{lstlisting}[language=bash]
make huffman_default_test
\end{lstlisting}
程序输出将存储在data/huffman/huffman[\_rs].out中.\par

如果要自定义生成数据进行测试, 如下
\begin{lstlisting}[language=bash]
make test
\end{lstlisting}
默认情况下会使用数据生成器data\_gen生成范围在0到1000范围内的5000个数字进行测试.\par

程序同样支持\textbf{debug模式}编译. 在debug模式下, 会输出更多的运行信息.
指定DEBUG变量如下:
\begin{lstlisting}[language=bash]
make test DEBUG=1
\end{lstlisting}

其中, huffman\_data\_gen生成的输入存储在data/huffman\_yasample.in中,
程序的输出存储在data/huffman\_yasample.out和data/huffman\_rs\_yasample.out中. \par

如果要使用数据生成器data\_gen生成指定范围内的均匀分布的随机数进行测试, 可以用
这条命令:
\begin{lstlisting}[language=bash]
make test [from=<range start>] [to=<range end>] [amount=<number of random numbers>]
# e.g. 
# make test from=0 to=12345 amount=50
\end{lstlisting}

由于分别使用c++和rust实现了命令行参数处理工具, 因此如果要使用自定义文件进行测试,
可以用这条命令:
\begin{lstlisting}
# c++版
../build/Huffman --in ${infilename} --out [${outfilename}]
# Rust版
../build/HuffmanRS --in ${infilename} --out [${outfilename}]
\end{lstlisting}
其中, \$\{infilename\}是按第一章中的\textbf{输入格式}\ref{sec:iosample}组织的输入数据.
默认情况下将同时输出到标准输出和data/huffman\_yasample.out, 可选自定义输出到
\$\{outfilename\}路径下. 输出格式为第一章中的\textbf{输出格式}\ref{sec:iosample}.

\subsubsection{文档编译}
除排序程序使用make管理之外, 本实验的课程报告使用latex进行编写, 因此同样可以
使用make进行管理. 如果要编译文档, 在Greedy/目录下, 运行这条命令:
\begin{lstlisting}[language=bash]
# cwd = Greddy/
make docs
\end{lstlisting}
make将调用xelatex进行编译; 为了确保cross-referencing的正确工作, make将执行两遍
xelatex编译.\par
编译生成的中间文件及文档都保存在docs/build下, docs/report.pdf和docs/build/
report.pdf均是生成的实验报告.

\subsection{程序执行示例}
\label{sec:dpBench}
此处演示程序从构建到测试的过程.
\begin{lstlisting}[language=bash]
# in Huffman/ directory
# user input
make clean

# console output
rm -rf ../build/Huffman*

# user input
make # Compile Rust and C++ programs

# console output
g++ -c -Wall -std=c++17 -O3 -DNDEBUG src/Huffman.cpp -o ../build/Huffman.o
g++ -Wall -std=c++17 -O3 -DNDEBUG ../build/CLIParser.o ../build/Huffman.o -o ../build/Huffman
rustc -O --edition 2021 --cfg 'feature=""' -o ../build/HuffmanRS src/main.rs

# user input
# Run default test with sample given in *DEBUG* mode
make clean huffman_default_test DEBUG=1
# or use `make clean huffman_default_test` to run in *RELEASE* mode

# data/huffman/huffman.in:
4
0.100 0.100 0.200 0.600

# console output
rm -rf ../build/Huffman*
g++ -c -Wall -Werror -Wextra -std=c++17 -O0 -DDEBUG src/Huffman.cpp -o ../build/Huffman.o
g++ -Wall -Werror -Wextra -std=c++17 -O0 -DDEBUG ../build/CLIParser.o ../build/Huffman.o -o ../build/Huffman
rustc -g --edition 2021 --cfg 'feature="debug"' -o ../build/HuffmanRS src/main.rs
../build/Huffman --in data/huffman/huffman.in --out data/huffman/huffman.out
data/huffman/huffman.in
data/huffman/huffman.out
[Huffman] Time measured: 5e-06 seconds.
Expectation:
1.600
 (2, 0.2): 00
 (0, 0.1): 010
 (1, 0.1): 011
 (3, 0.6): 1
--------------------END OF C++ OUTPUT-----------------------
------------------------------------------------------------
-------------------BEGIN OF RUST OUTPUT---------------------
../build/HuffmanRS --in data/huffman/huffman.in --out data/huffman/huffman_rs.out
Load from file: data/huffman/huffman.in
(1, 0.1): 000
(0, 0.1): 001
(2, 0.2): 01
(3, 0.6): 1

[Huffman RUST] Time measured: 0.00018425 seconds.
Expectation:
1.600
Write to file: data/huffman/huffman_rs.out
--------------------END OF RUST OUTPUT----------------------
diff data/huffman/huffman.out data/huffman/huffman_rs.out

# user input
# test with random number generator
make clean test # Compile data_gen and generate sample to test

# console output
rm -rf ../build/Huffman*
g++ -c -Wall -std=c++17 -O3 -DNDEBUG src/Huffman.cpp -o ../build/Huffman.o
g++ -Wall -std=c++17 -O3 -DNDEBUG ../build/CLIParser.o ../build/Huffman.o -o ../build/Huffman
rustc -O --edition 2021 --cfg 'feature=""' -o ../build/HuffmanRS src/main.rs
[INFO]: Generate 5000 random frequences sum to 1.
../build/huffman_data_gen 0 1000 5000
../build/Huffman --in data/huffman_yasample.in --out data/huffman_yasample.out
data/huffman_yasample.in
data/huffman_yasample.out
[Huffman] Time measured: 0.00162 seconds.
Expectation:
12.045
--------------------END OF C++ OUTPUT-----------------------
------------------------------------------------------------
-------------------BEGIN OF RUST OUTPUT---------------------
../build/HuffmanRS --in data/huffman_yasample.in --out data/huffman_rs_yasample.out
Load from file: data/huffman_yasample.in
[Huffman RUST] Time measured: 0.036416875 seconds.
Expectation:
12.045
Write to file: data/huffman_rs_yasample.out
--------------------END OF RUST OUTPUT----------------------
diff data/huffman_yasample.out data/huffman_rs_yasample.out

# user input
make test from=0 to=1000 amount=500
# It means generate 500 random numbers ranging from 0 to 1000.

# console output
[INFO]: Generate 500 random frequences sum to 1.
../build/huffman_data_gen 0 1000 500
../build/Huffman --in data/huffman_yasample.in --out data/huffman_yasample.out
data/huffman_yasample.in
data/huffman_yasample.out
[Huffman] Time measured: 0.00019 seconds.
Expectation:
8.729
--------------------END OF C++ OUTPUT-----------------------
------------------------------------------------------------
-------------------BEGIN OF RUST OUTPUT---------------------
../build/HuffmanRS --in data/huffman_yasample.in --out data/huffman_rs_yasample.out
Load from file: data/huffman_yasample.in
[Huffman RUST] Time measured: 0.000185541 seconds.
Expectation:
8.729
Write to file: data/huffman_rs_yasample.out
--------------------END OF RUST OUTPUT----------------------
diff data/huffman_yasample.out data/huffman_rs_yasample.out

\end{lstlisting}
