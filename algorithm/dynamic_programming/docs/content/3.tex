对一般方法和应用跳跃点优化的动态规划解0-1背包问题的程序执行环境和
运行方式均相同, 包括其测试结果都在此一并介绍.

\subsection{程序执行环境}
在以下环境下进行过测试.

\begin{description}
	\item[操作系统] macOS 13.1 22C5044e arm64;
	\item[编译器] clang-1400.0.29.102;
	\item[构建和测试工具] GNU Make 3.81;
	\item[文档编译工具] XeTeX 3.141592653-2.6-0.999994 (TeX Live 2022)
	\item[文本编辑器] nvim v0.8.0
\end{description}

\subsection{程序运行方式}
\subsubsection{排序程序的编译, 运行和测试方法}
一般方法和应用跳跃点优化的动态规划解0-1背包问题都采用
make进行统一构建和测试, 并使用diff命令将结果进行比较,
以确保排序结果正确.\par

程序(及文档)均使用make进行构建, 对于排序程序的构建, 在shell中执行以下命令:
\begin{lstlisting}[language=bash]
make
\end{lstlisting}
二进制文件 BackPack01, JumpPointBackPack 将被编译在build/目录下,
如果要运行默认测试, 可以在shell中执行这条命令:
\begin{lstlisting}[language=bash]
make test
\end{lstlisting}
默认情况下会使用数据生成器data\_gen生成范围在0到100000范围内的500个数字进行测试.
因为当数据太多的时候, 使用跳跃点方法进行计算的时候会发生溢出.\par

其中, data\_gen生成的输入存储在samples/yet\_another\_sample.in中,
程序的输出存储在samples/bag.out和samples/jump\_bag.out中. \par

如果要使用数据生成器data\_gen生成指定范围内的均匀分布的随机数进行测试, 可以用
这条命令:
\begin{lstlisting}[language=bash]
make test [from=<range start>] [to=<range end>] [amount=<number of random numbers>]
# e.g. 
# make test from=0 to=12345 amount=50
\end{lstlisting}

如果要使用自定义文件进行测试, 可以用这条命令:
\begin{lstlisting}
# 一般的0-1背包动态规划解法
build/BackPack01 ${infilename} [${outfilename}]
# 跳跃点优化的0-1背包动态规划解法
build/JumpPointBackPack ${infilename} [${outfilename}]
\end{lstlisting}
其中, \$\{infilename\}是按第一章中的\textbf{输入格式}\ref{sec:iosample}组织的输入数据.
默认情况下将同时输出到标准输出和samples/mergesort.out, 可选自定义输出到
\$\{outfilename\}路径下. 输出格式为第一章中的\textbf{输出格式}\ref{sec:iosample}.

\subsubsection{文档编译}
除排序程序使用make管理之外, 本实验的课程报告使用latex进行编写, 因此同样可以
使用make进行管理. 如果要编译文档, 运行这条命令:
\begin{lstlisting}[language=bash]
make docs
\end{lstlisting}
make将调用xelatex进行编译; 为了确保cross-referencing的正确工作, make将执行两遍
xelatex编译.\par
编译生成的中间文件及文档都保存在docs/build下, docs/report.pdf和docs/build/
report.pdf均是生成的实验报告.

\subsection{程序执行示例}
\label{sec:dpBench}
此处演示程序从构建到测试的过程.
\begin{lstlisting}[language=bash]
# user input
make clean

# console output
rm -rf build/*
/Library/Developer/CommandLineTools/usr/bin/make -C docs/ clean
rm -rf build/*

# user input
make # Compile backpack programs

# console output
clang++ -c -g -Wall -std=c++17 -O3 01bp/BackPack01.cpp -o build/BackPack01.o
clang++ -Wall -std=c++17 -O3 build/BackPack01.o -o build/BackPack01
clang++ -c -g -Wall -std=c++17 -O3 01bp_optimized/JumpPointBackPack.cpp -o build/JumpPointBackPack.o
clang++ -Wall -std=c++17 -O3 build/JumpPointBackPack.o -o build/JumpPointBackPack

# user input
make test_default # Run default test with sample given with assignment

# samples/bag.in:
70 3
71 100
69 1
1 2

# console output
build/BackPack01
[BackPack01DP] Time measured: 1.1083e-05 seconds.
3
build/JumpPointBackPack
[JumpPointBackPack] Time measured: 7.083e-06 seconds.
3

# user input
# test with random number generator
make test # Compile data_gen and generate sample to test

# console output
clang++ -Wall -std=c++17 -O3  data_gen/data_gen.cpp -o build/data_gen
echo "[INFO]: Generate 500 random numbers ranging from 0 to 1000."
[INFO]: Generate 500 random numbers ranging from 0 to 1000.
build/data_gen 0 1000 500
build/BackPack01 samples/yet_another_sample.in
[BackPack01DP] Time measured: 0.00201904 seconds. # 一般方法动态规划的求解时间
12204
build/JumpPointBackPack samples/yet_another_sample.in
[JumpPointBackPack] Time measured: 0.000105875 seconds. # 应用跳跃点优化的动态规划求解时间
12204
diff samples/bag.out samples/jump_bag.out

# user input
make test from=0 to=1000 amount=100
# It means generate 100 random numbers ranging from 0 to 1000.

# console output
echo "[INFO]: Generate 100 random numbers ranging from 0 to 1000."
[INFO]: Generate 100 random numbers ranging from 0 to 1000.
build/data_gen 0 1000 100
build/BackPack01 samples/yet_another_sample.in
[BackPack01DP] Time measured: 0.00020725 seconds.
6778
build/JumpPointBackPack samples/yet_another_sample.in
[JumpPointBackPack] Time measured: 2.2166e-05 seconds.
6778
diff samples/bag.out samples/jump_bag.out

\end{lstlisting}

通过上述的例子, 可以看出, 当所给数据的数量级较大的时候,
使用跳跃点优化的动态规划算法要平均比一般方法快出2个数量级.
