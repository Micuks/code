\documentclass{beamer}
\usepackage{xeCJK}
\usepackage[T1]{fontenc}
\usepackage{ifplatform}
\usepackage{mathptmx}
\usepackage{amsmath}
\usepackage{amsfonts}
\usepackage{chemformula}
\usepackage{cite}
\usepackage{graphicx}
\usepackage{hyperref}
\usepackage{minted}

\defaultfontfeatures{Mapping=tex-text, Scale=MatchLowercase}
\setmainfont{Times}
\setmonofont{Times}

% Avoid ^I in minted output
\setminted{tabsize=4}


% macOS font setting
\ifmacosx
	\setCJKmainfont{Songti SC}
	\setCJKmonofont{Songti SC}
\else
% Linux and other OS font setting
	\setCJKmainfont{Source Han Sans CN}
	\setCJKmonofont{Source Han Sans CN}
\fi

% Theme
% \usetheme{metropolis}

% Information to be included in the title page:
\title{模型介绍}
\author{吴清柳}
\institute{Beijing University of Posts and Telecommunications}
\date{\today}

\begin{document}
\AtBeginSection{\tableofcontents[currentsection]}
% Title page
\maketitle

% Table of contents
\begin{frame}
	\frametitle{Table of Contentes}
	\tableofcontents
\end{frame}

\section{概述}
\begin{frame}
	\frametitle{Overview}
	通过该ppt, 对9个实验使用的模型进行介绍. 九个实验分别是鸢尾花分类实验,emojify人
	脸表情识别实验, 贷款预测, 房价预测, MNIST手写数字识别, 股票价格预测, 泰坦尼克号
	生还预测, 红酒质量预测和假新闻预测.

	使用的模型有机器学习方法如决策树, 线性回归, XGBoost等; 以及深度学习方法如全连接
	网络, CNN等.
\end{frame}

\section{鸢尾花分类实验}
\begin{frame}
	\frametitle{鸢尾花分类实验}
	\begin{block}{代码和数据集}
		该论文的复现代码在\href{这里}{https://github.com/Micuks/code/blob/master/gnn/1\_iris/iris-classification/train.ipynb}
		数据集使用了鸢尾花分类数据集.
	\end{block}

	\begin{block}{模型信息}
		模型使用PyTorch编写, 为三层的全连接网络. 前两层使用ReLU激活函数, 第三层为完成分
		类功能, 使用Softmax作为激活函数, 输出分类结果.
		\begin{itemize}
			\item 第一层全连接层输入特征数量为4, 输出特征数量为25;
			\item 第二层全连接层输入特征数量为25, 输出特征数量为30;
			\item 第三层全连接层输入特征数量为30, 输出特征数量为3, 对应三类鸢尾花;
		\end{itemize}
	\end{block}
\end{frame}

\begin{frame}[fragile]
	\frametitle{模型信息}
	模型搭建代码如下.
	\begin{minted}[breaklines]{python}
class Model(nn.Module):
def __init__(self, input_feats=4, hidden_layer1=25, hidden_layer2=30, output_feats=3) -> None:
	super().__init__()
	self.fc1 = nn.Linear(input_feats, hidden_layer1)
	self.fc2=nn.Linear(hidden_layer1, hidden_layer2)
	self.out=nn.Linear(hidden_layer2, output_feats)

def forward(self, x):
	x=F.relu(self.fc1(x))
	x=F.relu(self.fc2(x))
	x=self.out(x)

	return x
	\end{minted}
\end{frame}

\subsection{模型介绍}
\begin{frame}
	\frametitle{模型介绍}

	输入数据为鸢尾花(Iris)的特征数据. 鸢尾花有三种, 分别是Setosa, Versicolour,
	Virginica. 每个都有4个特征: sepal length, sepal width, 以及petal width.

	模型的结构介绍如下.
	\begin{block}{模型结构}
		\begin{enumerate}
			\item \textbf{输入层} 神经元数量为4, 代表4个输入特征.
			\item \textbf{第一个隐藏层`fc1'} 全连接层(dense layer), 25个神经元, 激活函数为
			      ReLU, 当输入为正时输出输入本身, 否则输出0. 如果没有非线性激活函数, 则模型将等效
			      为线性拟合函数, 拟合性能下降.
			\item \textbf{第二个隐藏层`fc2'} 另一个全连接层, 30个神经元, 激活函数为ReLU;
			\item \textbf{输出层`out'} 一个全连接层, 3个神经元, 对应鸢尾花的三个分类:
			      Setosa, Versicolour, Virginica.
		\end{enumerate}
	\end{block}

\end{frame}

\subsection{训练方法}
\begin{frame}[fragile]
	\frametitle{训练方法}
	使用交叉熵作为损失函数, Adam为优化器, 学习率为0.01, 在训练集上进行100轮训练. 对loss可视化观察后看到大约40轮后模型已经接近收敛.
	\begin{block}{训练代码}
		\begin{minted}[breaklines=true]{python}
epochs=100
losses=[]
for i in range(epochs):
    y_pred=model.forward(X_train)
    loss=criterion(y_pred,y_train)
    losses.append(loss)
    print(f'epoch: {i:2} loss: {loss.item():10.8f}')
    
    optimizer.zero_grad()
    loss.backward()
    optimizer.step()
	\end{minted}
	\end{block}
\end{frame}

% Slide 1: Introduction to Training Duration and Loss Storage
\begin{frame}[fragile]
	\frametitle{初始化}
	\begin{itemize}
		\item 一轮代表对所有训练样本的一个完全的前向传播和反向传播过程;
		\item \texttt{losses}列表会存储在每轮中计算的loss值, 用于追踪模型的表现;
	\end{itemize}
	\begin{minted}[breaklines]{python}
		epochs = 100
		losses = []
		\end{minted}
\end{frame}

% Slide 2: The Training Loop - Forward Pass and Loss Calculation
\begin{frame}[fragile]
	\frametitle{训练循环: 前向传播和损失计算}
	Within the loop, the model predicts outcomes and computes the discrepancy between predictions and actual values.
	在训练循环中, 模型预测输出, 计算预测值和实际值之间的差异;

	\begin{minted}[breaklines]{python}
for i in range(epochs):
    y_pred = model.forward(X_train)
    loss = criterion(y_pred, y_train)
    losses.append(loss)
\end{minted}
\end{frame}

% Slide 3: The Training Loop - Logging, Backward Pass, and Optimization
\begin{frame}[fragile]
	\frametitle{训练循环: 日志和优化}
	\begin{itemize}
		\item 通过打印轮数和loss值来监控训练进程;
		\item 计算梯度并更新模型参数来降低loss;
	\end{itemize}

	\begin{minted}[breaklines]{python}
    print(f'epoch: {i:2} loss: {loss.item():10.8f}')
    optimizer.zero_grad()
    loss.backward()
    optimizer.step()
\end{minted}
\end{frame}

\section{Emojify表情识别实验}
\begin{frame}
	\frametitle{利用CNN进行面部表情识别}

	在该实验中, 利用Tensorflow来搭建CNN网络进行对摄像头采集的人脸进行表情识别. 模型使用
	GPU进行训练, 得到的权重文件作为表情识别的依据.

	此处使用Tensorflow是为了将Pytorch和Tensorflow两个常用框架都进行学习.

\end{frame}

\subsection{模型介绍}
% Slide 1: Introduction to the Model
\begin{frame}[fragile]
	\frametitle{CNN Model Structure}
	\begin{itemize}
		\item Purpose: Emotion recognition from grayscale images.
		\item Image input shape: \(48 \times 48 \times 1\).
		\item Layers are stacked sequentially.
	\end{itemize}
	\begin{minted}[breaklines]{python}
emotion_model = Sequential()
\end{minted}
\end{frame}

% Slide 2: Initial Convolution Layers
\begin{frame}[fragile]
	\frametitle{Initial Convolution Layers}
	\begin{minted}[breaklines]{python}
emotion_model.add(
    Conv2D(32, kernel_size=(3, 3), activation="relu", input_shape=(48, 48, 1))
)
emotion_model.add(Conv2D(64, kernel_size=(3, 3), activation="relu"))
emotion_model.add(MaxPooling2D(pool_size=(2, 2)))
emotion_model.add(Dropout(0.25))
\end{minted}
\end{frame}

% Slide 3: Deeper Convolution Layers
\begin{frame}[fragile]
	\frametitle{Deeper Convolution Layers}
	\begin{minted}[breaklines]{python}
emotion_model.add(Conv2D(128, kernel_size=(3, 3), activation="relu"))
emotion_model.add(MaxPooling2D(pool_size=(2, 2)))
emotion_model.add(Conv2D(128, kernel_size=(3, 3), activation="relu"))
emotion_model.add(MaxPooling2D(pool_size=(2, 2)))
emotion_model.add(Dropout(0.25))
\end{minted}
\end{frame}

% Slide 4: Fully Connected Layers
\begin{frame}[fragile]
	\frametitle{Fully Connected Layers}
	\begin{minted}[breaklines]{python}
emotion_model.add(Flatten())
emotion_model.add(Dense(1024, activation="relu"))
emotion_model.add(Dropout(0.25))
emotion_model.add(Dense(7, activation="softmax"))
\end{minted}
\end{frame}



\end{document}
