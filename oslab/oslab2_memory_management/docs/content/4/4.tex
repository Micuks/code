\label{ssub:实验心得}
通过这次实验, 我实践并测试了基本内存分配和虚拟内存分配管理, 操作系统的内存分配和
管理有了更加直观和亲身的认识. 通过基本内存分配, 比较了First Fit和Best Fit内存分
配方法的异同优劣; 通过虚拟内存分配管理实验, 亲身体验了TLB, 页表和内存, 以及二级
存储的协作处理虚拟地址的过程. 通过虚拟内存转换, 有利于更加有效的使用物理内存, 且
对程序屏蔽了物理内存的细节, 也更加方便了程序的编写工作.\par

在构建和管理项目的工具方面, 使用了make工具, 方便的实现了
对程序的构建和测试; 此外, 对于文档, 也使用了make工具作为管理,
很大程度上方便了实验报告修改后的再次编译. 在实验报告编写方面,
练习了latex的使用, 对latex语法和使用更加熟悉.\par

但是, 由于时间限制, 没有来得及添加更多内容, 以及进行更多测试. 例如, 在虚拟内存分
配中, 没有来得及实现多个进程的虚拟内存空间. 但是即便如此, 也已经有了丰富的收获.

总的来说, 这次实验在内存分配和管理之内及之外都学到了许多知识, 收获颇丰.
% subsection 实验心得 (end)
