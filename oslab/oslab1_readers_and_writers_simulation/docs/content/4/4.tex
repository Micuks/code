\label{ssub:实验心得}
通过这次实验, 我亲手实现并测试了读者优先进程调度和写者优先进程调度, 对信号量机制
来控制并行进程对关键区的访问有了更深刻的认识.

但是, 由于时间限制, 没有来得及进行更多更详细的测试, 例如, 从10\%读者到90\%读者每
个进行2000个进程运行的读者优先和写者优先策略测试等. 但是即便如此, 也已经有了丰富的收获.

在构建和管理项目的工具方面, 使用了make工具, 方便的实现了
对程序的构建和测试; 此外, 对于文档, 也使用了make工具作为管理,
很大程度上方便了实验报告修改后的再次编译. 在实验报告编写方面,
练习了latex的使用, 对latex语法和使用更加熟悉.\par

但是, 由于时间有限, 没有实现更多的优化内容, 比如Fibonacci Heap代替BinaryHeap等.\par

总的来说, 这次实验在动态规划算法之内及之外都学到了许多知识, 收获颇丰.
% subsection 实验心得 (end)
