使用信号量机制,编写程序,模拟多任务下的读者优先和写者优先,能够同时稳定运行多个
任务 20分钟以上。 \\

编写程序模拟了多个进程同时想要访问存储在共享内存中的数据的情况下, 使用读者有限和
写者优先管理信号量对其进行调度的过程. 下面介绍本实验中对读者优先和写者优先的定义.

\subsection{读者优先和写者优先的共同定义}
此处介绍读者优先和写者优先情况下设定的共同部分.\\
同时只能有一个写者正在访问同一块共享内存; 同时可以有多个读者正在访问同一块共享内
存. \\

当一个或多个读者正在访问同一块共享内存的时候, 如果队列里新增等待访问同一块共享内
存的读者, 则他被允许访问, 如果队列里新增等待访问同一块共享内存的写者, 则他被阻挡
在外, 直到所有读者离开. \\

当一个写者正在访问同一块共享内存的时候, 后面的读者或者写者都等待. 如果先来的是写
者, 则写者是下一个访问共享内存的进程; 如果先来的是读者, 则读者是下一个访问共享内
存的进程, 之后写者的等待和上一段所说相同.

\subsection{读者优先}
在读者优先的情况下, 当读者正在访问同一块共享内存的时候, 写者想要访问这一块共享内
存, 当且仅当访问同一块共享内存的读者都访问完毕, 且没有其他读者在等待访问这块共享
内存; 在其他情况下都是读者优先访问同一块共享内存.\\

当写者正在访问同一块共享内存的时候, 如果等待队列中有读者, 则下一个访问的进程将是
读者, 且直到等待队列中不再有读者的时候, 等待队列里的写者才能访问同一块共享内存.
因此如果不断有读者前来访问同一块共享内存, 会导致写者饥饿.

\subsection{写者优先}
在写者优先的情况下, 如果一个或多个读者正在访问同一块共享内存, 新来的读者和写者在
关键区外排队, 则在第一个排队的写者之前的读者能正常进入关键区来共同访问共享内存.
当第一个排队的写者前面的读者都完成对共享内存的访问后, 写者将阻挡读者在关键区之
外, 直到队列中所有的写者依次完成写操作后才会让队列中的读者进入关键区. 因此, 如果
队列中不断有写者进入, 则会导致读者的饥饿. \\

\subsection{避免饥饿的限制}
为了避免读者和写者的饥饿导致实验在运行一段时间后变成只读或者只写, 我在实验程序中
以宏的形式指定了每个读者或写者最多访问一块共享内存的次数. 当达到最大访问次数后,
进程将在日志中记录自己的访问等待时间等信息, 并销毁自己. 日志记录在logs/file.log
和logs/stat.log中. 其中, logs/file.log详细记录了各进程的读者等待时间和写者等待时
间.
