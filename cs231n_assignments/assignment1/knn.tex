\documentclass[11pt]{article}

    \usepackage[breakable]{tcolorbox}
    \usepackage{parskip} % Stop auto-indenting (to mimic markdown behaviour)
    \usepackage{xeCJK}
    \setCJKmainfont{Adobe Song Std}
    \setCJKsansfont{Adobe Song Std}
    \setCJKmonofont{Adobe Song Std}
    

    % Basic figure setup, for now with no caption control since it's done
    % automatically by Pandoc (which extracts ![](path) syntax from Markdown).
    \usepackage{graphicx}
    % Maintain compatibility with old templates. Remove in nbconvert 6.0
    \let\Oldincludegraphics\includegraphics
    % Ensure that by default, figures have no caption (until we provide a
    % proper Figure object with a Caption API and a way to capture that
    % in the conversion process - todo).
    \usepackage{caption}
    \DeclareCaptionFormat{nocaption}{}
    \captionsetup{format=nocaption,aboveskip=0pt,belowskip=0pt}

    \usepackage{float}
    \floatplacement{figure}{H} % forces figures to be placed at the correct location
    \usepackage{xcolor} % Allow colors to be defined
    \usepackage{enumerate} % Needed for markdown enumerations to work
    \usepackage{geometry} % Used to adjust the document margins
    \usepackage{amsmath} % Equations
    \usepackage{amssymb} % Equations
    \usepackage{textcomp} % defines textquotesingle
    % Hack from http://tex.stackexchange.com/a/47451/13684:
    \AtBeginDocument{%
        \def\PYZsq{\textquotesingle}% Upright quotes in Pygmentized code
    }
    \usepackage{upquote} % Upright quotes for verbatim code
    \usepackage{eurosym} % defines \euro

    \usepackage{iftex}
    \ifPDFTeX
        \usepackage[T1]{fontenc}
        \IfFileExists{alphabeta.sty}{
              \usepackage{alphabeta}
          }{
              \usepackage[mathletters]{ucs}
              \usepackage[utf8x]{inputenc}
          }
    \else
        \usepackage{fontspec}
        \usepackage{unicode-math}
    \fi

    \usepackage{fancyvrb} % verbatim replacement that allows latex
    \usepackage{grffile} % extends the file name processing of package graphics
                         % to support a larger range
    \makeatletter % fix for old versions of grffile with XeLaTeX
    \@ifpackagelater{grffile}{2019/11/01}
    {
      % Do nothing on new versions
    }
    {
      \def\Gread@@xetex#1{%
        \IfFileExists{"\Gin@base".bb}%
        {\Gread@eps{\Gin@base.bb}}%
        {\Gread@@xetex@aux#1}%
      }
    }
    \makeatother
    \usepackage[Export]{adjustbox} % Used to constrain images to a maximum size
    \adjustboxset{max size={0.9\linewidth}{0.9\paperheight}}

    % The hyperref package gives us a pdf with properly built
    % internal navigation ('pdf bookmarks' for the table of contents,
    % internal cross-reference links, web links for URLs, etc.)
    \usepackage{hyperref}
    % The default LaTeX title has an obnoxious amount of whitespace. By default,
    % titling removes some of it. It also provides customization options.
    \usepackage{titling}
    \usepackage{longtable} % longtable support required by pandoc >1.10
    \usepackage{booktabs}  % table support for pandoc > 1.12.2
    \usepackage{array}     % table support for pandoc >= 2.11.3
    \usepackage{calc}      % table minipage width calculation for pandoc >= 2.11.1
    \usepackage[inline]{enumitem} % IRkernel/repr support (it uses the enumerate* environment)
    \usepackage[normalem]{ulem} % ulem is needed to support strikethroughs (\sout)
                                % normalem makes italics be italics, not underlines
    \usepackage{mathrsfs}
    

    
    % Colors for the hyperref package
    \definecolor{urlcolor}{rgb}{0,.145,.698}
    \definecolor{linkcolor}{rgb}{.71,0.21,0.01}
    \definecolor{citecolor}{rgb}{.12,.54,.11}

    % ANSI colors
    \definecolor{ansi-black}{HTML}{3E424D}
    \definecolor{ansi-black-intense}{HTML}{282C36}
    \definecolor{ansi-red}{HTML}{E75C58}
    \definecolor{ansi-red-intense}{HTML}{B22B31}
    \definecolor{ansi-green}{HTML}{00A250}
    \definecolor{ansi-green-intense}{HTML}{007427}
    \definecolor{ansi-yellow}{HTML}{DDB62B}
    \definecolor{ansi-yellow-intense}{HTML}{B27D12}
    \definecolor{ansi-blue}{HTML}{208FFB}
    \definecolor{ansi-blue-intense}{HTML}{0065CA}
    \definecolor{ansi-magenta}{HTML}{D160C4}
    \definecolor{ansi-magenta-intense}{HTML}{A03196}
    \definecolor{ansi-cyan}{HTML}{60C6C8}
    \definecolor{ansi-cyan-intense}{HTML}{258F8F}
    \definecolor{ansi-white}{HTML}{C5C1B4}
    \definecolor{ansi-white-intense}{HTML}{A1A6B2}
    \definecolor{ansi-default-inverse-fg}{HTML}{FFFFFF}
    \definecolor{ansi-default-inverse-bg}{HTML}{000000}

    % common color for the border for error outputs.
    \definecolor{outerrorbackground}{HTML}{FFDFDF}

    % commands and environments needed by pandoc snippets
    % extracted from the output of `pandoc -s`
    \providecommand{\tightlist}{%
      \setlength{\itemsep}{0pt}\setlength{\parskip}{0pt}}
    \DefineVerbatimEnvironment{Highlighting}{Verbatim}{commandchars=\\\{\}}
    % Add ',fontsize=\small' for more characters per line
    \newenvironment{Shaded}{}{}
    \newcommand{\KeywordTok}[1]{\textcolor[rgb]{0.00,0.44,0.13}{\textbf{{#1}}}}
    \newcommand{\DataTypeTok}[1]{\textcolor[rgb]{0.56,0.13,0.00}{{#1}}}
    \newcommand{\DecValTok}[1]{\textcolor[rgb]{0.25,0.63,0.44}{{#1}}}
    \newcommand{\BaseNTok}[1]{\textcolor[rgb]{0.25,0.63,0.44}{{#1}}}
    \newcommand{\FloatTok}[1]{\textcolor[rgb]{0.25,0.63,0.44}{{#1}}}
    \newcommand{\CharTok}[1]{\textcolor[rgb]{0.25,0.44,0.63}{{#1}}}
    \newcommand{\StringTok}[1]{\textcolor[rgb]{0.25,0.44,0.63}{{#1}}}
    \newcommand{\CommentTok}[1]{\textcolor[rgb]{0.38,0.63,0.69}{\textit{{#1}}}}
    \newcommand{\OtherTok}[1]{\textcolor[rgb]{0.00,0.44,0.13}{{#1}}}
    \newcommand{\AlertTok}[1]{\textcolor[rgb]{1.00,0.00,0.00}{\textbf{{#1}}}}
    \newcommand{\FunctionTok}[1]{\textcolor[rgb]{0.02,0.16,0.49}{{#1}}}
    \newcommand{\RegionMarkerTok}[1]{{#1}}
    \newcommand{\ErrorTok}[1]{\textcolor[rgb]{1.00,0.00,0.00}{\textbf{{#1}}}}
    \newcommand{\NormalTok}[1]{{#1}}

    % Additional commands for more recent versions of Pandoc
    \newcommand{\ConstantTok}[1]{\textcolor[rgb]{0.53,0.00,0.00}{{#1}}}
    \newcommand{\SpecialCharTok}[1]{\textcolor[rgb]{0.25,0.44,0.63}{{#1}}}
    \newcommand{\VerbatimStringTok}[1]{\textcolor[rgb]{0.25,0.44,0.63}{{#1}}}
    \newcommand{\SpecialStringTok}[1]{\textcolor[rgb]{0.73,0.40,0.53}{{#1}}}
    \newcommand{\ImportTok}[1]{{#1}}
    \newcommand{\DocumentationTok}[1]{\textcolor[rgb]{0.73,0.13,0.13}{\textit{{#1}}}}
    \newcommand{\AnnotationTok}[1]{\textcolor[rgb]{0.38,0.63,0.69}{\textbf{\textit{{#1}}}}}
    \newcommand{\CommentVarTok}[1]{\textcolor[rgb]{0.38,0.63,0.69}{\textbf{\textit{{#1}}}}}
    \newcommand{\VariableTok}[1]{\textcolor[rgb]{0.10,0.09,0.49}{{#1}}}
    \newcommand{\ControlFlowTok}[1]{\textcolor[rgb]{0.00,0.44,0.13}{\textbf{{#1}}}}
    \newcommand{\OperatorTok}[1]{\textcolor[rgb]{0.40,0.40,0.40}{{#1}}}
    \newcommand{\BuiltInTok}[1]{{#1}}
    \newcommand{\ExtensionTok}[1]{{#1}}
    \newcommand{\PreprocessorTok}[1]{\textcolor[rgb]{0.74,0.48,0.00}{{#1}}}
    \newcommand{\AttributeTok}[1]{\textcolor[rgb]{0.49,0.56,0.16}{{#1}}}
    \newcommand{\InformationTok}[1]{\textcolor[rgb]{0.38,0.63,0.69}{\textbf{\textit{{#1}}}}}
    \newcommand{\WarningTok}[1]{\textcolor[rgb]{0.38,0.63,0.69}{\textbf{\textit{{#1}}}}}


    % Define a nice break command that doesn't care if a line doesn't already
    % exist.
    \def\br{\hspace*{\fill} \\* }
    % Math Jax compatibility definitions
    \def\gt{>}
    \def\lt{<}
    \let\Oldtex\TeX
    \let\Oldlatex\LaTeX
    \renewcommand{\TeX}{\textrm{\Oldtex}}
    \renewcommand{\LaTeX}{\textrm{\Oldlatex}}
    % Document parameters
    % Document title
    \title{knn}
    
    
    
    
    
% Pygments definitions
\makeatletter
\def\PY@reset{\let\PY@it=\relax \let\PY@bf=\relax%
    \let\PY@ul=\relax \let\PY@tc=\relax%
    \let\PY@bc=\relax \let\PY@ff=\relax}
\def\PY@tok#1{\csname PY@tok@#1\endcsname}
\def\PY@toks#1+{\ifx\relax#1\empty\else%
    \PY@tok{#1}\expandafter\PY@toks\fi}
\def\PY@do#1{\PY@bc{\PY@tc{\PY@ul{%
    \PY@it{\PY@bf{\PY@ff{#1}}}}}}}
\def\PY#1#2{\PY@reset\PY@toks#1+\relax+\PY@do{#2}}

\@namedef{PY@tok@w}{\def\PY@tc##1{\textcolor[rgb]{0.73,0.73,0.73}{##1}}}
\@namedef{PY@tok@c}{\let\PY@it=\textit\def\PY@tc##1{\textcolor[rgb]{0.24,0.48,0.48}{##1}}}
\@namedef{PY@tok@cp}{\def\PY@tc##1{\textcolor[rgb]{0.61,0.40,0.00}{##1}}}
\@namedef{PY@tok@k}{\let\PY@bf=\textbf\def\PY@tc##1{\textcolor[rgb]{0.00,0.50,0.00}{##1}}}
\@namedef{PY@tok@kp}{\def\PY@tc##1{\textcolor[rgb]{0.00,0.50,0.00}{##1}}}
\@namedef{PY@tok@kt}{\def\PY@tc##1{\textcolor[rgb]{0.69,0.00,0.25}{##1}}}
\@namedef{PY@tok@o}{\def\PY@tc##1{\textcolor[rgb]{0.40,0.40,0.40}{##1}}}
\@namedef{PY@tok@ow}{\let\PY@bf=\textbf\def\PY@tc##1{\textcolor[rgb]{0.67,0.13,1.00}{##1}}}
\@namedef{PY@tok@nb}{\def\PY@tc##1{\textcolor[rgb]{0.00,0.50,0.00}{##1}}}
\@namedef{PY@tok@nf}{\def\PY@tc##1{\textcolor[rgb]{0.00,0.00,1.00}{##1}}}
\@namedef{PY@tok@nc}{\let\PY@bf=\textbf\def\PY@tc##1{\textcolor[rgb]{0.00,0.00,1.00}{##1}}}
\@namedef{PY@tok@nn}{\let\PY@bf=\textbf\def\PY@tc##1{\textcolor[rgb]{0.00,0.00,1.00}{##1}}}
\@namedef{PY@tok@ne}{\let\PY@bf=\textbf\def\PY@tc##1{\textcolor[rgb]{0.80,0.25,0.22}{##1}}}
\@namedef{PY@tok@nv}{\def\PY@tc##1{\textcolor[rgb]{0.10,0.09,0.49}{##1}}}
\@namedef{PY@tok@no}{\def\PY@tc##1{\textcolor[rgb]{0.53,0.00,0.00}{##1}}}
\@namedef{PY@tok@nl}{\def\PY@tc##1{\textcolor[rgb]{0.46,0.46,0.00}{##1}}}
\@namedef{PY@tok@ni}{\let\PY@bf=\textbf\def\PY@tc##1{\textcolor[rgb]{0.44,0.44,0.44}{##1}}}
\@namedef{PY@tok@na}{\def\PY@tc##1{\textcolor[rgb]{0.41,0.47,0.13}{##1}}}
\@namedef{PY@tok@nt}{\let\PY@bf=\textbf\def\PY@tc##1{\textcolor[rgb]{0.00,0.50,0.00}{##1}}}
\@namedef{PY@tok@nd}{\def\PY@tc##1{\textcolor[rgb]{0.67,0.13,1.00}{##1}}}
\@namedef{PY@tok@s}{\def\PY@tc##1{\textcolor[rgb]{0.73,0.13,0.13}{##1}}}
\@namedef{PY@tok@sd}{\let\PY@it=\textit\def\PY@tc##1{\textcolor[rgb]{0.73,0.13,0.13}{##1}}}
\@namedef{PY@tok@si}{\let\PY@bf=\textbf\def\PY@tc##1{\textcolor[rgb]{0.64,0.35,0.47}{##1}}}
\@namedef{PY@tok@se}{\let\PY@bf=\textbf\def\PY@tc##1{\textcolor[rgb]{0.67,0.36,0.12}{##1}}}
\@namedef{PY@tok@sr}{\def\PY@tc##1{\textcolor[rgb]{0.64,0.35,0.47}{##1}}}
\@namedef{PY@tok@ss}{\def\PY@tc##1{\textcolor[rgb]{0.10,0.09,0.49}{##1}}}
\@namedef{PY@tok@sx}{\def\PY@tc##1{\textcolor[rgb]{0.00,0.50,0.00}{##1}}}
\@namedef{PY@tok@m}{\def\PY@tc##1{\textcolor[rgb]{0.40,0.40,0.40}{##1}}}
\@namedef{PY@tok@gh}{\let\PY@bf=\textbf\def\PY@tc##1{\textcolor[rgb]{0.00,0.00,0.50}{##1}}}
\@namedef{PY@tok@gu}{\let\PY@bf=\textbf\def\PY@tc##1{\textcolor[rgb]{0.50,0.00,0.50}{##1}}}
\@namedef{PY@tok@gd}{\def\PY@tc##1{\textcolor[rgb]{0.63,0.00,0.00}{##1}}}
\@namedef{PY@tok@gi}{\def\PY@tc##1{\textcolor[rgb]{0.00,0.52,0.00}{##1}}}
\@namedef{PY@tok@gr}{\def\PY@tc##1{\textcolor[rgb]{0.89,0.00,0.00}{##1}}}
\@namedef{PY@tok@ge}{\let\PY@it=\textit}
\@namedef{PY@tok@gs}{\let\PY@bf=\textbf}
\@namedef{PY@tok@gp}{\let\PY@bf=\textbf\def\PY@tc##1{\textcolor[rgb]{0.00,0.00,0.50}{##1}}}
\@namedef{PY@tok@go}{\def\PY@tc##1{\textcolor[rgb]{0.44,0.44,0.44}{##1}}}
\@namedef{PY@tok@gt}{\def\PY@tc##1{\textcolor[rgb]{0.00,0.27,0.87}{##1}}}
\@namedef{PY@tok@err}{\def\PY@bc##1{{\setlength{\fboxsep}{\string -\fboxrule}\fcolorbox[rgb]{1.00,0.00,0.00}{1,1,1}{\strut ##1}}}}
\@namedef{PY@tok@kc}{\let\PY@bf=\textbf\def\PY@tc##1{\textcolor[rgb]{0.00,0.50,0.00}{##1}}}
\@namedef{PY@tok@kd}{\let\PY@bf=\textbf\def\PY@tc##1{\textcolor[rgb]{0.00,0.50,0.00}{##1}}}
\@namedef{PY@tok@kn}{\let\PY@bf=\textbf\def\PY@tc##1{\textcolor[rgb]{0.00,0.50,0.00}{##1}}}
\@namedef{PY@tok@kr}{\let\PY@bf=\textbf\def\PY@tc##1{\textcolor[rgb]{0.00,0.50,0.00}{##1}}}
\@namedef{PY@tok@bp}{\def\PY@tc##1{\textcolor[rgb]{0.00,0.50,0.00}{##1}}}
\@namedef{PY@tok@fm}{\def\PY@tc##1{\textcolor[rgb]{0.00,0.00,1.00}{##1}}}
\@namedef{PY@tok@vc}{\def\PY@tc##1{\textcolor[rgb]{0.10,0.09,0.49}{##1}}}
\@namedef{PY@tok@vg}{\def\PY@tc##1{\textcolor[rgb]{0.10,0.09,0.49}{##1}}}
\@namedef{PY@tok@vi}{\def\PY@tc##1{\textcolor[rgb]{0.10,0.09,0.49}{##1}}}
\@namedef{PY@tok@vm}{\def\PY@tc##1{\textcolor[rgb]{0.10,0.09,0.49}{##1}}}
\@namedef{PY@tok@sa}{\def\PY@tc##1{\textcolor[rgb]{0.73,0.13,0.13}{##1}}}
\@namedef{PY@tok@sb}{\def\PY@tc##1{\textcolor[rgb]{0.73,0.13,0.13}{##1}}}
\@namedef{PY@tok@sc}{\def\PY@tc##1{\textcolor[rgb]{0.73,0.13,0.13}{##1}}}
\@namedef{PY@tok@dl}{\def\PY@tc##1{\textcolor[rgb]{0.73,0.13,0.13}{##1}}}
\@namedef{PY@tok@s2}{\def\PY@tc##1{\textcolor[rgb]{0.73,0.13,0.13}{##1}}}
\@namedef{PY@tok@sh}{\def\PY@tc##1{\textcolor[rgb]{0.73,0.13,0.13}{##1}}}
\@namedef{PY@tok@s1}{\def\PY@tc##1{\textcolor[rgb]{0.73,0.13,0.13}{##1}}}
\@namedef{PY@tok@mb}{\def\PY@tc##1{\textcolor[rgb]{0.40,0.40,0.40}{##1}}}
\@namedef{PY@tok@mf}{\def\PY@tc##1{\textcolor[rgb]{0.40,0.40,0.40}{##1}}}
\@namedef{PY@tok@mh}{\def\PY@tc##1{\textcolor[rgb]{0.40,0.40,0.40}{##1}}}
\@namedef{PY@tok@mi}{\def\PY@tc##1{\textcolor[rgb]{0.40,0.40,0.40}{##1}}}
\@namedef{PY@tok@il}{\def\PY@tc##1{\textcolor[rgb]{0.40,0.40,0.40}{##1}}}
\@namedef{PY@tok@mo}{\def\PY@tc##1{\textcolor[rgb]{0.40,0.40,0.40}{##1}}}
\@namedef{PY@tok@ch}{\let\PY@it=\textit\def\PY@tc##1{\textcolor[rgb]{0.24,0.48,0.48}{##1}}}
\@namedef{PY@tok@cm}{\let\PY@it=\textit\def\PY@tc##1{\textcolor[rgb]{0.24,0.48,0.48}{##1}}}
\@namedef{PY@tok@cpf}{\let\PY@it=\textit\def\PY@tc##1{\textcolor[rgb]{0.24,0.48,0.48}{##1}}}
\@namedef{PY@tok@c1}{\let\PY@it=\textit\def\PY@tc##1{\textcolor[rgb]{0.24,0.48,0.48}{##1}}}
\@namedef{PY@tok@cs}{\let\PY@it=\textit\def\PY@tc##1{\textcolor[rgb]{0.24,0.48,0.48}{##1}}}

\def\PYZbs{\char`\\}
\def\PYZus{\char`\_}
\def\PYZob{\char`\{}
\def\PYZcb{\char`\}}
\def\PYZca{\char`\^}
\def\PYZam{\char`\&}
\def\PYZlt{\char`\<}
\def\PYZgt{\char`\>}
\def\PYZsh{\char`\#}
\def\PYZpc{\char`\%}
\def\PYZdl{\char`\$}
\def\PYZhy{\char`\-}
\def\PYZsq{\char`\'}
\def\PYZdq{\char`\"}
\def\PYZti{\char`\~}
% for compatibility with earlier versions
\def\PYZat{@}
\def\PYZlb{[}
\def\PYZrb{]}
\makeatother


    % For linebreaks inside Verbatim environment from package fancyvrb.
    \makeatletter
        \newbox\Wrappedcontinuationbox
        \newbox\Wrappedvisiblespacebox
        \newcommand*\Wrappedvisiblespace {\textcolor{red}{\textvisiblespace}}
        \newcommand*\Wrappedcontinuationsymbol {\textcolor{red}{\llap{\tiny$\m@th\hookrightarrow$}}}
        \newcommand*\Wrappedcontinuationindent {3ex }
        \newcommand*\Wrappedafterbreak {\kern\Wrappedcontinuationindent\copy\Wrappedcontinuationbox}
        % Take advantage of the already applied Pygments mark-up to insert
        % potential linebreaks for TeX processing.
        %        {, <, #, %, $, ' and ": go to next line.
        %        _, }, ^, &, >, - and ~: stay at end of broken line.
        % Use of \textquotesingle for straight quote.
        \newcommand*\Wrappedbreaksatspecials {%
            \def\PYGZus{\discretionary{\char`\_}{\Wrappedafterbreak}{\char`\_}}%
            \def\PYGZob{\discretionary{}{\Wrappedafterbreak\char`\{}{\char`\{}}%
            \def\PYGZcb{\discretionary{\char`\}}{\Wrappedafterbreak}{\char`\}}}%
            \def\PYGZca{\discretionary{\char`\^}{\Wrappedafterbreak}{\char`\^}}%
            \def\PYGZam{\discretionary{\char`\&}{\Wrappedafterbreak}{\char`\&}}%
            \def\PYGZlt{\discretionary{}{\Wrappedafterbreak\char`\<}{\char`\<}}%
            \def\PYGZgt{\discretionary{\char`\>}{\Wrappedafterbreak}{\char`\>}}%
            \def\PYGZsh{\discretionary{}{\Wrappedafterbreak\char`\#}{\char`\#}}%
            \def\PYGZpc{\discretionary{}{\Wrappedafterbreak\char`\%}{\char`\%}}%
            \def\PYGZdl{\discretionary{}{\Wrappedafterbreak\char`\$}{\char`\$}}%
            \def\PYGZhy{\discretionary{\char`\-}{\Wrappedafterbreak}{\char`\-}}%
            \def\PYGZsq{\discretionary{}{\Wrappedafterbreak\textquotesingle}{\textquotesingle}}%
            \def\PYGZdq{\discretionary{}{\Wrappedafterbreak\char`\"}{\char`\"}}%
            \def\PYGZti{\discretionary{\char`\~}{\Wrappedafterbreak}{\char`\~}}%
        }
        % Some characters . , ; ? ! / are not pygmentized.
        % This macro makes them "active" and they will insert potential linebreaks
        \newcommand*\Wrappedbreaksatpunct {%
            \lccode`\~`\.\lowercase{\def~}{\discretionary{\hbox{\char`\.}}{\Wrappedafterbreak}{\hbox{\char`\.}}}%
            \lccode`\~`\,\lowercase{\def~}{\discretionary{\hbox{\char`\,}}{\Wrappedafterbreak}{\hbox{\char`\,}}}%
            \lccode`\~`\;\lowercase{\def~}{\discretionary{\hbox{\char`\;}}{\Wrappedafterbreak}{\hbox{\char`\;}}}%
            \lccode`\~`\:\lowercase{\def~}{\discretionary{\hbox{\char`\:}}{\Wrappedafterbreak}{\hbox{\char`\:}}}%
            \lccode`\~`\?\lowercase{\def~}{\discretionary{\hbox{\char`\?}}{\Wrappedafterbreak}{\hbox{\char`\?}}}%
            \lccode`\~`\!\lowercase{\def~}{\discretionary{\hbox{\char`\!}}{\Wrappedafterbreak}{\hbox{\char`\!}}}%
            \lccode`\~`\/\lowercase{\def~}{\discretionary{\hbox{\char`\/}}{\Wrappedafterbreak}{\hbox{\char`\/}}}%
            \catcode`\.\active
            \catcode`\,\active
            \catcode`\;\active
            \catcode`\:\active
            \catcode`\?\active
            \catcode`\!\active
            \catcode`\/\active
            \lccode`\~`\~
        }
    \makeatother

    \let\OriginalVerbatim=\Verbatim
    \makeatletter
    \renewcommand{\Verbatim}[1][1]{%
        %\parskip\z@skip
        \sbox\Wrappedcontinuationbox {\Wrappedcontinuationsymbol}%
        \sbox\Wrappedvisiblespacebox {\FV@SetupFont\Wrappedvisiblespace}%
        \def\FancyVerbFormatLine ##1{\hsize\linewidth
            \vtop{\raggedright\hyphenpenalty\z@\exhyphenpenalty\z@
                \doublehyphendemerits\z@\finalhyphendemerits\z@
                \strut ##1\strut}%
        }%
        % If the linebreak is at a space, the latter will be displayed as visible
        % space at end of first line, and a continuation symbol starts next line.
        % Stretch/shrink are however usually zero for typewriter font.
        \def\FV@Space {%
            \nobreak\hskip\z@ plus\fontdimen3\font minus\fontdimen4\font
            \discretionary{\copy\Wrappedvisiblespacebox}{\Wrappedafterbreak}
            {\kern\fontdimen2\font}%
        }%

        % Allow breaks at special characters using \PYG... macros.
        \Wrappedbreaksatspecials
        % Breaks at punctuation characters . , ; ? ! and / need catcode=\active
        \OriginalVerbatim[#1,codes*=\Wrappedbreaksatpunct]%
    }
    \makeatother

    % Exact colors from NB
    \definecolor{incolor}{HTML}{303F9F}
    \definecolor{outcolor}{HTML}{D84315}
    \definecolor{cellborder}{HTML}{CFCFCF}
    \definecolor{cellbackground}{HTML}{F7F7F7}

    % prompt
    \makeatletter
    \newcommand{\boxspacing}{\kern\kvtcb@left@rule\kern\kvtcb@boxsep}
    \makeatother
    \newcommand{\prompt}[4]{
        {\ttfamily\llap{{\color{#2}[#3]:\hspace{3pt}#4}}\vspace{-\baselineskip}}
    }
    

    
    % Prevent overflowing lines due to hard-to-break entities
    \sloppy
    % Setup hyperref package
    \hypersetup{
      breaklinks=true,  % so long urls are correctly broken across lines
      colorlinks=true,
      urlcolor=urlcolor,
      linkcolor=linkcolor,
      citecolor=citecolor,
      }
    % Slightly bigger margins than the latex defaults
    
    \geometry{verbose,tmargin=1in,bmargin=1in,lmargin=1in,rmargin=1in}
    
    

\begin{document}
    
    \maketitle
    
    

    
    \begin{tcolorbox}[breakable, size=fbox, boxrule=1pt, pad at break*=1mm,colback=cellbackground, colframe=cellborder]
\prompt{In}{incolor}{26}{\boxspacing}
\begin{Verbatim}[commandchars=\\\{\}]
\PY{c+c1}{\PYZsh{} This mounts your Google Drive to the Colab VM.}
\PY{k+kn}{from} \PY{n+nn}{google}\PY{n+nn}{.}\PY{n+nn}{colab} \PY{k+kn}{import} \PY{n}{drive}
\PY{n}{drive}\PY{o}{.}\PY{n}{mount}\PY{p}{(}\PY{l+s+s1}{\PYZsq{}}\PY{l+s+s1}{/content/drive}\PY{l+s+s1}{\PYZsq{}}\PY{p}{)}

\PY{c+c1}{\PYZsh{} TODO: Enter the foldername in your Drive where you have saved the unzipped}
\PY{c+c1}{\PYZsh{} assignment folder, e.g. \PYZsq{}cs231n/assignments/assignment1/\PYZsq{}}
\PY{n}{FOLDERNAME} \PY{o}{=} \PY{l+s+s1}{\PYZsq{}}\PY{l+s+s1}{cs231n/assignments/assignment1/}\PY{l+s+s1}{\PYZsq{}}
\PY{k}{assert} \PY{n}{FOLDERNAME} \PY{o+ow}{is} \PY{o+ow}{not} \PY{k+kc}{None}\PY{p}{,} \PY{l+s+s2}{\PYZdq{}}\PY{l+s+s2}{[!] Enter the foldername.}\PY{l+s+s2}{\PYZdq{}}

\PY{c+c1}{\PYZsh{} Now that we\PYZsq{}ve mounted your Drive, this ensures that}
\PY{c+c1}{\PYZsh{} the Python interpreter of the Colab VM can load}
\PY{c+c1}{\PYZsh{} python files from within it.}
\PY{k+kn}{import} \PY{n+nn}{sys}
\PY{n}{sys}\PY{o}{.}\PY{n}{path}\PY{o}{.}\PY{n}{append}\PY{p}{(}\PY{l+s+s1}{\PYZsq{}}\PY{l+s+s1}{/content/drive/My Drive/}\PY{l+s+si}{\PYZob{}\PYZcb{}}\PY{l+s+s1}{\PYZsq{}}\PY{o}{.}\PY{n}{format}\PY{p}{(}\PY{n}{FOLDERNAME}\PY{p}{)}\PY{p}{)}

\PY{c+c1}{\PYZsh{} This downloads the CIFAR\PYZhy{}10 dataset to your Drive}
\PY{c+c1}{\PYZsh{} if it doesn\PYZsq{}t already exist.}
\PY{o}{\PYZpc{}}\PY{n}{cd} \PY{o}{/}\PY{n}{content}\PY{o}{/}\PY{n}{drive}\PY{o}{/}\PY{n}{My}\PYZbs{} \PY{n}{Drive}\PY{o}{/}\PY{err}{\PYZdl{}}\PY{n}{FOLDERNAME}\PY{o}{/}\PY{n}{cs231n}\PY{o}{/}\PY{n}{datasets}\PY{o}{/}
\PY{err}{!}\PY{n}{bash} \PY{n}{get\PYZus{}datasets}\PY{o}{.}\PY{n}{sh}
\PY{o}{\PYZpc{}}\PY{n}{cd} \PY{o}{/}\PY{n}{content}\PY{o}{/}\PY{n}{drive}\PY{o}{/}\PY{n}{My}\PYZbs{} \PY{n}{Drive}\PY{o}{/}\PY{err}{\PYZdl{}}\PY{n}{FOLDERNAME}
\end{Verbatim}
\end{tcolorbox}

    \begin{Verbatim}[commandchars=\\\{\}]
Drive already mounted at /content/drive; to attempt to forcibly remount, call
drive.mount("/content/drive", force\_remount=True).
/content/drive/My Drive/cs231n/assignments/assignment1/cs231n/datasets
/content/drive/My Drive/cs231n/assignments/assignment1
    \end{Verbatim}

    \hypertarget{k-nearest-neighbor-knn-exercise}{%
\section{k-Nearest Neighbor (kNN)
exercise}\label{k-nearest-neighbor-knn-exercise}}

\emph{Complete and hand in this completed worksheet (including its
outputs and any supporting code outside of the worksheet) with your
assignment submission. For more details see the
\href{http://vision.stanford.edu/teaching/cs231n/assignments.html}{assignments
page} on the course website.}

The kNN classifier consists of two stages:

\begin{itemize}
\tightlist
\item
  During training, the classifier takes the training data and simply
  remembers it
\item
  During testing, kNN classifies every test image by comparing to all
  training images and transfering the labels of the k most similar
  training examples
\item
  The value of k is cross-validated
\end{itemize}

In this exercise you will implement these steps and understand the basic
Image Classification pipeline, cross-validation, and gain proficiency in
writing efficient, vectorized code.

    \begin{tcolorbox}[breakable, size=fbox, boxrule=1pt, pad at break*=1mm,colback=cellbackground, colframe=cellborder]
\prompt{In}{incolor}{27}{\boxspacing}
\begin{Verbatim}[commandchars=\\\{\}]
\PY{c+c1}{\PYZsh{} Run some setup code for this notebook.}

\PY{k+kn}{import} \PY{n+nn}{random}
\PY{k+kn}{import} \PY{n+nn}{numpy} \PY{k}{as} \PY{n+nn}{np}
\PY{k+kn}{from} \PY{n+nn}{cs231n}\PY{n+nn}{.}\PY{n+nn}{data\PYZus{}utils} \PY{k+kn}{import} \PY{n}{load\PYZus{}CIFAR10}
\PY{k+kn}{import} \PY{n+nn}{matplotlib}\PY{n+nn}{.}\PY{n+nn}{pyplot} \PY{k}{as} \PY{n+nn}{plt}

\PY{c+c1}{\PYZsh{} This is a bit of magic to make matplotlib figures appear inline in the notebook}
\PY{c+c1}{\PYZsh{} rather than in a new window.}
\PY{o}{\PYZpc{}}\PY{n}{matplotlib} \PY{n}{inline}
\PY{n}{plt}\PY{o}{.}\PY{n}{rcParams}\PY{p}{[}\PY{l+s+s1}{\PYZsq{}}\PY{l+s+s1}{figure.figsize}\PY{l+s+s1}{\PYZsq{}}\PY{p}{]} \PY{o}{=} \PY{p}{(}\PY{l+m+mf}{10.0}\PY{p}{,} \PY{l+m+mf}{8.0}\PY{p}{)} \PY{c+c1}{\PYZsh{} set default size of plots}
\PY{n}{plt}\PY{o}{.}\PY{n}{rcParams}\PY{p}{[}\PY{l+s+s1}{\PYZsq{}}\PY{l+s+s1}{image.interpolation}\PY{l+s+s1}{\PYZsq{}}\PY{p}{]} \PY{o}{=} \PY{l+s+s1}{\PYZsq{}}\PY{l+s+s1}{nearest}\PY{l+s+s1}{\PYZsq{}}
\PY{n}{plt}\PY{o}{.}\PY{n}{rcParams}\PY{p}{[}\PY{l+s+s1}{\PYZsq{}}\PY{l+s+s1}{image.cmap}\PY{l+s+s1}{\PYZsq{}}\PY{p}{]} \PY{o}{=} \PY{l+s+s1}{\PYZsq{}}\PY{l+s+s1}{gray}\PY{l+s+s1}{\PYZsq{}}

\PY{c+c1}{\PYZsh{} Some more magic so that the notebook will reload external python modules;}
\PY{c+c1}{\PYZsh{} see http://stackoverflow.com/questions/1907993/autoreload\PYZhy{}of\PYZhy{}modules\PYZhy{}in\PYZhy{}ipython}
\PY{o}{\PYZpc{}}\PY{n}{load\PYZus{}ext} \PY{n}{autoreload}
\PY{o}{\PYZpc{}}\PY{n}{autoreload} \PY{l+m+mi}{2}
\end{Verbatim}
\end{tcolorbox}

    \begin{Verbatim}[commandchars=\\\{\}]
The autoreload extension is already loaded. To reload it, use:
  \%reload\_ext autoreload
    \end{Verbatim}

    \begin{tcolorbox}[breakable, size=fbox, boxrule=1pt, pad at break*=1mm,colback=cellbackground, colframe=cellborder]
\prompt{In}{incolor}{28}{\boxspacing}
\begin{Verbatim}[commandchars=\\\{\}]
\PY{c+c1}{\PYZsh{} Load the raw CIFAR\PYZhy{}10 data.}
\PY{n}{cifar10\PYZus{}dir} \PY{o}{=} \PY{l+s+s1}{\PYZsq{}}\PY{l+s+s1}{cs231n/datasets/cifar\PYZhy{}10\PYZhy{}batches\PYZhy{}py}\PY{l+s+s1}{\PYZsq{}}

\PY{c+c1}{\PYZsh{} Cleaning up variables to prevent loading data multiple times (which may cause memory issue)}
\PY{k}{try}\PY{p}{:}
   \PY{k}{del} \PY{n}{X\PYZus{}train}\PY{p}{,} \PY{n}{y\PYZus{}train}
   \PY{k}{del} \PY{n}{X\PYZus{}test}\PY{p}{,} \PY{n}{y\PYZus{}test}
   \PY{n+nb}{print}\PY{p}{(}\PY{l+s+s1}{\PYZsq{}}\PY{l+s+s1}{Clear previously loaded data.}\PY{l+s+s1}{\PYZsq{}}\PY{p}{)}
\PY{k}{except}\PY{p}{:}
   \PY{k}{pass}

\PY{n}{X\PYZus{}train}\PY{p}{,} \PY{n}{y\PYZus{}train}\PY{p}{,} \PY{n}{X\PYZus{}test}\PY{p}{,} \PY{n}{y\PYZus{}test} \PY{o}{=} \PY{n}{load\PYZus{}CIFAR10}\PY{p}{(}\PY{n}{cifar10\PYZus{}dir}\PY{p}{)}

\PY{c+c1}{\PYZsh{} As a sanity check, we print out the size of the training and test data.}
\PY{n+nb}{print}\PY{p}{(}\PY{l+s+s1}{\PYZsq{}}\PY{l+s+s1}{Training data shape: }\PY{l+s+s1}{\PYZsq{}}\PY{p}{,} \PY{n}{X\PYZus{}train}\PY{o}{.}\PY{n}{shape}\PY{p}{)}
\PY{n+nb}{print}\PY{p}{(}\PY{l+s+s1}{\PYZsq{}}\PY{l+s+s1}{Training labels shape: }\PY{l+s+s1}{\PYZsq{}}\PY{p}{,} \PY{n}{y\PYZus{}train}\PY{o}{.}\PY{n}{shape}\PY{p}{)}
\PY{n+nb}{print}\PY{p}{(}\PY{l+s+s1}{\PYZsq{}}\PY{l+s+s1}{Test data shape: }\PY{l+s+s1}{\PYZsq{}}\PY{p}{,} \PY{n}{X\PYZus{}test}\PY{o}{.}\PY{n}{shape}\PY{p}{)}
\PY{n+nb}{print}\PY{p}{(}\PY{l+s+s1}{\PYZsq{}}\PY{l+s+s1}{Test labels shape: }\PY{l+s+s1}{\PYZsq{}}\PY{p}{,} \PY{n}{y\PYZus{}test}\PY{o}{.}\PY{n}{shape}\PY{p}{)}
\end{Verbatim}
\end{tcolorbox}

    \begin{Verbatim}[commandchars=\\\{\}]
Clear previously loaded data.
Training data shape:  (50000, 32, 32, 3)
Training labels shape:  (50000,)
Test data shape:  (10000, 32, 32, 3)
Test labels shape:  (10000,)
    \end{Verbatim}

    \begin{tcolorbox}[breakable, size=fbox, boxrule=1pt, pad at break*=1mm,colback=cellbackground, colframe=cellborder]
\prompt{In}{incolor}{29}{\boxspacing}
\begin{Verbatim}[commandchars=\\\{\}]
\PY{c+c1}{\PYZsh{} Visualize some examples from the dataset.}
\PY{c+c1}{\PYZsh{} We show a few examples of training images from each class.}
\PY{n}{classes} \PY{o}{=} \PY{p}{[}\PY{l+s+s1}{\PYZsq{}}\PY{l+s+s1}{plane}\PY{l+s+s1}{\PYZsq{}}\PY{p}{,} \PY{l+s+s1}{\PYZsq{}}\PY{l+s+s1}{car}\PY{l+s+s1}{\PYZsq{}}\PY{p}{,} \PY{l+s+s1}{\PYZsq{}}\PY{l+s+s1}{bird}\PY{l+s+s1}{\PYZsq{}}\PY{p}{,} \PY{l+s+s1}{\PYZsq{}}\PY{l+s+s1}{cat}\PY{l+s+s1}{\PYZsq{}}\PY{p}{,} \PY{l+s+s1}{\PYZsq{}}\PY{l+s+s1}{deer}\PY{l+s+s1}{\PYZsq{}}\PY{p}{,} \PY{l+s+s1}{\PYZsq{}}\PY{l+s+s1}{dog}\PY{l+s+s1}{\PYZsq{}}\PY{p}{,} \PY{l+s+s1}{\PYZsq{}}\PY{l+s+s1}{frog}\PY{l+s+s1}{\PYZsq{}}\PY{p}{,} \PY{l+s+s1}{\PYZsq{}}\PY{l+s+s1}{horse}\PY{l+s+s1}{\PYZsq{}}\PY{p}{,} \PY{l+s+s1}{\PYZsq{}}\PY{l+s+s1}{ship}\PY{l+s+s1}{\PYZsq{}}\PY{p}{,} \PY{l+s+s1}{\PYZsq{}}\PY{l+s+s1}{truck}\PY{l+s+s1}{\PYZsq{}}\PY{p}{]}
\PY{n}{num\PYZus{}classes} \PY{o}{=} \PY{n+nb}{len}\PY{p}{(}\PY{n}{classes}\PY{p}{)}
\PY{n}{samples\PYZus{}per\PYZus{}class} \PY{o}{=} \PY{l+m+mi}{7}
\PY{k}{for} \PY{n}{y}\PY{p}{,} \PY{n+nb+bp}{cls} \PY{o+ow}{in} \PY{n+nb}{enumerate}\PY{p}{(}\PY{n}{classes}\PY{p}{)}\PY{p}{:}
    \PY{n}{idxs} \PY{o}{=} \PY{n}{np}\PY{o}{.}\PY{n}{flatnonzero}\PY{p}{(}\PY{n}{y\PYZus{}train} \PY{o}{==} \PY{n}{y}\PY{p}{)}
    \PY{n}{idxs} \PY{o}{=} \PY{n}{np}\PY{o}{.}\PY{n}{random}\PY{o}{.}\PY{n}{choice}\PY{p}{(}\PY{n}{idxs}\PY{p}{,} \PY{n}{samples\PYZus{}per\PYZus{}class}\PY{p}{,} \PY{n}{replace}\PY{o}{=}\PY{k+kc}{False}\PY{p}{)}
    \PY{k}{for} \PY{n}{i}\PY{p}{,} \PY{n}{idx} \PY{o+ow}{in} \PY{n+nb}{enumerate}\PY{p}{(}\PY{n}{idxs}\PY{p}{)}\PY{p}{:}
        \PY{n}{plt\PYZus{}idx} \PY{o}{=} \PY{n}{i} \PY{o}{*} \PY{n}{num\PYZus{}classes} \PY{o}{+} \PY{n}{y} \PY{o}{+} \PY{l+m+mi}{1}
        \PY{n}{plt}\PY{o}{.}\PY{n}{subplot}\PY{p}{(}\PY{n}{samples\PYZus{}per\PYZus{}class}\PY{p}{,} \PY{n}{num\PYZus{}classes}\PY{p}{,} \PY{n}{plt\PYZus{}idx}\PY{p}{)}
        \PY{n}{plt}\PY{o}{.}\PY{n}{imshow}\PY{p}{(}\PY{n}{X\PYZus{}train}\PY{p}{[}\PY{n}{idx}\PY{p}{]}\PY{o}{.}\PY{n}{astype}\PY{p}{(}\PY{l+s+s1}{\PYZsq{}}\PY{l+s+s1}{uint8}\PY{l+s+s1}{\PYZsq{}}\PY{p}{)}\PY{p}{)}
        \PY{n}{plt}\PY{o}{.}\PY{n}{axis}\PY{p}{(}\PY{l+s+s1}{\PYZsq{}}\PY{l+s+s1}{off}\PY{l+s+s1}{\PYZsq{}}\PY{p}{)}
        \PY{k}{if} \PY{n}{i} \PY{o}{==} \PY{l+m+mi}{0}\PY{p}{:}
            \PY{n}{plt}\PY{o}{.}\PY{n}{title}\PY{p}{(}\PY{n+nb+bp}{cls}\PY{p}{)}
\PY{n}{plt}\PY{o}{.}\PY{n}{show}\PY{p}{(}\PY{p}{)}
\end{Verbatim}
\end{tcolorbox}

    \begin{center}
    \adjustimage{max size={0.9\linewidth}{0.9\paperheight}}{knn_files/knn_4_0.png}
    \end{center}
    { \hspace*{\fill} \\}
    
    \begin{tcolorbox}[breakable, size=fbox, boxrule=1pt, pad at break*=1mm,colback=cellbackground, colframe=cellborder]
\prompt{In}{incolor}{30}{\boxspacing}
\begin{Verbatim}[commandchars=\\\{\}]
\PY{c+c1}{\PYZsh{} Subsample the data for more efficient code execution in this exercise}
\PY{n}{num\PYZus{}training} \PY{o}{=} \PY{l+m+mi}{5000}
\PY{n}{mask} \PY{o}{=} \PY{n+nb}{list}\PY{p}{(}\PY{n+nb}{range}\PY{p}{(}\PY{n}{num\PYZus{}training}\PY{p}{)}\PY{p}{)}
\PY{n}{X\PYZus{}train} \PY{o}{=} \PY{n}{X\PYZus{}train}\PY{p}{[}\PY{n}{mask}\PY{p}{]}
\PY{n}{y\PYZus{}train} \PY{o}{=} \PY{n}{y\PYZus{}train}\PY{p}{[}\PY{n}{mask}\PY{p}{]}

\PY{n}{num\PYZus{}test} \PY{o}{=} \PY{l+m+mi}{500}
\PY{n}{mask} \PY{o}{=} \PY{n+nb}{list}\PY{p}{(}\PY{n+nb}{range}\PY{p}{(}\PY{n}{num\PYZus{}test}\PY{p}{)}\PY{p}{)}
\PY{n}{X\PYZus{}test} \PY{o}{=} \PY{n}{X\PYZus{}test}\PY{p}{[}\PY{n}{mask}\PY{p}{]}
\PY{n}{y\PYZus{}test} \PY{o}{=} \PY{n}{y\PYZus{}test}\PY{p}{[}\PY{n}{mask}\PY{p}{]}

\PY{n+nb}{print}\PY{p}{(}\PY{n}{X\PYZus{}train}\PY{o}{.}\PY{n}{shape}\PY{p}{,} \PY{n}{X\PYZus{}test}\PY{o}{.}\PY{n}{shape}\PY{p}{)}

\PY{c+c1}{\PYZsh{} Reshape the image data into rows, reshape前后没区别}
\PY{n}{X\PYZus{}train} \PY{o}{=} \PY{n}{np}\PY{o}{.}\PY{n}{reshape}\PY{p}{(}\PY{n}{X\PYZus{}train}\PY{p}{,} \PY{p}{(}\PY{n}{X\PYZus{}train}\PY{o}{.}\PY{n}{shape}\PY{p}{[}\PY{l+m+mi}{0}\PY{p}{]}\PY{p}{,} \PY{o}{\PYZhy{}}\PY{l+m+mi}{1}\PY{p}{)}\PY{p}{)}
\PY{n}{X\PYZus{}test} \PY{o}{=} \PY{n}{np}\PY{o}{.}\PY{n}{reshape}\PY{p}{(}\PY{n}{X\PYZus{}test}\PY{p}{,} \PY{p}{(}\PY{n}{X\PYZus{}test}\PY{o}{.}\PY{n}{shape}\PY{p}{[}\PY{l+m+mi}{0}\PY{p}{]}\PY{p}{,} \PY{o}{\PYZhy{}}\PY{l+m+mi}{1}\PY{p}{)}\PY{p}{)}
\PY{n+nb}{print}\PY{p}{(}\PY{n}{X\PYZus{}train}\PY{o}{.}\PY{n}{shape}\PY{p}{,} \PY{n}{X\PYZus{}test}\PY{o}{.}\PY{n}{shape}\PY{p}{)}
\end{Verbatim}
\end{tcolorbox}

    \begin{Verbatim}[commandchars=\\\{\}]
(5000, 32, 32, 3) (500, 32, 32, 3)
(5000, 3072) (500, 3072)
    \end{Verbatim}

    \begin{tcolorbox}[breakable, size=fbox, boxrule=1pt, pad at break*=1mm,colback=cellbackground, colframe=cellborder]
\prompt{In}{incolor}{31}{\boxspacing}
\begin{Verbatim}[commandchars=\\\{\}]
\PY{k+kn}{from} \PY{n+nn}{cs231n}\PY{n+nn}{.}\PY{n+nn}{classifiers} \PY{k+kn}{import} \PY{n}{KNearestNeighbor}

\PY{c+c1}{\PYZsh{} Create a kNN classifier instance. }
\PY{c+c1}{\PYZsh{} Remember that training a kNN classifier is a noop: }
\PY{c+c1}{\PYZsh{} the Classifier simply remembers the data and does no further processing }
\PY{n}{classifier} \PY{o}{=} \PY{n}{KNearestNeighbor}\PY{p}{(}\PY{p}{)}
\PY{n}{classifier}\PY{o}{.}\PY{n}{train}\PY{p}{(}\PY{n}{X\PYZus{}train}\PY{p}{,} \PY{n}{y\PYZus{}train}\PY{p}{)} \PY{c+c1}{\PYZsh{} lazy learning, 仅识别的时候进行分析,训练没有作用}
\end{Verbatim}
\end{tcolorbox}

    We would now like to classify the test data with the kNN classifier.
Recall that we can break down this process into two steps:

\begin{enumerate}
\def\labelenumi{\arabic{enumi}.}
\tightlist
\item
  First we must compute the distances between all test examples and all
  train examples.
\item
  Given these distances, for each test example we find the k nearest
  examples and have them vote for the label
\end{enumerate}

Lets begin with computing the distance matrix between all training and
test examples. For example, if there are \textbf{Ntr} training examples
and \textbf{Nte} test examples, this stage should result in a
\textbf{Nte x Ntr} matrix where each element (i,j) is the distance
between the i-th test and j-th train example.

\textbf{Note: For the three distance computations that we require you to
implement in this notebook, you may not use the np.linalg.norm()
function that numpy provides.}

First, open \texttt{cs231n/classifiers/k\_nearest\_neighbor.py} and
implement the function \texttt{compute\_distances\_two\_loops} that uses
a (very inefficient) double loop over all pairs of (test, train)
examples and computes the distance matrix one element at a time.

    \begin{tcolorbox}[breakable, size=fbox, boxrule=1pt, pad at break*=1mm,colback=cellbackground, colframe=cellborder]
\prompt{In}{incolor}{32}{\boxspacing}
\begin{Verbatim}[commandchars=\\\{\}]
\PY{c+c1}{\PYZsh{} Open cs231n/classifiers/k\PYZus{}nearest\PYZus{}neighbor.py and implement}
\PY{c+c1}{\PYZsh{} compute\PYZus{}distances\PYZus{}two\PYZus{}loops.}

\PY{c+c1}{\PYZsh{} Test your implementation:}
\PY{n}{dists} \PY{o}{=} \PY{n}{classifier}\PY{o}{.}\PY{n}{compute\PYZus{}distances\PYZus{}two\PYZus{}loops}\PY{p}{(}\PY{n}{X\PYZus{}test}\PY{p}{)}
\PY{n+nb}{print}\PY{p}{(}\PY{n}{dists}\PY{o}{.}\PY{n}{shape}\PY{p}{)}
\end{Verbatim}
\end{tcolorbox}

    \begin{Verbatim}[commandchars=\\\{\}]
(500, 5000)
    \end{Verbatim}

    \begin{tcolorbox}[breakable, size=fbox, boxrule=1pt, pad at break*=1mm,colback=cellbackground, colframe=cellborder]
\prompt{In}{incolor}{33}{\boxspacing}
\begin{Verbatim}[commandchars=\\\{\}]
\PY{c+c1}{\PYZsh{} We can visualize the distance matrix: each row is a single test example and}
\PY{c+c1}{\PYZsh{} its distances to training examples}
\PY{n}{plt}\PY{o}{.}\PY{n}{imshow}\PY{p}{(}\PY{n}{dists}\PY{p}{,} \PY{n}{interpolation}\PY{o}{=}\PY{l+s+s1}{\PYZsq{}}\PY{l+s+s1}{none}\PY{l+s+s1}{\PYZsq{}}\PY{p}{)}
\PY{n}{plt}\PY{o}{.}\PY{n}{show}\PY{p}{(}\PY{p}{)}
\end{Verbatim}
\end{tcolorbox}

    \begin{center}
    \adjustimage{max size={0.9\linewidth}{0.9\paperheight}}{knn_files/knn_9_0.png}
    \end{center}
    { \hspace*{\fill} \\}
    
    \textbf{Inline Question 1}

Notice the structured patterns in the distance matrix, where some rows
or columns are visibly brighter. (Note that with the default color
scheme black indicates low distances while white indicates high
distances.)

\begin{itemize}
\tightlist
\item
  What in the data is the cause behind the distinctly bright rows?
\item
  What causes the columns?
\end{itemize}

\(\color{blue}{\textit Your Answer:}\)

\begin{itemize}
\tightlist
\item
  这张图片与训练集中大部分图片的欧几里得距离较远,说明类似这张图片的图片在训练集
  中很少出现
\item
  这张图片与测试集中大部分图片的欧几里得距离较远,说明类似这张图片的图片在测试集
  中很少出现
\end{itemize}

    \begin{tcolorbox}[breakable, size=fbox, boxrule=1pt, pad at break*=1mm,colback=cellbackground, colframe=cellborder]
\prompt{In}{incolor}{34}{\boxspacing}
\begin{Verbatim}[commandchars=\\\{\}]
\PY{c+c1}{\PYZsh{} Now implement the function predict\PYZus{}labels and run the code below:}
\PY{c+c1}{\PYZsh{} We use k = 1 (which is Nearest Neighbor).}
\PY{n}{y\PYZus{}test\PYZus{}pred} \PY{o}{=} \PY{n}{classifier}\PY{o}{.}\PY{n}{predict\PYZus{}labels}\PY{p}{(}\PY{n}{dists}\PY{p}{,} \PY{n}{k}\PY{o}{=}\PY{l+m+mi}{1}\PY{p}{)}

\PY{c+c1}{\PYZsh{} Compute and print the fraction of correctly predicted examples}
\PY{n}{num\PYZus{}correct} \PY{o}{=} \PY{n}{np}\PY{o}{.}\PY{n}{sum}\PY{p}{(}\PY{n}{y\PYZus{}test\PYZus{}pred} \PY{o}{==} \PY{n}{y\PYZus{}test}\PY{p}{)}
\PY{n}{accuracy} \PY{o}{=} \PY{n+nb}{float}\PY{p}{(}\PY{n}{num\PYZus{}correct}\PY{p}{)} \PY{o}{/} \PY{n}{num\PYZus{}test}
\PY{n+nb}{print}\PY{p}{(}\PY{l+s+s1}{\PYZsq{}}\PY{l+s+s1}{Got }\PY{l+s+si}{\PYZpc{}d}\PY{l+s+s1}{ / }\PY{l+s+si}{\PYZpc{}d}\PY{l+s+s1}{ correct =\PYZgt{} accuracy: }\PY{l+s+si}{\PYZpc{}f}\PY{l+s+s1}{\PYZsq{}} \PY{o}{\PYZpc{}} \PY{p}{(}\PY{n}{num\PYZus{}correct}\PY{p}{,} \PY{n}{num\PYZus{}test}\PY{p}{,} \PY{n}{accuracy}\PY{p}{)}\PY{p}{)}
\end{Verbatim}
\end{tcolorbox}

    \begin{Verbatim}[commandchars=\\\{\}]
Got 137 / 500 correct => accuracy: 0.274000
    \end{Verbatim}

    You should expect to see approximately \texttt{27\%} accuracy. Now lets
try out a larger \texttt{k}, say \texttt{k\ =\ 5}:

    \begin{tcolorbox}[breakable, size=fbox, boxrule=1pt, pad at break*=1mm,colback=cellbackground, colframe=cellborder]
\prompt{In}{incolor}{24}{\boxspacing}
\begin{Verbatim}[commandchars=\\\{\}]
\PY{n}{y\PYZus{}test\PYZus{}pred} \PY{o}{=} \PY{n}{classifier}\PY{o}{.}\PY{n}{predict\PYZus{}labels}\PY{p}{(}\PY{n}{dists}\PY{p}{,} \PY{n}{k}\PY{o}{=}\PY{l+m+mi}{5}\PY{p}{)}
\PY{n}{num\PYZus{}correct} \PY{o}{=} \PY{n}{np}\PY{o}{.}\PY{n}{sum}\PY{p}{(}\PY{n}{y\PYZus{}test\PYZus{}pred} \PY{o}{==} \PY{n}{y\PYZus{}test}\PY{p}{)}
\PY{n}{accuracy} \PY{o}{=} \PY{n+nb}{float}\PY{p}{(}\PY{n}{num\PYZus{}correct}\PY{p}{)} \PY{o}{/} \PY{n}{num\PYZus{}test}
\PY{n+nb}{print}\PY{p}{(}\PY{l+s+s1}{\PYZsq{}}\PY{l+s+s1}{Got }\PY{l+s+si}{\PYZpc{}d}\PY{l+s+s1}{ / }\PY{l+s+si}{\PYZpc{}d}\PY{l+s+s1}{ correct =\PYZgt{} accuracy: }\PY{l+s+si}{\PYZpc{}f}\PY{l+s+s1}{\PYZsq{}} \PY{o}{\PYZpc{}} \PY{p}{(}\PY{n}{num\PYZus{}correct}\PY{p}{,} \PY{n}{num\PYZus{}test}\PY{p}{,} \PY{n}{accuracy}\PY{p}{)}\PY{p}{)}
\end{Verbatim}
\end{tcolorbox}

    \begin{Verbatim}[commandchars=\\\{\}]
Got 139 / 500 correct => accuracy: 0.278000
    \end{Verbatim}

    You should expect to see a slightly better performance than with
\texttt{k\ =\ 1}.

    \textbf{Inline Question 2}

We can also use other distance metrics such as L1 distance. For pixel
values \(p_{ij}^{(k)}\) at location \((i,j)\) of some image \(I_k\),

the mean \(\mu\) across all pixels over all images is

测试集中所有图片所有像素的平均值
\[\mu=\frac{1}{nhw}\sum_{k=1}^n\sum_{i=1}^{h}\sum_{j=1}^{w}p_{ij}^{(k)}\]

And the pixel-wise mean \(\mu_{ij}\) across all images is

所有照片{[}i,j{]}处像素的平均值
\[\mu_{ij}=\frac{1}{n}\sum_{k=1}^np_{ij}^{(k)}.\]

The general standard deviation \(\sigma\) and pixel-wise standard
deviation \(\sigma_{ij}\) is defined similarly.

Which of the following preprocessing steps will not change the
performance of a Nearest Neighbor classifier that uses L1 distance?
Select all that apply. 1. Subtracting the mean \(\mu\)
(\(\tilde{p}_{ij}^{(k)}=p_{ij}^{(k)}-\mu\).) 2. Subtracting the per
pixel mean \(\mu_{ij}\)
(\(\tilde{p}_{ij}^{(k)}=p_{ij}^{(k)}-\mu_{ij}\).) 3. Subtracting the
mean \(\mu\) and dividing by the standard deviation \(\sigma\). 4.
Subtracting the pixel-wise mean \(\mu_{ij}\) and dividing by the
pixel-wise standard deviation \(\sigma_{ij}\). 5. Rotating the
coordinate axes of the data.

\(\color{blue}{\textit Your Answer:}\)

没有变化的:1,2,3,5

\(\color{blue}{\textit Your Explanation:}\)

\begin{enumerate}
\def\labelenumi{\arabic{enumi}.}
\tightlist
\item
  所有照片的所有像素减去一个相同的值
\item
  所有照片的同一位置像素减去一个相同值
\item
  (\(\tilde{p}_{ij}^{(k)}=\frac{p_{ij}^{(k)}-\mu}{\sigma}\))
\end{enumerate}

所有照片的所有位置像素减去相同值后缩放相同值 4.
(\(\tilde{p}_{ij}^{(k)}=\frac{p_{ij}^{(k)}-\mu_{ij}}{\sigma_{ij}}\))

所有照片的同一位置像素减去相同值后缩放相同值

5.将图片旋转

knn使用两个图片对应像素值的差值作为分类的依据,1和2同一位置减去的同一个值可以抵
消,不会影响结果;

3在1和2的基础上对所有像素等倍缩放,是矩阵的线性变换,对所有图片和像素的影响程度
相同,不会影响结果。

5对图片旋转,相当于对矩阵进行转置,不会影响两张图片像素的对应关系,也不会影响
结果。

    \begin{tcolorbox}[breakable, size=fbox, boxrule=1pt, pad at break*=1mm,colback=cellbackground, colframe=cellborder]
\prompt{In}{incolor}{40}{\boxspacing}
\begin{Verbatim}[commandchars=\\\{\}]
\PY{c+c1}{\PYZsh{} Now lets speed up distance matrix computation by using partial vectorization}
\PY{c+c1}{\PYZsh{} with one loop. Implement the function compute\PYZus{}distances\PYZus{}one\PYZus{}loop and run the}
\PY{c+c1}{\PYZsh{} code below:}
\PY{n}{dists\PYZus{}one} \PY{o}{=} \PY{n}{classifier}\PY{o}{.}\PY{n}{compute\PYZus{}distances\PYZus{}one\PYZus{}loop}\PY{p}{(}\PY{n}{X\PYZus{}test}\PY{p}{)}

\PY{c+c1}{\PYZsh{} To ensure that our vectorized implementation is correct, we make sure that it}
\PY{c+c1}{\PYZsh{} agrees with the naive implementation. There are many ways to decide whether}
\PY{c+c1}{\PYZsh{} two matrices are similar; one of the simplest is the Frobenius norm. In case}
\PY{c+c1}{\PYZsh{} you haven\PYZsq{}t seen it before, the Frobenius norm of two matrices is the square}
\PY{c+c1}{\PYZsh{} root of the squared sum of differences of all elements; in other words, reshape}
\PY{c+c1}{\PYZsh{} the matrices into vectors and compute the Euclidean distance between them.}
\PY{n}{difference} \PY{o}{=} \PY{n}{np}\PY{o}{.}\PY{n}{linalg}\PY{o}{.}\PY{n}{norm}\PY{p}{(}\PY{n}{dists} \PY{o}{\PYZhy{}} \PY{n}{dists\PYZus{}one}\PY{p}{,} \PY{n+nb}{ord}\PY{o}{=}\PY{l+s+s1}{\PYZsq{}}\PY{l+s+s1}{fro}\PY{l+s+s1}{\PYZsq{}}\PY{p}{)}
\PY{n+nb}{print}\PY{p}{(}\PY{l+s+s1}{\PYZsq{}}\PY{l+s+s1}{One loop difference was: }\PY{l+s+si}{\PYZpc{}f}\PY{l+s+s1}{\PYZsq{}} \PY{o}{\PYZpc{}} \PY{p}{(}\PY{n}{difference}\PY{p}{,} \PY{p}{)}\PY{p}{)}
\PY{k}{if} \PY{n}{difference} \PY{o}{\PYZlt{}} \PY{l+m+mf}{0.001}\PY{p}{:}
    \PY{n+nb}{print}\PY{p}{(}\PY{l+s+s1}{\PYZsq{}}\PY{l+s+s1}{Good! The distance matrices are the same}\PY{l+s+s1}{\PYZsq{}}\PY{p}{)}
\PY{k}{else}\PY{p}{:}
    \PY{n+nb}{print}\PY{p}{(}\PY{l+s+s1}{\PYZsq{}}\PY{l+s+s1}{Uh\PYZhy{}oh! The distance matrices are different}\PY{l+s+s1}{\PYZsq{}}\PY{p}{)}
\end{Verbatim}
\end{tcolorbox}

    \begin{Verbatim}[commandchars=\\\{\}]
One loop difference was: 0.000000
Good! The distance matrices are the same
    \end{Verbatim}

    \begin{tcolorbox}[breakable, size=fbox, boxrule=1pt, pad at break*=1mm,colback=cellbackground, colframe=cellborder]
\prompt{In}{incolor}{103}{\boxspacing}
\begin{Verbatim}[commandchars=\\\{\}]
\PY{c+c1}{\PYZsh{} Now implement the fully vectorized version inside compute\PYZus{}distances\PYZus{}no\PYZus{}loops}
\PY{c+c1}{\PYZsh{} and run the code}
\PY{n}{dists\PYZus{}two} \PY{o}{=} \PY{n}{classifier}\PY{o}{.}\PY{n}{compute\PYZus{}distances\PYZus{}no\PYZus{}loops}\PY{p}{(}\PY{n}{X\PYZus{}test}\PY{p}{)}

\PY{c+c1}{\PYZsh{} check that the distance matrix agrees with the one we computed before:}
\PY{n}{difference} \PY{o}{=} \PY{n}{np}\PY{o}{.}\PY{n}{linalg}\PY{o}{.}\PY{n}{norm}\PY{p}{(}\PY{n}{dists} \PY{o}{\PYZhy{}} \PY{n}{dists\PYZus{}two}\PY{p}{,} \PY{n+nb}{ord}\PY{o}{=}\PY{l+s+s1}{\PYZsq{}}\PY{l+s+s1}{fro}\PY{l+s+s1}{\PYZsq{}}\PY{p}{)}
\PY{n+nb}{print}\PY{p}{(}\PY{l+s+s1}{\PYZsq{}}\PY{l+s+s1}{No loop difference was: }\PY{l+s+si}{\PYZpc{}f}\PY{l+s+s1}{\PYZsq{}} \PY{o}{\PYZpc{}} \PY{p}{(}\PY{n}{difference}\PY{p}{,} \PY{p}{)}\PY{p}{)}
\PY{k}{if} \PY{n}{difference} \PY{o}{\PYZlt{}} \PY{l+m+mf}{0.001}\PY{p}{:}
    \PY{n+nb}{print}\PY{p}{(}\PY{l+s+s1}{\PYZsq{}}\PY{l+s+s1}{Good! The distance matrices are the same}\PY{l+s+s1}{\PYZsq{}}\PY{p}{)}
\PY{k}{else}\PY{p}{:}
    \PY{n+nb}{print}\PY{p}{(}\PY{l+s+s1}{\PYZsq{}}\PY{l+s+s1}{Uh\PYZhy{}oh! The distance matrices are different}\PY{l+s+s1}{\PYZsq{}}\PY{p}{)}
\end{Verbatim}
\end{tcolorbox}

    \begin{Verbatim}[commandchars=\\\{\}]
No loop difference was: 0.000000
Good! The distance matrices are the same
    \end{Verbatim}

    \begin{tcolorbox}[breakable, size=fbox, boxrule=1pt, pad at break*=1mm,colback=cellbackground, colframe=cellborder]
\prompt{In}{incolor}{104}{\boxspacing}
\begin{Verbatim}[commandchars=\\\{\}]
\PY{c+c1}{\PYZsh{} Let\PYZsq{}s compare how fast the implementations are}
\PY{k}{def} \PY{n+nf}{time\PYZus{}function}\PY{p}{(}\PY{n}{f}\PY{p}{,} \PY{o}{*}\PY{n}{args}\PY{p}{)}\PY{p}{:}
    \PY{l+s+sd}{\PYZdq{}\PYZdq{}\PYZdq{}}
\PY{l+s+sd}{    Call a function f with args and return the time (in seconds) that it took to execute.}
\PY{l+s+sd}{    \PYZdq{}\PYZdq{}\PYZdq{}}
    \PY{k+kn}{import} \PY{n+nn}{time}
    \PY{n}{tic} \PY{o}{=} \PY{n}{time}\PY{o}{.}\PY{n}{time}\PY{p}{(}\PY{p}{)}
    \PY{n}{f}\PY{p}{(}\PY{o}{*}\PY{n}{args}\PY{p}{)}
    \PY{n}{toc} \PY{o}{=} \PY{n}{time}\PY{o}{.}\PY{n}{time}\PY{p}{(}\PY{p}{)}
    \PY{k}{return} \PY{n}{toc} \PY{o}{\PYZhy{}} \PY{n}{tic}

\PY{n}{two\PYZus{}loop\PYZus{}time} \PY{o}{=} \PY{n}{time\PYZus{}function}\PY{p}{(}\PY{n}{classifier}\PY{o}{.}\PY{n}{compute\PYZus{}distances\PYZus{}two\PYZus{}loops}\PY{p}{,} \PY{n}{X\PYZus{}test}\PY{p}{)}
\PY{n+nb}{print}\PY{p}{(}\PY{l+s+s1}{\PYZsq{}}\PY{l+s+s1}{Two loop version took }\PY{l+s+si}{\PYZpc{}f}\PY{l+s+s1}{ seconds}\PY{l+s+s1}{\PYZsq{}} \PY{o}{\PYZpc{}} \PY{n}{two\PYZus{}loop\PYZus{}time}\PY{p}{)}

\PY{n}{one\PYZus{}loop\PYZus{}time} \PY{o}{=} \PY{n}{time\PYZus{}function}\PY{p}{(}\PY{n}{classifier}\PY{o}{.}\PY{n}{compute\PYZus{}distances\PYZus{}one\PYZus{}loop}\PY{p}{,} \PY{n}{X\PYZus{}test}\PY{p}{)}
\PY{n+nb}{print}\PY{p}{(}\PY{l+s+s1}{\PYZsq{}}\PY{l+s+s1}{One loop version took }\PY{l+s+si}{\PYZpc{}f}\PY{l+s+s1}{ seconds}\PY{l+s+s1}{\PYZsq{}} \PY{o}{\PYZpc{}} \PY{n}{one\PYZus{}loop\PYZus{}time}\PY{p}{)}

\PY{n}{no\PYZus{}loop\PYZus{}time} \PY{o}{=} \PY{n}{time\PYZus{}function}\PY{p}{(}\PY{n}{classifier}\PY{o}{.}\PY{n}{compute\PYZus{}distances\PYZus{}no\PYZus{}loops}\PY{p}{,} \PY{n}{X\PYZus{}test}\PY{p}{)}
\PY{n+nb}{print}\PY{p}{(}\PY{l+s+s1}{\PYZsq{}}\PY{l+s+s1}{No loop version took }\PY{l+s+si}{\PYZpc{}f}\PY{l+s+s1}{ seconds}\PY{l+s+s1}{\PYZsq{}} \PY{o}{\PYZpc{}} \PY{n}{no\PYZus{}loop\PYZus{}time}\PY{p}{)}

\PY{c+c1}{\PYZsh{} You should see significantly faster performance with the fully vectorized implementation!}

\PY{c+c1}{\PYZsh{} NOTE: depending on what machine you\PYZsq{}re using, }
\PY{c+c1}{\PYZsh{} you might not see a speedup when you go from two loops to one loop, }
\PY{c+c1}{\PYZsh{} and might even see a slow\PYZhy{}down.}
\end{Verbatim}
\end{tcolorbox}

    \begin{Verbatim}[commandchars=\\\{\}]
Two loop version took 33.958122 seconds
One loop version took 40.322520 seconds
No loop version took 0.578593 seconds
    \end{Verbatim}

    \hypertarget{cross-validation}{%
\subsubsection{Cross-validation}\label{cross-validation}}

We have implemented the k-Nearest Neighbor classifier but we set the
value k = 5 arbitrarily. We will now determine the best value of this
hyperparameter with cross-validation.

    \begin{tcolorbox}[breakable, size=fbox, boxrule=1pt, pad at break*=1mm,colback=cellbackground, colframe=cellborder]
\prompt{In}{incolor}{152}{\boxspacing}
\begin{Verbatim}[commandchars=\\\{\}]
\PY{n}{num\PYZus{}folds} \PY{o}{=} \PY{l+m+mi}{5}
\PY{n}{k\PYZus{}choices} \PY{o}{=} \PY{p}{[}\PY{l+m+mi}{1}\PY{p}{,} \PY{l+m+mi}{3}\PY{p}{,} \PY{l+m+mi}{5}\PY{p}{,} \PY{l+m+mi}{8}\PY{p}{,} \PY{l+m+mi}{10}\PY{p}{,} \PY{l+m+mi}{12}\PY{p}{,} \PY{l+m+mi}{15}\PY{p}{,} \PY{l+m+mi}{20}\PY{p}{,} \PY{l+m+mi}{50}\PY{p}{,} \PY{l+m+mi}{100}\PY{p}{]}

\PY{n}{X\PYZus{}train\PYZus{}folds} \PY{o}{=} \PY{p}{[}\PY{p}{]}
\PY{n}{y\PYZus{}train\PYZus{}folds} \PY{o}{=} \PY{p}{[}\PY{p}{]}
\PY{c+c1}{\PYZsh{}\PYZsh{}\PYZsh{}\PYZsh{}\PYZsh{}\PYZsh{}\PYZsh{}\PYZsh{}\PYZsh{}\PYZsh{}\PYZsh{}\PYZsh{}\PYZsh{}\PYZsh{}\PYZsh{}\PYZsh{}\PYZsh{}\PYZsh{}\PYZsh{}\PYZsh{}\PYZsh{}\PYZsh{}\PYZsh{}\PYZsh{}\PYZsh{}\PYZsh{}\PYZsh{}\PYZsh{}\PYZsh{}\PYZsh{}\PYZsh{}\PYZsh{}\PYZsh{}\PYZsh{}\PYZsh{}\PYZsh{}\PYZsh{}\PYZsh{}\PYZsh{}\PYZsh{}\PYZsh{}\PYZsh{}\PYZsh{}\PYZsh{}\PYZsh{}\PYZsh{}\PYZsh{}\PYZsh{}\PYZsh{}\PYZsh{}\PYZsh{}\PYZsh{}\PYZsh{}\PYZsh{}\PYZsh{}\PYZsh{}\PYZsh{}\PYZsh{}\PYZsh{}\PYZsh{}\PYZsh{}\PYZsh{}\PYZsh{}\PYZsh{}\PYZsh{}\PYZsh{}\PYZsh{}\PYZsh{}\PYZsh{}\PYZsh{}\PYZsh{}\PYZsh{}\PYZsh{}\PYZsh{}\PYZsh{}\PYZsh{}\PYZsh{}\PYZsh{}\PYZsh{}\PYZsh{}}
\PY{c+c1}{\PYZsh{} TODO:                                                                        \PYZsh{}}
\PY{c+c1}{\PYZsh{} Split up the training data into folds. After splitting, X\PYZus{}train\PYZus{}folds and    \PYZsh{}}
\PY{c+c1}{\PYZsh{} y\PYZus{}train\PYZus{}folds should each be lists of length num\PYZus{}folds, where                \PYZsh{}}
\PY{c+c1}{\PYZsh{} y\PYZus{}train\PYZus{}folds[i] is the label vector for the points in X\PYZus{}train\PYZus{}folds[i].     \PYZsh{}}
\PY{c+c1}{\PYZsh{} Hint: Look up the numpy array\PYZus{}split function.                                \PYZsh{}}
\PY{c+c1}{\PYZsh{}\PYZsh{}\PYZsh{}\PYZsh{}\PYZsh{}\PYZsh{}\PYZsh{}\PYZsh{}\PYZsh{}\PYZsh{}\PYZsh{}\PYZsh{}\PYZsh{}\PYZsh{}\PYZsh{}\PYZsh{}\PYZsh{}\PYZsh{}\PYZsh{}\PYZsh{}\PYZsh{}\PYZsh{}\PYZsh{}\PYZsh{}\PYZsh{}\PYZsh{}\PYZsh{}\PYZsh{}\PYZsh{}\PYZsh{}\PYZsh{}\PYZsh{}\PYZsh{}\PYZsh{}\PYZsh{}\PYZsh{}\PYZsh{}\PYZsh{}\PYZsh{}\PYZsh{}\PYZsh{}\PYZsh{}\PYZsh{}\PYZsh{}\PYZsh{}\PYZsh{}\PYZsh{}\PYZsh{}\PYZsh{}\PYZsh{}\PYZsh{}\PYZsh{}\PYZsh{}\PYZsh{}\PYZsh{}\PYZsh{}\PYZsh{}\PYZsh{}\PYZsh{}\PYZsh{}\PYZsh{}\PYZsh{}\PYZsh{}\PYZsh{}\PYZsh{}\PYZsh{}\PYZsh{}\PYZsh{}\PYZsh{}\PYZsh{}\PYZsh{}\PYZsh{}\PYZsh{}\PYZsh{}\PYZsh{}\PYZsh{}\PYZsh{}\PYZsh{}\PYZsh{}\PYZsh{}}
\PY{c+c1}{\PYZsh{} *****START OF YOUR CODE (DO NOT DELETE/MODIFY THIS LINE)*****}

\PY{n}{X\PYZus{}train\PYZus{}folds} \PY{o}{=} \PY{n}{np}\PY{o}{.}\PY{n}{array\PYZus{}split}\PY{p}{(}\PY{n}{X\PYZus{}train}\PY{p}{,} \PY{n}{num\PYZus{}folds}\PY{p}{)}
\PY{c+c1}{\PYZsh{} print(X\PYZus{}train\PYZus{}folds)}
\PY{n}{y\PYZus{}train\PYZus{}folds} \PY{o}{=} \PY{n}{np}\PY{o}{.}\PY{n}{array\PYZus{}split}\PY{p}{(}\PY{n}{y\PYZus{}train}\PY{p}{,} \PY{n}{num\PYZus{}folds}\PY{p}{)}
\PY{c+c1}{\PYZsh{} print(y\PYZus{}train\PYZus{}folds)}

\PY{c+c1}{\PYZsh{} *****END OF YOUR CODE (DO NOT DELETE/MODIFY THIS LINE)*****}

\PY{c+c1}{\PYZsh{} A dictionary holding the accuracies for different values of k that we find}
\PY{c+c1}{\PYZsh{} when running cross\PYZhy{}validation. After running cross\PYZhy{}validation,}
\PY{c+c1}{\PYZsh{} k\PYZus{}to\PYZus{}accuracies[k] should be a list of length num\PYZus{}folds giving the different}
\PY{c+c1}{\PYZsh{} accuracy values that we found when using that value of k.}
\PY{n}{k\PYZus{}to\PYZus{}accuracies} \PY{o}{=} \PY{p}{\PYZob{}}\PY{p}{\PYZcb{}}


\PY{c+c1}{\PYZsh{}\PYZsh{}\PYZsh{}\PYZsh{}\PYZsh{}\PYZsh{}\PYZsh{}\PYZsh{}\PYZsh{}\PYZsh{}\PYZsh{}\PYZsh{}\PYZsh{}\PYZsh{}\PYZsh{}\PYZsh{}\PYZsh{}\PYZsh{}\PYZsh{}\PYZsh{}\PYZsh{}\PYZsh{}\PYZsh{}\PYZsh{}\PYZsh{}\PYZsh{}\PYZsh{}\PYZsh{}\PYZsh{}\PYZsh{}\PYZsh{}\PYZsh{}\PYZsh{}\PYZsh{}\PYZsh{}\PYZsh{}\PYZsh{}\PYZsh{}\PYZsh{}\PYZsh{}\PYZsh{}\PYZsh{}\PYZsh{}\PYZsh{}\PYZsh{}\PYZsh{}\PYZsh{}\PYZsh{}\PYZsh{}\PYZsh{}\PYZsh{}\PYZsh{}\PYZsh{}\PYZsh{}\PYZsh{}\PYZsh{}\PYZsh{}\PYZsh{}\PYZsh{}\PYZsh{}\PYZsh{}\PYZsh{}\PYZsh{}\PYZsh{}\PYZsh{}\PYZsh{}\PYZsh{}\PYZsh{}\PYZsh{}\PYZsh{}\PYZsh{}\PYZsh{}\PYZsh{}\PYZsh{}\PYZsh{}\PYZsh{}\PYZsh{}\PYZsh{}\PYZsh{}\PYZsh{}}
\PY{c+c1}{\PYZsh{} TODO:                                                                        \PYZsh{}}
\PY{c+c1}{\PYZsh{} Perform k\PYZhy{}fold cross validation to find the best value of k. For each        \PYZsh{}}
\PY{c+c1}{\PYZsh{} possible value of k, run the k\PYZhy{}nearest\PYZhy{}neighbor algorithm num\PYZus{}folds times,   \PYZsh{}}
\PY{c+c1}{\PYZsh{} where in each case you use all but one of the folds as training data and the \PYZsh{}}
\PY{c+c1}{\PYZsh{} last fold as a validation set. Store the accuracies for all fold and all     \PYZsh{}}
\PY{c+c1}{\PYZsh{} values of k in the k\PYZus{}to\PYZus{}accuracies dictionary.                               \PYZsh{}}
\PY{c+c1}{\PYZsh{}\PYZsh{}\PYZsh{}\PYZsh{}\PYZsh{}\PYZsh{}\PYZsh{}\PYZsh{}\PYZsh{}\PYZsh{}\PYZsh{}\PYZsh{}\PYZsh{}\PYZsh{}\PYZsh{}\PYZsh{}\PYZsh{}\PYZsh{}\PYZsh{}\PYZsh{}\PYZsh{}\PYZsh{}\PYZsh{}\PYZsh{}\PYZsh{}\PYZsh{}\PYZsh{}\PYZsh{}\PYZsh{}\PYZsh{}\PYZsh{}\PYZsh{}\PYZsh{}\PYZsh{}\PYZsh{}\PYZsh{}\PYZsh{}\PYZsh{}\PYZsh{}\PYZsh{}\PYZsh{}\PYZsh{}\PYZsh{}\PYZsh{}\PYZsh{}\PYZsh{}\PYZsh{}\PYZsh{}\PYZsh{}\PYZsh{}\PYZsh{}\PYZsh{}\PYZsh{}\PYZsh{}\PYZsh{}\PYZsh{}\PYZsh{}\PYZsh{}\PYZsh{}\PYZsh{}\PYZsh{}\PYZsh{}\PYZsh{}\PYZsh{}\PYZsh{}\PYZsh{}\PYZsh{}\PYZsh{}\PYZsh{}\PYZsh{}\PYZsh{}\PYZsh{}\PYZsh{}\PYZsh{}\PYZsh{}\PYZsh{}\PYZsh{}\PYZsh{}\PYZsh{}\PYZsh{}}
\PY{c+c1}{\PYZsh{} *****START OF YOUR CODE (DO NOT DELETE/MODIFY THIS LINE)*****}

\PY{k}{for} \PY{n}{k} \PY{o+ow}{in} \PY{n}{k\PYZus{}choices}\PY{p}{:}
  \PY{n}{k\PYZus{}to\PYZus{}accuracies}\PY{p}{[}\PY{n}{k}\PY{p}{]}\PY{o}{=}\PY{p}{[}\PY{p}{]}
  \PY{k}{for} \PY{n}{i} \PY{o+ow}{in} \PY{n+nb}{range}\PY{p}{(}\PY{n}{num\PYZus{}folds}\PY{p}{)}\PY{p}{:}
    \PY{c+c1}{\PYZsh{} Subsample the data for more efficient code execution in this exercise}
    \PY{c+c1}{\PYZsh{} print(f\PYZsq{}new\PYZus{}X\PYZus{}train[\PYZob{}new\PYZus{}X\PYZus{}train\PYZcb{}],new\PYZus{}y\PYZus{}train[\PYZob{}new\PYZus{}y\PYZus{}train\PYZcb{}]\PYZsq{})}
    \PY{n}{idxs} \PY{o}{=} \PY{p}{[}\PY{n}{j} \PY{k}{for} \PY{n}{j} \PY{o+ow}{in} \PY{n+nb}{range}\PY{p}{(}\PY{n}{num\PYZus{}folds}\PY{p}{)} \PY{k}{if} \PY{n}{j} \PY{o}{!=} \PY{n}{i}\PY{p}{]}
    \PY{n}{new\PYZus{}X\PYZus{}train}\PY{o}{=}\PY{n}{np}\PY{o}{.}\PY{n}{concatenate}\PY{p}{(}\PY{p}{[}\PY{n}{X\PYZus{}train\PYZus{}folds}\PY{p}{[}\PY{n}{j}\PY{p}{]} \PY{k}{for} \PY{n}{j} \PY{o+ow}{in} \PY{n}{idxs}\PY{p}{]}\PY{p}{,}\PY{n}{axis}\PY{o}{=}\PY{l+m+mi}{0}\PY{p}{)}
    \PY{n}{new\PYZus{}y\PYZus{}train}\PY{o}{=}\PY{n}{np}\PY{o}{.}\PY{n}{concatenate}\PY{p}{(}\PY{p}{[}\PY{n}{y\PYZus{}train\PYZus{}folds}\PY{p}{[}\PY{n}{j}\PY{p}{]} \PY{k}{for} \PY{n}{j} \PY{o+ow}{in} \PY{n}{idxs}\PY{p}{]}\PY{p}{,}\PY{n}{axis}\PY{o}{=}\PY{l+m+mi}{0}\PY{p}{)}
    \PY{c+c1}{\PYZsh{} print(new\PYZus{}X\PYZus{}train)}

    \PY{n}{new\PYZus{}X\PYZus{}test} \PY{o}{=} \PY{n}{X\PYZus{}train\PYZus{}folds}\PY{p}{[}\PY{n}{i}\PY{p}{]}
    \PY{n}{new\PYZus{}y\PYZus{}test} \PY{o}{=} \PY{n}{y\PYZus{}train\PYZus{}folds}\PY{p}{[}\PY{n}{i}\PY{p}{]}

    \PY{n}{new\PYZus{}num\PYZus{}test} \PY{o}{=} \PY{n+nb}{len}\PY{p}{(}\PY{n}{new\PYZus{}y\PYZus{}test}\PY{p}{)}

    \PY{c+c1}{\PYZsh{} Reshape the image data into rows}
    \PY{n}{new\PYZus{}X\PYZus{}train} \PY{o}{=} \PY{n}{np}\PY{o}{.}\PY{n}{reshape}\PY{p}{(}\PY{n}{new\PYZus{}X\PYZus{}train}\PY{p}{,} \PY{p}{(}\PY{n}{new\PYZus{}X\PYZus{}train}\PY{o}{.}\PY{n}{shape}\PY{p}{[}\PY{l+m+mi}{0}\PY{p}{]}\PY{p}{,} \PY{o}{\PYZhy{}}\PY{l+m+mi}{1}\PY{p}{)}\PY{p}{)}
    \PY{n}{new\PYZus{}X\PYZus{}test} \PY{o}{=} \PY{n}{np}\PY{o}{.}\PY{n}{reshape}\PY{p}{(}\PY{n}{new\PYZus{}X\PYZus{}test}\PY{p}{,} \PY{p}{(}\PY{n}{new\PYZus{}X\PYZus{}test}\PY{o}{.}\PY{n}{shape}\PY{p}{[}\PY{l+m+mi}{0}\PY{p}{]}\PY{p}{,} \PY{o}{\PYZhy{}}\PY{l+m+mi}{1}\PY{p}{)}\PY{p}{)}
    \PY{c+c1}{\PYZsh{} print(new\PYZus{}X\PYZus{}train.shape, new\PYZus{}X\PYZus{}test.shape)}

    \PY{n}{classifier}\PY{o}{.}\PY{n}{train}\PY{p}{(}\PY{n}{new\PYZus{}X\PYZus{}train}\PY{p}{,} \PY{n}{new\PYZus{}y\PYZus{}train}\PY{p}{)} \PY{c+c1}{\PYZsh{} lazy learning, 仅识别的时候进行分析,训练没有作用}

    \PY{n}{new\PYZus{}dists} \PY{o}{=} \PY{n}{classifier}\PY{o}{.}\PY{n}{compute\PYZus{}distances\PYZus{}no\PYZus{}loops}\PY{p}{(}\PY{n}{new\PYZus{}X\PYZus{}test}\PY{p}{)}
    \PY{c+c1}{\PYZsh{} print(new\PYZus{}dists.shape)}

    \PY{n}{y\PYZus{}test\PYZus{}pred} \PY{o}{=} \PY{n}{classifier}\PY{o}{.}\PY{n}{predict\PYZus{}labels}\PY{p}{(}\PY{n}{new\PYZus{}dists}\PY{p}{,} \PY{n}{k}\PY{p}{)}

    \PY{n}{num\PYZus{}correct} \PY{o}{=} \PY{n}{np}\PY{o}{.}\PY{n}{sum}\PY{p}{(}\PY{n}{y\PYZus{}test\PYZus{}pred} \PY{o}{==} \PY{n}{new\PYZus{}y\PYZus{}test}\PY{p}{)}
    \PY{n}{accuracy} \PY{o}{=} \PY{n+nb}{float}\PY{p}{(}\PY{n}{num\PYZus{}correct}\PY{p}{)} \PY{o}{/} \PY{n}{new\PYZus{}num\PYZus{}test}
    \PY{c+c1}{\PYZsh{} print(\PYZsq{}Got \PYZpc{}d / \PYZpc{}d correct =\PYZgt{} accuracy: \PYZpc{}f\PYZsq{} \PYZpc{} (num\PYZus{}correct, num\PYZus{}test, accuracy))}
    \PY{n}{k\PYZus{}to\PYZus{}accuracies}\PY{p}{[}\PY{n}{k}\PY{p}{]}\PY{o}{.}\PY{n}{append}\PY{p}{(}\PY{n}{accuracy}\PY{p}{)}

\PY{c+c1}{\PYZsh{} print(k\PYZus{}to\PYZus{}accuracies)}
\PY{c+c1}{\PYZsh{} *****END OF YOUR CODE (DO NOT DELETE/MODIFY THIS LINE)*****}

\PY{c+c1}{\PYZsh{} Print out the computed accuracies}
\PY{k}{for} \PY{n}{k} \PY{o+ow}{in} \PY{n+nb}{sorted}\PY{p}{(}\PY{n}{k\PYZus{}to\PYZus{}accuracies}\PY{p}{)}\PY{p}{:}
    \PY{k}{for} \PY{n}{accuracy} \PY{o+ow}{in} \PY{n}{k\PYZus{}to\PYZus{}accuracies}\PY{p}{[}\PY{n}{k}\PY{p}{]}\PY{p}{:}
        \PY{n+nb}{print}\PY{p}{(}\PY{l+s+s1}{\PYZsq{}}\PY{l+s+s1}{k = }\PY{l+s+si}{\PYZpc{}d}\PY{l+s+s1}{, accuracy = }\PY{l+s+si}{\PYZpc{}f}\PY{l+s+s1}{\PYZsq{}} \PY{o}{\PYZpc{}} \PY{p}{(}\PY{n}{k}\PY{p}{,} \PY{n}{accuracy}\PY{p}{)}\PY{p}{)}
\end{Verbatim}
\end{tcolorbox}

    \begin{Verbatim}[commandchars=\\\{\}]
k = 1, accuracy = 0.263000
k = 1, accuracy = 0.257000
k = 1, accuracy = 0.264000
k = 1, accuracy = 0.278000
k = 1, accuracy = 0.266000
k = 3, accuracy = 0.239000
k = 3, accuracy = 0.249000
k = 3, accuracy = 0.240000
k = 3, accuracy = 0.266000
k = 3, accuracy = 0.254000
k = 5, accuracy = 0.248000
k = 5, accuracy = 0.266000
k = 5, accuracy = 0.280000
k = 5, accuracy = 0.292000
k = 5, accuracy = 0.280000
k = 8, accuracy = 0.262000
k = 8, accuracy = 0.282000
k = 8, accuracy = 0.273000
k = 8, accuracy = 0.290000
k = 8, accuracy = 0.273000
k = 10, accuracy = 0.265000
k = 10, accuracy = 0.296000
k = 10, accuracy = 0.276000
k = 10, accuracy = 0.284000
k = 10, accuracy = 0.280000
k = 12, accuracy = 0.260000
k = 12, accuracy = 0.295000
k = 12, accuracy = 0.279000
k = 12, accuracy = 0.283000
k = 12, accuracy = 0.280000
k = 15, accuracy = 0.252000
k = 15, accuracy = 0.289000
k = 15, accuracy = 0.278000
k = 15, accuracy = 0.282000
k = 15, accuracy = 0.274000
k = 20, accuracy = 0.270000
k = 20, accuracy = 0.279000
k = 20, accuracy = 0.279000
k = 20, accuracy = 0.282000
k = 20, accuracy = 0.285000
k = 50, accuracy = 0.271000
k = 50, accuracy = 0.288000
k = 50, accuracy = 0.278000
k = 50, accuracy = 0.269000
k = 50, accuracy = 0.266000
k = 100, accuracy = 0.256000
k = 100, accuracy = 0.270000
k = 100, accuracy = 0.263000
k = 100, accuracy = 0.256000
k = 100, accuracy = 0.263000
    \end{Verbatim}

    \begin{tcolorbox}[breakable, size=fbox, boxrule=1pt, pad at break*=1mm,colback=cellbackground, colframe=cellborder]
\prompt{In}{incolor}{153}{\boxspacing}
\begin{Verbatim}[commandchars=\\\{\}]
\PY{c+c1}{\PYZsh{} plot the raw observations}
\PY{k}{for} \PY{n}{k} \PY{o+ow}{in} \PY{n}{k\PYZus{}choices}\PY{p}{:}
    \PY{n}{accuracies} \PY{o}{=} \PY{n}{k\PYZus{}to\PYZus{}accuracies}\PY{p}{[}\PY{n}{k}\PY{p}{]}
    \PY{n}{plt}\PY{o}{.}\PY{n}{scatter}\PY{p}{(}\PY{p}{[}\PY{n}{k}\PY{p}{]} \PY{o}{*} \PY{n+nb}{len}\PY{p}{(}\PY{n}{accuracies}\PY{p}{)}\PY{p}{,} \PY{n}{accuracies}\PY{p}{)}

\PY{c+c1}{\PYZsh{} plot the trend line with error bars that correspond to standard deviation}
\PY{n}{accuracies\PYZus{}mean} \PY{o}{=} \PY{n}{np}\PY{o}{.}\PY{n}{array}\PY{p}{(}\PY{p}{[}\PY{n}{np}\PY{o}{.}\PY{n}{mean}\PY{p}{(}\PY{n}{v}\PY{p}{)} \PY{k}{for} \PY{n}{k}\PY{p}{,}\PY{n}{v} \PY{o+ow}{in} \PY{n+nb}{sorted}\PY{p}{(}\PY{n}{k\PYZus{}to\PYZus{}accuracies}\PY{o}{.}\PY{n}{items}\PY{p}{(}\PY{p}{)}\PY{p}{)}\PY{p}{]}\PY{p}{)}
\PY{n}{accuracies\PYZus{}std} \PY{o}{=} \PY{n}{np}\PY{o}{.}\PY{n}{array}\PY{p}{(}\PY{p}{[}\PY{n}{np}\PY{o}{.}\PY{n}{std}\PY{p}{(}\PY{n}{v}\PY{p}{)} \PY{k}{for} \PY{n}{k}\PY{p}{,}\PY{n}{v} \PY{o+ow}{in} \PY{n+nb}{sorted}\PY{p}{(}\PY{n}{k\PYZus{}to\PYZus{}accuracies}\PY{o}{.}\PY{n}{items}\PY{p}{(}\PY{p}{)}\PY{p}{)}\PY{p}{]}\PY{p}{)}
\PY{n}{plt}\PY{o}{.}\PY{n}{errorbar}\PY{p}{(}\PY{n}{k\PYZus{}choices}\PY{p}{,} \PY{n}{accuracies\PYZus{}mean}\PY{p}{,} \PY{n}{yerr}\PY{o}{=}\PY{n}{accuracies\PYZus{}std}\PY{p}{)}
\PY{n}{plt}\PY{o}{.}\PY{n}{title}\PY{p}{(}\PY{l+s+s1}{\PYZsq{}}\PY{l+s+s1}{Cross\PYZhy{}validation on k}\PY{l+s+s1}{\PYZsq{}}\PY{p}{)}
\PY{n}{plt}\PY{o}{.}\PY{n}{xlabel}\PY{p}{(}\PY{l+s+s1}{\PYZsq{}}\PY{l+s+s1}{k}\PY{l+s+s1}{\PYZsq{}}\PY{p}{)}
\PY{n}{plt}\PY{o}{.}\PY{n}{ylabel}\PY{p}{(}\PY{l+s+s1}{\PYZsq{}}\PY{l+s+s1}{Cross\PYZhy{}validation accuracy}\PY{l+s+s1}{\PYZsq{}}\PY{p}{)}
\PY{n}{plt}\PY{o}{.}\PY{n}{show}\PY{p}{(}\PY{p}{)}
\end{Verbatim}
\end{tcolorbox}

    \begin{center}
    \adjustimage{max size={0.9\linewidth}{0.9\paperheight}}{knn_files/knn_21_0.png}
    \end{center}
    { \hspace*{\fill} \\}
    
    \begin{tcolorbox}[breakable, size=fbox, boxrule=1pt, pad at break*=1mm,colback=cellbackground, colframe=cellborder]
\prompt{In}{incolor}{156}{\boxspacing}
\begin{Verbatim}[commandchars=\\\{\}]
\PY{c+c1}{\PYZsh{} Based on the cross\PYZhy{}validation results above, choose the best value for k,   }
\PY{c+c1}{\PYZsh{} retrain the classifier using all the training data, and test it on the test}
\PY{c+c1}{\PYZsh{} data. You should be able to get above 28\PYZpc{} accuracy on the test data.}
\PY{n}{best\PYZus{}k} \PY{o}{=} \PY{l+m+mi}{10}

\PY{n}{classifier} \PY{o}{=} \PY{n}{KNearestNeighbor}\PY{p}{(}\PY{p}{)}
\PY{n}{classifier}\PY{o}{.}\PY{n}{train}\PY{p}{(}\PY{n}{X\PYZus{}train}\PY{p}{,} \PY{n}{y\PYZus{}train}\PY{p}{)}
\PY{n}{y\PYZus{}test\PYZus{}pred} \PY{o}{=} \PY{n}{classifier}\PY{o}{.}\PY{n}{predict}\PY{p}{(}\PY{n}{X\PYZus{}test}\PY{p}{,} \PY{n}{k}\PY{o}{=}\PY{n}{best\PYZus{}k}\PY{p}{)}

\PY{c+c1}{\PYZsh{} Compute and display the accuracy}
\PY{n}{num\PYZus{}correct} \PY{o}{=} \PY{n}{np}\PY{o}{.}\PY{n}{sum}\PY{p}{(}\PY{n}{y\PYZus{}test\PYZus{}pred} \PY{o}{==} \PY{n}{y\PYZus{}test}\PY{p}{)}
\PY{n}{accuracy} \PY{o}{=} \PY{n+nb}{float}\PY{p}{(}\PY{n}{num\PYZus{}correct}\PY{p}{)} \PY{o}{/} \PY{n}{num\PYZus{}test}
\PY{n+nb}{print}\PY{p}{(}\PY{l+s+s1}{\PYZsq{}}\PY{l+s+s1}{Got }\PY{l+s+si}{\PYZpc{}d}\PY{l+s+s1}{ / }\PY{l+s+si}{\PYZpc{}d}\PY{l+s+s1}{ correct =\PYZgt{} accuracy: }\PY{l+s+si}{\PYZpc{}f}\PY{l+s+s1}{\PYZsq{}} \PY{o}{\PYZpc{}} \PY{p}{(}\PY{n}{num\PYZus{}correct}\PY{p}{,} \PY{n}{num\PYZus{}test}\PY{p}{,} \PY{n}{accuracy}\PY{p}{)}\PY{p}{)}
\end{Verbatim}
\end{tcolorbox}

    \begin{Verbatim}[commandchars=\\\{\}]
Got 141 / 500 correct => accuracy: 0.282000
    \end{Verbatim}

    \textbf{Inline Question 3}

Which of the following statements about \(k\)-Nearest Neighbor
(\(k\)-NN) are true in a classification setting, and for all \(k\)?
Select all that apply. 1. The decision boundary of the k-NN classifier
is linear. 2. The training error of a 1-NN will always be lower than or
equal to that of 5-NN. 3. The test error of a 1-NN will always be lower
than that of a 5-NN. 4. The time needed to classify a test example with
the k-NN classifier grows with the size of the training set. 5. None of
the above.

\(\color{blue}{\textit Your Answer:}\)

1.False 2.True 3.False 4.True

\(\color{blue}{\textit Your Explanation:}\)

\begin{enumerate}
\def\labelenumi{\arabic{enumi}.}
\tightlist
\item
  knn没有类似svm的超平面,对test
  point进行分类的时候,用于训练的点的边界通常是 曲线,即knn
  classifier是非线性的。
\item
  training
  error指使用训练集作为数据集进行预测时,同一个点可能出现的差错。由于k=1时仅使用最近的一个点进行预测,而对测试集为训练集的情况来说,离一个点自己最近的就是自己,故1-nn的traing
  error永远不大于5-nn。
\item
  test
  error指使用训练集之外的数据作为数据集进行预测时出现的差错。假设有一个一维情况,一共有2类,训练集为\(X_{train}=\{1,1,1,2,2\}\),\(y_{train}=\{0,0,0,1,1\}\),对于数据\(x=3, y=0\),1nn认为他属于类1,而5nn认为他属于类0。
\item
  knn每次预测的时候都需要计算所给数据与训练集数据的所有距离,训练集越大,计算距离的计算量就越大,需要时间就越长。
\end{enumerate}


    % Add a bibliography block to the postdoc
    
    
    
\end{document}
