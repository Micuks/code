\documentclass[11pt]{article}

    \usepackage[breakable]{tcolorbox}
    \usepackage{parskip} % Stop auto-indenting (to mimic markdown behaviour)
    \usepackage{xeCJK}
    
    \setCJKmainfont{Source Han Serif SC VF}
    \setCJKsansfont{Source Han Sans SC VF}
    \setCJKmonofont{Source Han Sans SC VF}

    % Basic figure setup, for now with no caption control since it's done
    % automatically by Pandoc (which extracts ![](path) syntax from Markdown).
    \usepackage{graphicx}
    % Maintain compatibility with old templates. Remove in nbconvert 6.0
    \let\Oldincludegraphics\includegraphics
    % Ensure that by default, figures have no caption (until we provide a
    % proper Figure object with a Caption API and a way to capture that
    % in the conversion process - todo).
    \usepackage{caption}
    \DeclareCaptionFormat{nocaption}{}
    \captionsetup{format=nocaption,aboveskip=0pt,belowskip=0pt}

    \usepackage{float}
    \floatplacement{figure}{H} % forces figures to be placed at the correct location
    \usepackage{xcolor} % Allow colors to be defined
    \usepackage{enumerate} % Needed for markdown enumerations to work
    \usepackage{geometry} % Used to adjust the document margins
    \usepackage{amsmath} % Equations
    \usepackage{amssymb} % Equations
    \usepackage{textcomp} % defines textquotesingle
    % Hack from http://tex.stackexchange.com/a/47451/13684:
    \AtBeginDocument{%
        \def\PYZsq{\textquotesingle}% Upright quotes in Pygmentized code
    }
    \usepackage{upquote} % Upright quotes for verbatim code
    \usepackage{eurosym} % defines \euro

    \usepackage{iftex}
    \ifPDFTeX
        \usepackage[T1]{fontenc}
        \IfFileExists{alphabeta.sty}{
              \usepackage{alphabeta}
          }{
              \usepackage[mathletters]{ucs}
              \usepackage[utf8x]{inputenc}
          }
    \else
        \usepackage{fontspec}
        \usepackage{unicode-math}
    \fi

    \usepackage{fancyvrb} % verbatim replacement that allows latex
    \usepackage{grffile} % extends the file name processing of package graphics
                         % to support a larger range
    \makeatletter % fix for old versions of grffile with XeLaTeX
    \@ifpackagelater{grffile}{2019/11/01}
    {
      % Do nothing on new versions
    }
    {
      \def\Gread@@xetex#1{%
        \IfFileExists{"\Gin@base".bb}%
        {\Gread@eps{\Gin@base.bb}}%
        {\Gread@@xetex@aux#1}%
      }
    }
    \makeatother
    \usepackage[Export]{adjustbox} % Used to constrain images to a maximum size
    \adjustboxset{max size={0.9\linewidth}{0.9\paperheight}}

    % The hyperref package gives us a pdf with properly built
    % internal navigation ('pdf bookmarks' for the table of contents,
    % internal cross-reference links, web links for URLs, etc.)
    \usepackage{hyperref}
    % The default LaTeX title has an obnoxious amount of whitespace. By default,
    % titling removes some of it. It also provides customization options.
    \usepackage{titling}
    \usepackage{longtable} % longtable support required by pandoc >1.10
    \usepackage{booktabs}  % table support for pandoc > 1.12.2
    \usepackage{array}     % table support for pandoc >= 2.11.3
    \usepackage{calc}      % table minipage width calculation for pandoc >= 2.11.1
    \usepackage[inline]{enumitem} % IRkernel/repr support (it uses the enumerate* environment)
    \usepackage[normalem]{ulem} % ulem is needed to support strikethroughs (\sout)
                                % normalem makes italics be italics, not underlines
    \usepackage{mathrsfs}
    

    
    % Colors for the hyperref package
    \definecolor{urlcolor}{rgb}{0,.145,.698}
    \definecolor{linkcolor}{rgb}{.71,0.21,0.01}
    \definecolor{citecolor}{rgb}{.12,.54,.11}

    % ANSI colors
    \definecolor{ansi-black}{HTML}{3E424D}
    \definecolor{ansi-black-intense}{HTML}{282C36}
    \definecolor{ansi-red}{HTML}{E75C58}
    \definecolor{ansi-red-intense}{HTML}{B22B31}
    \definecolor{ansi-green}{HTML}{00A250}
    \definecolor{ansi-green-intense}{HTML}{007427}
    \definecolor{ansi-yellow}{HTML}{DDB62B}
    \definecolor{ansi-yellow-intense}{HTML}{B27D12}
    \definecolor{ansi-blue}{HTML}{208FFB}
    \definecolor{ansi-blue-intense}{HTML}{0065CA}
    \definecolor{ansi-magenta}{HTML}{D160C4}
    \definecolor{ansi-magenta-intense}{HTML}{A03196}
    \definecolor{ansi-cyan}{HTML}{60C6C8}
    \definecolor{ansi-cyan-intense}{HTML}{258F8F}
    \definecolor{ansi-white}{HTML}{C5C1B4}
    \definecolor{ansi-white-intense}{HTML}{A1A6B2}
    \definecolor{ansi-default-inverse-fg}{HTML}{FFFFFF}
    \definecolor{ansi-default-inverse-bg}{HTML}{000000}

    % common color for the border for error outputs.
    \definecolor{outerrorbackground}{HTML}{FFDFDF}

    % commands and environments needed by pandoc snippets
    % extracted from the output of `pandoc -s`
    \providecommand{\tightlist}{%
      \setlength{\itemsep}{0pt}\setlength{\parskip}{0pt}}
    \DefineVerbatimEnvironment{Highlighting}{Verbatim}{commandchars=\\\{\}}
    % Add ',fontsize=\small' for more characters per line
    \newenvironment{Shaded}{}{}
    \newcommand{\KeywordTok}[1]{\textcolor[rgb]{0.00,0.44,0.13}{\textbf{{#1}}}}
    \newcommand{\DataTypeTok}[1]{\textcolor[rgb]{0.56,0.13,0.00}{{#1}}}
    \newcommand{\DecValTok}[1]{\textcolor[rgb]{0.25,0.63,0.44}{{#1}}}
    \newcommand{\BaseNTok}[1]{\textcolor[rgb]{0.25,0.63,0.44}{{#1}}}
    \newcommand{\FloatTok}[1]{\textcolor[rgb]{0.25,0.63,0.44}{{#1}}}
    \newcommand{\CharTok}[1]{\textcolor[rgb]{0.25,0.44,0.63}{{#1}}}
    \newcommand{\StringTok}[1]{\textcolor[rgb]{0.25,0.44,0.63}{{#1}}}
    \newcommand{\CommentTok}[1]{\textcolor[rgb]{0.38,0.63,0.69}{\textit{{#1}}}}
    \newcommand{\OtherTok}[1]{\textcolor[rgb]{0.00,0.44,0.13}{{#1}}}
    \newcommand{\AlertTok}[1]{\textcolor[rgb]{1.00,0.00,0.00}{\textbf{{#1}}}}
    \newcommand{\FunctionTok}[1]{\textcolor[rgb]{0.02,0.16,0.49}{{#1}}}
    \newcommand{\RegionMarkerTok}[1]{{#1}}
    \newcommand{\ErrorTok}[1]{\textcolor[rgb]{1.00,0.00,0.00}{\textbf{{#1}}}}
    \newcommand{\NormalTok}[1]{{#1}}

    % Additional commands for more recent versions of Pandoc
    \newcommand{\ConstantTok}[1]{\textcolor[rgb]{0.53,0.00,0.00}{{#1}}}
    \newcommand{\SpecialCharTok}[1]{\textcolor[rgb]{0.25,0.44,0.63}{{#1}}}
    \newcommand{\VerbatimStringTok}[1]{\textcolor[rgb]{0.25,0.44,0.63}{{#1}}}
    \newcommand{\SpecialStringTok}[1]{\textcolor[rgb]{0.73,0.40,0.53}{{#1}}}
    \newcommand{\ImportTok}[1]{{#1}}
    \newcommand{\DocumentationTok}[1]{\textcolor[rgb]{0.73,0.13,0.13}{\textit{{#1}}}}
    \newcommand{\AnnotationTok}[1]{\textcolor[rgb]{0.38,0.63,0.69}{\textbf{\textit{{#1}}}}}
    \newcommand{\CommentVarTok}[1]{\textcolor[rgb]{0.38,0.63,0.69}{\textbf{\textit{{#1}}}}}
    \newcommand{\VariableTok}[1]{\textcolor[rgb]{0.10,0.09,0.49}{{#1}}}
    \newcommand{\ControlFlowTok}[1]{\textcolor[rgb]{0.00,0.44,0.13}{\textbf{{#1}}}}
    \newcommand{\OperatorTok}[1]{\textcolor[rgb]{0.40,0.40,0.40}{{#1}}}
    \newcommand{\BuiltInTok}[1]{{#1}}
    \newcommand{\ExtensionTok}[1]{{#1}}
    \newcommand{\PreprocessorTok}[1]{\textcolor[rgb]{0.74,0.48,0.00}{{#1}}}
    \newcommand{\AttributeTok}[1]{\textcolor[rgb]{0.49,0.56,0.16}{{#1}}}
    \newcommand{\InformationTok}[1]{\textcolor[rgb]{0.38,0.63,0.69}{\textbf{\textit{{#1}}}}}
    \newcommand{\WarningTok}[1]{\textcolor[rgb]{0.38,0.63,0.69}{\textbf{\textit{{#1}}}}}


    % Define a nice break command that doesn't care if a line doesn't already
    % exist.
    \def\br{\hspace*{\fill} \\* }
    % Math Jax compatibility definitions
    \def\gt{>}
    \def\lt{<}
    \let\Oldtex\TeX
    \let\Oldlatex\LaTeX
    \renewcommand{\TeX}{\textrm{\Oldtex}}
    \renewcommand{\LaTeX}{\textrm{\Oldlatex}}
    % Document parameters
    % Document title
    \title{two\_layer\_net}
    
    
    
    
    
% Pygments definitions
\makeatletter
\def\PY@reset{\let\PY@it=\relax \let\PY@bf=\relax%
    \let\PY@ul=\relax \let\PY@tc=\relax%
    \let\PY@bc=\relax \let\PY@ff=\relax}
\def\PY@tok#1{\csname PY@tok@#1\endcsname}
\def\PY@toks#1+{\ifx\relax#1\empty\else%
    \PY@tok{#1}\expandafter\PY@toks\fi}
\def\PY@do#1{\PY@bc{\PY@tc{\PY@ul{%
    \PY@it{\PY@bf{\PY@ff{#1}}}}}}}
\def\PY#1#2{\PY@reset\PY@toks#1+\relax+\PY@do{#2}}

\@namedef{PY@tok@w}{\def\PY@tc##1{\textcolor[rgb]{0.73,0.73,0.73}{##1}}}
\@namedef{PY@tok@c}{\let\PY@it=\textit\def\PY@tc##1{\textcolor[rgb]{0.24,0.48,0.48}{##1}}}
\@namedef{PY@tok@cp}{\def\PY@tc##1{\textcolor[rgb]{0.61,0.40,0.00}{##1}}}
\@namedef{PY@tok@k}{\let\PY@bf=\textbf\def\PY@tc##1{\textcolor[rgb]{0.00,0.50,0.00}{##1}}}
\@namedef{PY@tok@kp}{\def\PY@tc##1{\textcolor[rgb]{0.00,0.50,0.00}{##1}}}
\@namedef{PY@tok@kt}{\def\PY@tc##1{\textcolor[rgb]{0.69,0.00,0.25}{##1}}}
\@namedef{PY@tok@o}{\def\PY@tc##1{\textcolor[rgb]{0.40,0.40,0.40}{##1}}}
\@namedef{PY@tok@ow}{\let\PY@bf=\textbf\def\PY@tc##1{\textcolor[rgb]{0.67,0.13,1.00}{##1}}}
\@namedef{PY@tok@nb}{\def\PY@tc##1{\textcolor[rgb]{0.00,0.50,0.00}{##1}}}
\@namedef{PY@tok@nf}{\def\PY@tc##1{\textcolor[rgb]{0.00,0.00,1.00}{##1}}}
\@namedef{PY@tok@nc}{\let\PY@bf=\textbf\def\PY@tc##1{\textcolor[rgb]{0.00,0.00,1.00}{##1}}}
\@namedef{PY@tok@nn}{\let\PY@bf=\textbf\def\PY@tc##1{\textcolor[rgb]{0.00,0.00,1.00}{##1}}}
\@namedef{PY@tok@ne}{\let\PY@bf=\textbf\def\PY@tc##1{\textcolor[rgb]{0.80,0.25,0.22}{##1}}}
\@namedef{PY@tok@nv}{\def\PY@tc##1{\textcolor[rgb]{0.10,0.09,0.49}{##1}}}
\@namedef{PY@tok@no}{\def\PY@tc##1{\textcolor[rgb]{0.53,0.00,0.00}{##1}}}
\@namedef{PY@tok@nl}{\def\PY@tc##1{\textcolor[rgb]{0.46,0.46,0.00}{##1}}}
\@namedef{PY@tok@ni}{\let\PY@bf=\textbf\def\PY@tc##1{\textcolor[rgb]{0.44,0.44,0.44}{##1}}}
\@namedef{PY@tok@na}{\def\PY@tc##1{\textcolor[rgb]{0.41,0.47,0.13}{##1}}}
\@namedef{PY@tok@nt}{\let\PY@bf=\textbf\def\PY@tc##1{\textcolor[rgb]{0.00,0.50,0.00}{##1}}}
\@namedef{PY@tok@nd}{\def\PY@tc##1{\textcolor[rgb]{0.67,0.13,1.00}{##1}}}
\@namedef{PY@tok@s}{\def\PY@tc##1{\textcolor[rgb]{0.73,0.13,0.13}{##1}}}
\@namedef{PY@tok@sd}{\let\PY@it=\textit\def\PY@tc##1{\textcolor[rgb]{0.73,0.13,0.13}{##1}}}
\@namedef{PY@tok@si}{\let\PY@bf=\textbf\def\PY@tc##1{\textcolor[rgb]{0.64,0.35,0.47}{##1}}}
\@namedef{PY@tok@se}{\let\PY@bf=\textbf\def\PY@tc##1{\textcolor[rgb]{0.67,0.36,0.12}{##1}}}
\@namedef{PY@tok@sr}{\def\PY@tc##1{\textcolor[rgb]{0.64,0.35,0.47}{##1}}}
\@namedef{PY@tok@ss}{\def\PY@tc##1{\textcolor[rgb]{0.10,0.09,0.49}{##1}}}
\@namedef{PY@tok@sx}{\def\PY@tc##1{\textcolor[rgb]{0.00,0.50,0.00}{##1}}}
\@namedef{PY@tok@m}{\def\PY@tc##1{\textcolor[rgb]{0.40,0.40,0.40}{##1}}}
\@namedef{PY@tok@gh}{\let\PY@bf=\textbf\def\PY@tc##1{\textcolor[rgb]{0.00,0.00,0.50}{##1}}}
\@namedef{PY@tok@gu}{\let\PY@bf=\textbf\def\PY@tc##1{\textcolor[rgb]{0.50,0.00,0.50}{##1}}}
\@namedef{PY@tok@gd}{\def\PY@tc##1{\textcolor[rgb]{0.63,0.00,0.00}{##1}}}
\@namedef{PY@tok@gi}{\def\PY@tc##1{\textcolor[rgb]{0.00,0.52,0.00}{##1}}}
\@namedef{PY@tok@gr}{\def\PY@tc##1{\textcolor[rgb]{0.89,0.00,0.00}{##1}}}
\@namedef{PY@tok@ge}{\let\PY@it=\textit}
\@namedef{PY@tok@gs}{\let\PY@bf=\textbf}
\@namedef{PY@tok@gp}{\let\PY@bf=\textbf\def\PY@tc##1{\textcolor[rgb]{0.00,0.00,0.50}{##1}}}
\@namedef{PY@tok@go}{\def\PY@tc##1{\textcolor[rgb]{0.44,0.44,0.44}{##1}}}
\@namedef{PY@tok@gt}{\def\PY@tc##1{\textcolor[rgb]{0.00,0.27,0.87}{##1}}}
\@namedef{PY@tok@err}{\def\PY@bc##1{{\setlength{\fboxsep}{\string -\fboxrule}\fcolorbox[rgb]{1.00,0.00,0.00}{1,1,1}{\strut ##1}}}}
\@namedef{PY@tok@kc}{\let\PY@bf=\textbf\def\PY@tc##1{\textcolor[rgb]{0.00,0.50,0.00}{##1}}}
\@namedef{PY@tok@kd}{\let\PY@bf=\textbf\def\PY@tc##1{\textcolor[rgb]{0.00,0.50,0.00}{##1}}}
\@namedef{PY@tok@kn}{\let\PY@bf=\textbf\def\PY@tc##1{\textcolor[rgb]{0.00,0.50,0.00}{##1}}}
\@namedef{PY@tok@kr}{\let\PY@bf=\textbf\def\PY@tc##1{\textcolor[rgb]{0.00,0.50,0.00}{##1}}}
\@namedef{PY@tok@bp}{\def\PY@tc##1{\textcolor[rgb]{0.00,0.50,0.00}{##1}}}
\@namedef{PY@tok@fm}{\def\PY@tc##1{\textcolor[rgb]{0.00,0.00,1.00}{##1}}}
\@namedef{PY@tok@vc}{\def\PY@tc##1{\textcolor[rgb]{0.10,0.09,0.49}{##1}}}
\@namedef{PY@tok@vg}{\def\PY@tc##1{\textcolor[rgb]{0.10,0.09,0.49}{##1}}}
\@namedef{PY@tok@vi}{\def\PY@tc##1{\textcolor[rgb]{0.10,0.09,0.49}{##1}}}
\@namedef{PY@tok@vm}{\def\PY@tc##1{\textcolor[rgb]{0.10,0.09,0.49}{##1}}}
\@namedef{PY@tok@sa}{\def\PY@tc##1{\textcolor[rgb]{0.73,0.13,0.13}{##1}}}
\@namedef{PY@tok@sb}{\def\PY@tc##1{\textcolor[rgb]{0.73,0.13,0.13}{##1}}}
\@namedef{PY@tok@sc}{\def\PY@tc##1{\textcolor[rgb]{0.73,0.13,0.13}{##1}}}
\@namedef{PY@tok@dl}{\def\PY@tc##1{\textcolor[rgb]{0.73,0.13,0.13}{##1}}}
\@namedef{PY@tok@s2}{\def\PY@tc##1{\textcolor[rgb]{0.73,0.13,0.13}{##1}}}
\@namedef{PY@tok@sh}{\def\PY@tc##1{\textcolor[rgb]{0.73,0.13,0.13}{##1}}}
\@namedef{PY@tok@s1}{\def\PY@tc##1{\textcolor[rgb]{0.73,0.13,0.13}{##1}}}
\@namedef{PY@tok@mb}{\def\PY@tc##1{\textcolor[rgb]{0.40,0.40,0.40}{##1}}}
\@namedef{PY@tok@mf}{\def\PY@tc##1{\textcolor[rgb]{0.40,0.40,0.40}{##1}}}
\@namedef{PY@tok@mh}{\def\PY@tc##1{\textcolor[rgb]{0.40,0.40,0.40}{##1}}}
\@namedef{PY@tok@mi}{\def\PY@tc##1{\textcolor[rgb]{0.40,0.40,0.40}{##1}}}
\@namedef{PY@tok@il}{\def\PY@tc##1{\textcolor[rgb]{0.40,0.40,0.40}{##1}}}
\@namedef{PY@tok@mo}{\def\PY@tc##1{\textcolor[rgb]{0.40,0.40,0.40}{##1}}}
\@namedef{PY@tok@ch}{\let\PY@it=\textit\def\PY@tc##1{\textcolor[rgb]{0.24,0.48,0.48}{##1}}}
\@namedef{PY@tok@cm}{\let\PY@it=\textit\def\PY@tc##1{\textcolor[rgb]{0.24,0.48,0.48}{##1}}}
\@namedef{PY@tok@cpf}{\let\PY@it=\textit\def\PY@tc##1{\textcolor[rgb]{0.24,0.48,0.48}{##1}}}
\@namedef{PY@tok@c1}{\let\PY@it=\textit\def\PY@tc##1{\textcolor[rgb]{0.24,0.48,0.48}{##1}}}
\@namedef{PY@tok@cs}{\let\PY@it=\textit\def\PY@tc##1{\textcolor[rgb]{0.24,0.48,0.48}{##1}}}

\def\PYZbs{\char`\\}
\def\PYZus{\char`\_}
\def\PYZob{\char`\{}
\def\PYZcb{\char`\}}
\def\PYZca{\char`\^}
\def\PYZam{\char`\&}
\def\PYZlt{\char`\<}
\def\PYZgt{\char`\>}
\def\PYZsh{\char`\#}
\def\PYZpc{\char`\%}
\def\PYZdl{\char`\$}
\def\PYZhy{\char`\-}
\def\PYZsq{\char`\'}
\def\PYZdq{\char`\"}
\def\PYZti{\char`\~}
% for compatibility with earlier versions
\def\PYZat{@}
\def\PYZlb{[}
\def\PYZrb{]}
\makeatother


    % For linebreaks inside Verbatim environment from package fancyvrb.
    \makeatletter
        \newbox\Wrappedcontinuationbox
        \newbox\Wrappedvisiblespacebox
        \newcommand*\Wrappedvisiblespace {\textcolor{red}{\textvisiblespace}}
        \newcommand*\Wrappedcontinuationsymbol {\textcolor{red}{\llap{\tiny$\m@th\hookrightarrow$}}}
        \newcommand*\Wrappedcontinuationindent {3ex }
        \newcommand*\Wrappedafterbreak {\kern\Wrappedcontinuationindent\copy\Wrappedcontinuationbox}
        % Take advantage of the already applied Pygments mark-up to insert
        % potential linebreaks for TeX processing.
        %        {, <, #, %, $, ' and ": go to next line.
        %        _, }, ^, &, >, - and ~: stay at end of broken line.
        % Use of \textquotesingle for straight quote.
        \newcommand*\Wrappedbreaksatspecials {%
            \def\PYGZus{\discretionary{\char`\_}{\Wrappedafterbreak}{\char`\_}}%
            \def\PYGZob{\discretionary{}{\Wrappedafterbreak\char`\{}{\char`\{}}%
            \def\PYGZcb{\discretionary{\char`\}}{\Wrappedafterbreak}{\char`\}}}%
            \def\PYGZca{\discretionary{\char`\^}{\Wrappedafterbreak}{\char`\^}}%
            \def\PYGZam{\discretionary{\char`\&}{\Wrappedafterbreak}{\char`\&}}%
            \def\PYGZlt{\discretionary{}{\Wrappedafterbreak\char`\<}{\char`\<}}%
            \def\PYGZgt{\discretionary{\char`\>}{\Wrappedafterbreak}{\char`\>}}%
            \def\PYGZsh{\discretionary{}{\Wrappedafterbreak\char`\#}{\char`\#}}%
            \def\PYGZpc{\discretionary{}{\Wrappedafterbreak\char`\%}{\char`\%}}%
            \def\PYGZdl{\discretionary{}{\Wrappedafterbreak\char`\$}{\char`\$}}%
            \def\PYGZhy{\discretionary{\char`\-}{\Wrappedafterbreak}{\char`\-}}%
            \def\PYGZsq{\discretionary{}{\Wrappedafterbreak\textquotesingle}{\textquotesingle}}%
            \def\PYGZdq{\discretionary{}{\Wrappedafterbreak\char`\"}{\char`\"}}%
            \def\PYGZti{\discretionary{\char`\~}{\Wrappedafterbreak}{\char`\~}}%
        }
        % Some characters . , ; ? ! / are not pygmentized.
        % This macro makes them "active" and they will insert potential linebreaks
        \newcommand*\Wrappedbreaksatpunct {%
            \lccode`\~`\.\lowercase{\def~}{\discretionary{\hbox{\char`\.}}{\Wrappedafterbreak}{\hbox{\char`\.}}}%
            \lccode`\~`\,\lowercase{\def~}{\discretionary{\hbox{\char`\,}}{\Wrappedafterbreak}{\hbox{\char`\,}}}%
            \lccode`\~`\;\lowercase{\def~}{\discretionary{\hbox{\char`\;}}{\Wrappedafterbreak}{\hbox{\char`\;}}}%
            \lccode`\~`\:\lowercase{\def~}{\discretionary{\hbox{\char`\:}}{\Wrappedafterbreak}{\hbox{\char`\:}}}%
            \lccode`\~`\?\lowercase{\def~}{\discretionary{\hbox{\char`\?}}{\Wrappedafterbreak}{\hbox{\char`\?}}}%
            \lccode`\~`\!\lowercase{\def~}{\discretionary{\hbox{\char`\!}}{\Wrappedafterbreak}{\hbox{\char`\!}}}%
            \lccode`\~`\/\lowercase{\def~}{\discretionary{\hbox{\char`\/}}{\Wrappedafterbreak}{\hbox{\char`\/}}}%
            \catcode`\.\active
            \catcode`\,\active
            \catcode`\;\active
            \catcode`\:\active
            \catcode`\?\active
            \catcode`\!\active
            \catcode`\/\active
            \lccode`\~`\~
        }
    \makeatother

    \let\OriginalVerbatim=\Verbatim
    \makeatletter
    \renewcommand{\Verbatim}[1][1]{%
        %\parskip\z@skip
        \sbox\Wrappedcontinuationbox {\Wrappedcontinuationsymbol}%
        \sbox\Wrappedvisiblespacebox {\FV@SetupFont\Wrappedvisiblespace}%
        \def\FancyVerbFormatLine ##1{\hsize\linewidth
            \vtop{\raggedright\hyphenpenalty\z@\exhyphenpenalty\z@
                \doublehyphendemerits\z@\finalhyphendemerits\z@
                \strut ##1\strut}%
        }%
        % If the linebreak is at a space, the latter will be displayed as visible
        % space at end of first line, and a continuation symbol starts next line.
        % Stretch/shrink are however usually zero for typewriter font.
        \def\FV@Space {%
            \nobreak\hskip\z@ plus\fontdimen3\font minus\fontdimen4\font
            \discretionary{\copy\Wrappedvisiblespacebox}{\Wrappedafterbreak}
            {\kern\fontdimen2\font}%
        }%

        % Allow breaks at special characters using \PYG... macros.
        \Wrappedbreaksatspecials
        % Breaks at punctuation characters . , ; ? ! and / need catcode=\active
        \OriginalVerbatim[#1,codes*=\Wrappedbreaksatpunct]%
    }
    \makeatother

    % Exact colors from NB
    \definecolor{incolor}{HTML}{303F9F}
    \definecolor{outcolor}{HTML}{D84315}
    \definecolor{cellborder}{HTML}{CFCFCF}
    \definecolor{cellbackground}{HTML}{F7F7F7}

    % prompt
    \makeatletter
    \newcommand{\boxspacing}{\kern\kvtcb@left@rule\kern\kvtcb@boxsep}
    \makeatother
    \newcommand{\prompt}[4]{
        {\ttfamily\llap{{\color{#2}[#3]:\hspace{3pt}#4}}\vspace{-\baselineskip}}
    }
    

    
    % Prevent overflowing lines due to hard-to-break entities
    \sloppy
    % Setup hyperref package
    \hypersetup{
      breaklinks=true,  % so long urls are correctly broken across lines
      colorlinks=true,
      urlcolor=urlcolor,
      linkcolor=linkcolor,
      citecolor=citecolor,
      }
    % Slightly bigger margins than the latex defaults
    
    \geometry{verbose,tmargin=1in,bmargin=1in,lmargin=1in,rmargin=1in}
    
    

\begin{document}
    
    \maketitle
    
    

    
    \begin{tcolorbox}[breakable, size=fbox, boxrule=1pt, pad at break*=1mm,colback=cellbackground, colframe=cellborder]
\prompt{In}{incolor}{ }{\boxspacing}
\begin{Verbatim}[commandchars=\\\{\}]
\PY{c+c1}{\PYZsh{} This mounts your Google Drive to the Colab VM.}
\PY{k+kn}{from} \PY{n+nn}{google}\PY{n+nn}{.}\PY{n+nn}{colab} \PY{k+kn}{import} \PY{n}{drive}
\PY{n}{drive}\PY{o}{.}\PY{n}{mount}\PY{p}{(}\PY{l+s+s1}{\PYZsq{}}\PY{l+s+s1}{/content/drive}\PY{l+s+s1}{\PYZsq{}}\PY{p}{)}

\PY{c+c1}{\PYZsh{} TODO: Enter the foldername in your Drive where you have saved the unzipped}
\PY{c+c1}{\PYZsh{} assignment folder, e.g. \PYZsq{}cs231n/assignments/assignment1/\PYZsq{}}
\PY{n}{FOLDERNAME} \PY{o}{=} \PY{l+s+s1}{\PYZsq{}}\PY{l+s+s1}{cs231n/assignments/assignment1/}\PY{l+s+s1}{\PYZsq{}}
\PY{k}{assert} \PY{n}{FOLDERNAME} \PY{o+ow}{is} \PY{o+ow}{not} \PY{k+kc}{None}\PY{p}{,} \PY{l+s+s2}{\PYZdq{}}\PY{l+s+s2}{[!] Enter the foldername.}\PY{l+s+s2}{\PYZdq{}}

\PY{c+c1}{\PYZsh{} Now that we\PYZsq{}ve mounted your Drive, this ensures that}
\PY{c+c1}{\PYZsh{} the Python interpreter of the Colab VM can load}
\PY{c+c1}{\PYZsh{} python files from within it.}
\PY{k+kn}{import} \PY{n+nn}{sys}
\PY{n}{sys}\PY{o}{.}\PY{n}{path}\PY{o}{.}\PY{n}{append}\PY{p}{(}\PY{l+s+s1}{\PYZsq{}}\PY{l+s+s1}{/content/drive/My Drive/}\PY{l+s+si}{\PYZob{}\PYZcb{}}\PY{l+s+s1}{\PYZsq{}}\PY{o}{.}\PY{n}{format}\PY{p}{(}\PY{n}{FOLDERNAME}\PY{p}{)}\PY{p}{)}

\PY{c+c1}{\PYZsh{} This downloads the CIFAR\PYZhy{}10 dataset to your Drive}
\PY{c+c1}{\PYZsh{} if it doesn\PYZsq{}t already exist.}
\PY{o}{\PYZpc{}}\PY{n}{cd} \PY{o}{/}\PY{n}{content}\PY{o}{/}\PY{n}{drive}\PY{o}{/}\PY{n}{My}\PYZbs{} \PY{n}{Drive}\PY{o}{/}\PY{err}{\PYZdl{}}\PY{n}{FOLDERNAME}\PY{o}{/}\PY{n}{cs231n}\PY{o}{/}\PY{n}{datasets}\PY{o}{/}
\PY{err}{!}\PY{n}{bash} \PY{n}{get\PYZus{}datasets}\PY{o}{.}\PY{n}{sh}
\PY{o}{\PYZpc{}}\PY{n}{cd} \PY{o}{/}\PY{n}{content}\PY{o}{/}\PY{n}{drive}\PY{o}{/}\PY{n}{My}\PYZbs{} \PY{n}{Drive}\PY{o}{/}\PY{err}{\PYZdl{}}\PY{n}{FOLDERNAME}
\end{Verbatim}
\end{tcolorbox}

    \begin{Verbatim}[commandchars=\\\{\}]
Drive already mounted at /content/drive; to attempt to forcibly remount, call
drive.mount("/content/drive", force\_remount=True).
/content/drive/My Drive/cs231n/assignments/assignment1/cs231n/datasets
/content/drive/My Drive/cs231n/assignments/assignment1
    \end{Verbatim}

    \hypertarget{fully-connected-neural-nets}{%
\section{Fully-Connected Neural
Nets}\label{fully-connected-neural-nets}}

In this exercise we will implement fully-connected networks using a
modular approach. For each layer we will implement a \texttt{forward}
and a \texttt{backward} function. The \texttt{forward} function will
receive inputs, weights, and other parameters and will return both an
output and a \texttt{cache} object storing data needed for the backward
pass, like this:

\begin{Shaded}
\begin{Highlighting}[]
\KeywordTok{def}\NormalTok{ layer_forward(x, w):}
  \CommentTok{""" Receive inputs x and weights w """}
  \CommentTok{# Do some computations ...}
\NormalTok{  z }\OperatorTok{=} \CommentTok{# ... some intermediate value}
  \CommentTok{# Do some more computations ...}
\NormalTok{  out }\OperatorTok{=} \CommentTok{# the output}
   
\NormalTok{  cache }\OperatorTok{=}\NormalTok{ (x, w, z, out) }\CommentTok{# Values we need to compute gradients}
   
  \ControlFlowTok{return}\NormalTok{ out, cache}
\end{Highlighting}
\end{Shaded}

The backward pass will receive upstream derivatives and the
\texttt{cache} object, and will return gradients with respect to the
inputs and weights, like this:

\begin{Shaded}
\begin{Highlighting}[]
\KeywordTok{def}\NormalTok{ layer_backward(dout, cache):}
  \CommentTok{"""}
\CommentTok{  Receive dout (derivative of loss with respect to outputs) and cache,}
\CommentTok{  and compute derivative with respect to inputs.}
\CommentTok{  """}
  \CommentTok{# Unpack cache values}
\NormalTok{  x, w, z, out }\OperatorTok{=}\NormalTok{ cache}
  
  \CommentTok{# Use values in cache to compute derivatives}
\NormalTok{  dx }\OperatorTok{=} \CommentTok{# Derivative of loss with respect to x}
\NormalTok{  dw }\OperatorTok{=} \CommentTok{# Derivative of loss with respect to w}
  
  \ControlFlowTok{return}\NormalTok{ dx, dw}
\end{Highlighting}
\end{Shaded}

After implementing a bunch of layers this way, we will be able to easily
combine them to build classifiers with different architectures.

    \begin{tcolorbox}[breakable, size=fbox, boxrule=1pt, pad at break*=1mm,colback=cellbackground, colframe=cellborder]
\prompt{In}{incolor}{ }{\boxspacing}
\begin{Verbatim}[commandchars=\\\{\}]
\PY{c+c1}{\PYZsh{} As usual, a bit of setup}
\PY{k+kn}{from} \PY{n+nn}{\PYZus{}\PYZus{}future\PYZus{}\PYZus{}} \PY{k+kn}{import} \PY{n}{print\PYZus{}function}
\PY{k+kn}{import} \PY{n+nn}{time}
\PY{k+kn}{import} \PY{n+nn}{numpy} \PY{k}{as} \PY{n+nn}{np}
\PY{k+kn}{import} \PY{n+nn}{matplotlib}\PY{n+nn}{.}\PY{n+nn}{pyplot} \PY{k}{as} \PY{n+nn}{plt}
\PY{k+kn}{from} \PY{n+nn}{cs231n}\PY{n+nn}{.}\PY{n+nn}{classifiers}\PY{n+nn}{.}\PY{n+nn}{fc\PYZus{}net} \PY{k+kn}{import} \PY{o}{*}
\PY{k+kn}{from} \PY{n+nn}{cs231n}\PY{n+nn}{.}\PY{n+nn}{data\PYZus{}utils} \PY{k+kn}{import} \PY{n}{get\PYZus{}CIFAR10\PYZus{}data}
\PY{k+kn}{from} \PY{n+nn}{cs231n}\PY{n+nn}{.}\PY{n+nn}{gradient\PYZus{}check} \PY{k+kn}{import} \PY{n}{eval\PYZus{}numerical\PYZus{}gradient}\PY{p}{,} \PY{n}{eval\PYZus{}numerical\PYZus{}gradient\PYZus{}array}
\PY{k+kn}{from} \PY{n+nn}{cs231n}\PY{n+nn}{.}\PY{n+nn}{solver} \PY{k+kn}{import} \PY{n}{Solver}

\PY{o}{\PYZpc{}}\PY{n}{matplotlib} \PY{n}{inline}
\PY{n}{plt}\PY{o}{.}\PY{n}{rcParams}\PY{p}{[}\PY{l+s+s1}{\PYZsq{}}\PY{l+s+s1}{figure.figsize}\PY{l+s+s1}{\PYZsq{}}\PY{p}{]} \PY{o}{=} \PY{p}{(}\PY{l+m+mf}{10.0}\PY{p}{,} \PY{l+m+mf}{8.0}\PY{p}{)} \PY{c+c1}{\PYZsh{} set default size of plots}
\PY{n}{plt}\PY{o}{.}\PY{n}{rcParams}\PY{p}{[}\PY{l+s+s1}{\PYZsq{}}\PY{l+s+s1}{image.interpolation}\PY{l+s+s1}{\PYZsq{}}\PY{p}{]} \PY{o}{=} \PY{l+s+s1}{\PYZsq{}}\PY{l+s+s1}{nearest}\PY{l+s+s1}{\PYZsq{}}
\PY{n}{plt}\PY{o}{.}\PY{n}{rcParams}\PY{p}{[}\PY{l+s+s1}{\PYZsq{}}\PY{l+s+s1}{image.cmap}\PY{l+s+s1}{\PYZsq{}}\PY{p}{]} \PY{o}{=} \PY{l+s+s1}{\PYZsq{}}\PY{l+s+s1}{gray}\PY{l+s+s1}{\PYZsq{}}

\PY{c+c1}{\PYZsh{} for auto\PYZhy{}reloading external modules}
\PY{c+c1}{\PYZsh{} see http://stackoverflow.com/questions/1907993/autoreload\PYZhy{}of\PYZhy{}modules\PYZhy{}in\PYZhy{}ipython}
\PY{o}{\PYZpc{}}\PY{n}{load\PYZus{}ext} \PY{n}{autoreload}
\PY{o}{\PYZpc{}}\PY{n}{autoreload} \PY{l+m+mi}{2}

\PY{k}{def} \PY{n+nf}{rel\PYZus{}error}\PY{p}{(}\PY{n}{x}\PY{p}{,} \PY{n}{y}\PY{p}{)}\PY{p}{:}
  \PY{l+s+sd}{\PYZdq{}\PYZdq{}\PYZdq{} returns relative error \PYZdq{}\PYZdq{}\PYZdq{}}
  \PY{k}{return} \PY{n}{np}\PY{o}{.}\PY{n}{max}\PY{p}{(}\PY{n}{np}\PY{o}{.}\PY{n}{abs}\PY{p}{(}\PY{n}{x} \PY{o}{\PYZhy{}} \PY{n}{y}\PY{p}{)} \PY{o}{/} \PY{p}{(}\PY{n}{np}\PY{o}{.}\PY{n}{maximum}\PY{p}{(}\PY{l+m+mf}{1e\PYZhy{}8}\PY{p}{,} \PY{n}{np}\PY{o}{.}\PY{n}{abs}\PY{p}{(}\PY{n}{x}\PY{p}{)} \PY{o}{+} \PY{n}{np}\PY{o}{.}\PY{n}{abs}\PY{p}{(}\PY{n}{y}\PY{p}{)}\PY{p}{)}\PY{p}{)}\PY{p}{)}
\end{Verbatim}
\end{tcolorbox}

    \begin{Verbatim}[commandchars=\\\{\}]
The autoreload extension is already loaded. To reload it, use:
  \%reload\_ext autoreload
    \end{Verbatim}

    \begin{tcolorbox}[breakable, size=fbox, boxrule=1pt, pad at break*=1mm,colback=cellbackground, colframe=cellborder]
\prompt{In}{incolor}{ }{\boxspacing}
\begin{Verbatim}[commandchars=\\\{\}]
\PY{c+c1}{\PYZsh{} Load the (preprocessed) CIFAR10 data.}

\PY{n}{data} \PY{o}{=} \PY{n}{get\PYZus{}CIFAR10\PYZus{}data}\PY{p}{(}\PY{p}{)}
\PY{k}{for} \PY{n}{k}\PY{p}{,} \PY{n}{v} \PY{o+ow}{in} \PY{n+nb}{list}\PY{p}{(}\PY{n}{data}\PY{o}{.}\PY{n}{items}\PY{p}{(}\PY{p}{)}\PY{p}{)}\PY{p}{:}
  \PY{n+nb}{print}\PY{p}{(}\PY{p}{(}\PY{l+s+s1}{\PYZsq{}}\PY{l+s+si}{\PYZpc{}s}\PY{l+s+s1}{: }\PY{l+s+s1}{\PYZsq{}} \PY{o}{\PYZpc{}} \PY{n}{k}\PY{p}{,} \PY{n}{v}\PY{o}{.}\PY{n}{shape}\PY{p}{)}\PY{p}{)}
\end{Verbatim}
\end{tcolorbox}

    \begin{Verbatim}[commandchars=\\\{\}]
('X\_train: ', (49000, 3, 32, 32))
('y\_train: ', (49000,))
('X\_val: ', (1000, 3, 32, 32))
('y\_val: ', (1000,))
('X\_test: ', (1000, 3, 32, 32))
('y\_test: ', (1000,))
    \end{Verbatim}

    \hypertarget{affine-layer-forward}{%
\section{Affine layer: forward}\label{affine-layer-forward}}

Open the file \texttt{cs231n/layers.py} and implement the
\texttt{affine\_forward} function.

Once you are done you can test your implementaion by running the
following:

    \begin{tcolorbox}[breakable, size=fbox, boxrule=1pt, pad at break*=1mm,colback=cellbackground, colframe=cellborder]
\prompt{In}{incolor}{ }{\boxspacing}
\begin{Verbatim}[commandchars=\\\{\}]
\PY{c+c1}{\PYZsh{} Test the affine\PYZus{}forward function}

\PY{n}{num\PYZus{}inputs} \PY{o}{=} \PY{l+m+mi}{2}
\PY{n}{input\PYZus{}shape} \PY{o}{=} \PY{p}{(}\PY{l+m+mi}{4}\PY{p}{,} \PY{l+m+mi}{5}\PY{p}{,} \PY{l+m+mi}{6}\PY{p}{)}
\PY{n}{output\PYZus{}dim} \PY{o}{=} \PY{l+m+mi}{3} 
\PY{n}{input\PYZus{}size} \PY{o}{=} \PY{n}{num\PYZus{}inputs} \PY{o}{*} \PY{n}{np}\PY{o}{.}\PY{n}{prod}\PY{p}{(}\PY{n}{input\PYZus{}shape}\PY{p}{)}
\PY{n}{weight\PYZus{}size} \PY{o}{=} \PY{n}{output\PYZus{}dim} \PY{o}{*} \PY{n}{np}\PY{o}{.}\PY{n}{prod}\PY{p}{(}\PY{n}{input\PYZus{}shape}\PY{p}{)}

\PY{n}{x} \PY{o}{=} \PY{n}{np}\PY{o}{.}\PY{n}{linspace}\PY{p}{(}\PY{o}{\PYZhy{}}\PY{l+m+mf}{0.1}\PY{p}{,} \PY{l+m+mf}{0.5}\PY{p}{,} \PY{n}{num}\PY{o}{=}\PY{n}{input\PYZus{}size}\PY{p}{)}\PY{o}{.}\PY{n}{reshape}\PY{p}{(}\PY{n}{num\PYZus{}inputs}\PY{p}{,} \PY{o}{*}\PY{n}{input\PYZus{}shape}\PY{p}{)}
\PY{n}{w} \PY{o}{=} \PY{n}{np}\PY{o}{.}\PY{n}{linspace}\PY{p}{(}\PY{o}{\PYZhy{}}\PY{l+m+mf}{0.2}\PY{p}{,} \PY{l+m+mf}{0.3}\PY{p}{,} \PY{n}{num}\PY{o}{=}\PY{n}{weight\PYZus{}size}\PY{p}{)}\PY{o}{.}\PY{n}{reshape}\PY{p}{(}\PY{n}{np}\PY{o}{.}\PY{n}{prod}\PY{p}{(}\PY{n}{input\PYZus{}shape}\PY{p}{)}\PY{p}{,} \PY{n}{output\PYZus{}dim}\PY{p}{)}
\PY{n}{b} \PY{o}{=} \PY{n}{np}\PY{o}{.}\PY{n}{linspace}\PY{p}{(}\PY{o}{\PYZhy{}}\PY{l+m+mf}{0.3}\PY{p}{,} \PY{l+m+mf}{0.1}\PY{p}{,} \PY{n}{num}\PY{o}{=}\PY{n}{output\PYZus{}dim}\PY{p}{)}

\PY{n}{out}\PY{p}{,} \PY{n}{\PYZus{}} \PY{o}{=} \PY{n}{affine\PYZus{}forward}\PY{p}{(}\PY{n}{x}\PY{p}{,} \PY{n}{w}\PY{p}{,} \PY{n}{b}\PY{p}{)}
\PY{n}{correct\PYZus{}out} \PY{o}{=} \PY{n}{np}\PY{o}{.}\PY{n}{array}\PY{p}{(}\PY{p}{[}\PY{p}{[} \PY{l+m+mf}{1.49834967}\PY{p}{,}  \PY{l+m+mf}{1.70660132}\PY{p}{,}  \PY{l+m+mf}{1.91485297}\PY{p}{]}\PY{p}{,}
                        \PY{p}{[} \PY{l+m+mf}{3.25553199}\PY{p}{,}  \PY{l+m+mf}{3.5141327}\PY{p}{,}   \PY{l+m+mf}{3.77273342}\PY{p}{]}\PY{p}{]}\PY{p}{)}

\PY{c+c1}{\PYZsh{} Compare your output with ours. The error should be around e\PYZhy{}9 or less.}
\PY{n+nb}{print}\PY{p}{(}\PY{l+s+s1}{\PYZsq{}}\PY{l+s+s1}{Testing affine\PYZus{}forward function:}\PY{l+s+s1}{\PYZsq{}}\PY{p}{)}
\PY{n+nb}{print}\PY{p}{(}\PY{l+s+s1}{\PYZsq{}}\PY{l+s+s1}{difference: }\PY{l+s+s1}{\PYZsq{}}\PY{p}{,} \PY{n}{rel\PYZus{}error}\PY{p}{(}\PY{n}{out}\PY{p}{,} \PY{n}{correct\PYZus{}out}\PY{p}{)}\PY{p}{)}
\end{Verbatim}
\end{tcolorbox}

    \begin{Verbatim}[commandchars=\\\{\}]
Testing affine\_forward function:
difference:  9.769849468192957e-10
    \end{Verbatim}

    \hypertarget{affine-layer-backward}{%
\section{Affine layer: backward}\label{affine-layer-backward}}

Now implement the \texttt{affine\_backward} function and test your
implementation using numeric gradient checking.

    \begin{tcolorbox}[breakable, size=fbox, boxrule=1pt, pad at break*=1mm,colback=cellbackground, colframe=cellborder]
\prompt{In}{incolor}{ }{\boxspacing}
\begin{Verbatim}[commandchars=\\\{\}]
\PY{c+c1}{\PYZsh{} Test the affine\PYZus{}backward function}
\PY{n}{np}\PY{o}{.}\PY{n}{random}\PY{o}{.}\PY{n}{seed}\PY{p}{(}\PY{l+m+mi}{231}\PY{p}{)}
\PY{n}{x} \PY{o}{=} \PY{n}{np}\PY{o}{.}\PY{n}{random}\PY{o}{.}\PY{n}{randn}\PY{p}{(}\PY{l+m+mi}{10}\PY{p}{,} \PY{l+m+mi}{2}\PY{p}{,} \PY{l+m+mi}{3}\PY{p}{)}
\PY{n}{w} \PY{o}{=} \PY{n}{np}\PY{o}{.}\PY{n}{random}\PY{o}{.}\PY{n}{randn}\PY{p}{(}\PY{l+m+mi}{6}\PY{p}{,} \PY{l+m+mi}{5}\PY{p}{)}
\PY{n}{b} \PY{o}{=} \PY{n}{np}\PY{o}{.}\PY{n}{random}\PY{o}{.}\PY{n}{randn}\PY{p}{(}\PY{l+m+mi}{5}\PY{p}{)}
\PY{n}{dout} \PY{o}{=} \PY{n}{np}\PY{o}{.}\PY{n}{random}\PY{o}{.}\PY{n}{randn}\PY{p}{(}\PY{l+m+mi}{10}\PY{p}{,} \PY{l+m+mi}{5}\PY{p}{)}

\PY{n}{dx\PYZus{}num} \PY{o}{=} \PY{n}{eval\PYZus{}numerical\PYZus{}gradient\PYZus{}array}\PY{p}{(}\PY{k}{lambda} \PY{n}{x}\PY{p}{:} \PY{n}{affine\PYZus{}forward}\PY{p}{(}\PY{n}{x}\PY{p}{,} \PY{n}{w}\PY{p}{,} \PY{n}{b}\PY{p}{)}\PY{p}{[}\PY{l+m+mi}{0}\PY{p}{]}\PY{p}{,} \PY{n}{x}\PY{p}{,} \PY{n}{dout}\PY{p}{)}
\PY{n}{dw\PYZus{}num} \PY{o}{=} \PY{n}{eval\PYZus{}numerical\PYZus{}gradient\PYZus{}array}\PY{p}{(}\PY{k}{lambda} \PY{n}{w}\PY{p}{:} \PY{n}{affine\PYZus{}forward}\PY{p}{(}\PY{n}{x}\PY{p}{,} \PY{n}{w}\PY{p}{,} \PY{n}{b}\PY{p}{)}\PY{p}{[}\PY{l+m+mi}{0}\PY{p}{]}\PY{p}{,} \PY{n}{w}\PY{p}{,} \PY{n}{dout}\PY{p}{)}
\PY{n}{db\PYZus{}num} \PY{o}{=} \PY{n}{eval\PYZus{}numerical\PYZus{}gradient\PYZus{}array}\PY{p}{(}\PY{k}{lambda} \PY{n}{b}\PY{p}{:} \PY{n}{affine\PYZus{}forward}\PY{p}{(}\PY{n}{x}\PY{p}{,} \PY{n}{w}\PY{p}{,} \PY{n}{b}\PY{p}{)}\PY{p}{[}\PY{l+m+mi}{0}\PY{p}{]}\PY{p}{,} \PY{n}{b}\PY{p}{,} \PY{n}{dout}\PY{p}{)}

\PY{n}{\PYZus{}}\PY{p}{,} \PY{n}{cache} \PY{o}{=} \PY{n}{affine\PYZus{}forward}\PY{p}{(}\PY{n}{x}\PY{p}{,} \PY{n}{w}\PY{p}{,} \PY{n}{b}\PY{p}{)}
\PY{n}{dx}\PY{p}{,} \PY{n}{dw}\PY{p}{,} \PY{n}{db} \PY{o}{=} \PY{n}{affine\PYZus{}backward}\PY{p}{(}\PY{n}{dout}\PY{p}{,} \PY{n}{cache}\PY{p}{)}

\PY{c+c1}{\PYZsh{} The error should be around e\PYZhy{}10 or less}
\PY{n+nb}{print}\PY{p}{(}\PY{l+s+s1}{\PYZsq{}}\PY{l+s+s1}{Testing affine\PYZus{}backward function:}\PY{l+s+s1}{\PYZsq{}}\PY{p}{)}
\PY{n+nb}{print}\PY{p}{(}\PY{l+s+s1}{\PYZsq{}}\PY{l+s+s1}{dx error: }\PY{l+s+s1}{\PYZsq{}}\PY{p}{,} \PY{n}{rel\PYZus{}error}\PY{p}{(}\PY{n}{dx\PYZus{}num}\PY{p}{,} \PY{n}{dx}\PY{p}{)}\PY{p}{)}
\PY{n+nb}{print}\PY{p}{(}\PY{l+s+s1}{\PYZsq{}}\PY{l+s+s1}{dw error: }\PY{l+s+s1}{\PYZsq{}}\PY{p}{,} \PY{n}{rel\PYZus{}error}\PY{p}{(}\PY{n}{dw\PYZus{}num}\PY{p}{,} \PY{n}{dw}\PY{p}{)}\PY{p}{)}
\PY{n+nb}{print}\PY{p}{(}\PY{l+s+s1}{\PYZsq{}}\PY{l+s+s1}{db error: }\PY{l+s+s1}{\PYZsq{}}\PY{p}{,} \PY{n}{rel\PYZus{}error}\PY{p}{(}\PY{n}{db\PYZus{}num}\PY{p}{,} \PY{n}{db}\PY{p}{)}\PY{p}{)}
\end{Verbatim}
\end{tcolorbox}

    \begin{Verbatim}[commandchars=\\\{\}]
Testing affine\_backward function:
dx error:  5.399100368651805e-11
dw error:  9.904211865398145e-11
db error:  2.4122867568119087e-11
    \end{Verbatim}

    \hypertarget{relu-activation-forward}{%
\section{ReLU activation: forward}\label{relu-activation-forward}}

Implement the forward pass for the ReLU activation function in the
\texttt{relu\_forward} function and test your implementation using the
following:

    \begin{tcolorbox}[breakable, size=fbox, boxrule=1pt, pad at break*=1mm,colback=cellbackground, colframe=cellborder]
\prompt{In}{incolor}{ }{\boxspacing}
\begin{Verbatim}[commandchars=\\\{\}]
\PY{c+c1}{\PYZsh{} Test the relu\PYZus{}forward function}

\PY{n}{x} \PY{o}{=} \PY{n}{np}\PY{o}{.}\PY{n}{linspace}\PY{p}{(}\PY{o}{\PYZhy{}}\PY{l+m+mf}{0.5}\PY{p}{,} \PY{l+m+mf}{0.5}\PY{p}{,} \PY{n}{num}\PY{o}{=}\PY{l+m+mi}{12}\PY{p}{)}\PY{o}{.}\PY{n}{reshape}\PY{p}{(}\PY{l+m+mi}{3}\PY{p}{,} \PY{l+m+mi}{4}\PY{p}{)}

\PY{n}{out}\PY{p}{,} \PY{n}{\PYZus{}} \PY{o}{=} \PY{n}{relu\PYZus{}forward}\PY{p}{(}\PY{n}{x}\PY{p}{)}
\PY{n}{correct\PYZus{}out} \PY{o}{=} \PY{n}{np}\PY{o}{.}\PY{n}{array}\PY{p}{(}\PY{p}{[}\PY{p}{[} \PY{l+m+mf}{0.}\PY{p}{,}          \PY{l+m+mf}{0.}\PY{p}{,}          \PY{l+m+mf}{0.}\PY{p}{,}          \PY{l+m+mf}{0.}\PY{p}{,}        \PY{p}{]}\PY{p}{,}
                        \PY{p}{[} \PY{l+m+mf}{0.}\PY{p}{,}          \PY{l+m+mf}{0.}\PY{p}{,}          \PY{l+m+mf}{0.04545455}\PY{p}{,}  \PY{l+m+mf}{0.13636364}\PY{p}{,}\PY{p}{]}\PY{p}{,}
                        \PY{p}{[} \PY{l+m+mf}{0.22727273}\PY{p}{,}  \PY{l+m+mf}{0.31818182}\PY{p}{,}  \PY{l+m+mf}{0.40909091}\PY{p}{,}  \PY{l+m+mf}{0.5}\PY{p}{,}       \PY{p}{]}\PY{p}{]}\PY{p}{)}

\PY{c+c1}{\PYZsh{} Compare your output with ours. The error should be on the order of e\PYZhy{}8}
\PY{n+nb}{print}\PY{p}{(}\PY{l+s+s1}{\PYZsq{}}\PY{l+s+s1}{Testing relu\PYZus{}forward function:}\PY{l+s+s1}{\PYZsq{}}\PY{p}{)}
\PY{n+nb}{print}\PY{p}{(}\PY{l+s+s1}{\PYZsq{}}\PY{l+s+s1}{difference: }\PY{l+s+s1}{\PYZsq{}}\PY{p}{,} \PY{n}{rel\PYZus{}error}\PY{p}{(}\PY{n}{out}\PY{p}{,} \PY{n}{correct\PYZus{}out}\PY{p}{)}\PY{p}{)}
\end{Verbatim}
\end{tcolorbox}

    \begin{Verbatim}[commandchars=\\\{\}]
Testing relu\_forward function:
difference:  4.999999798022158e-08
    \end{Verbatim}

    \hypertarget{relu-activation-backward}{%
\section{ReLU activation: backward}\label{relu-activation-backward}}

Now implement the backward pass for the ReLU activation function in the
\texttt{relu\_backward} function and test your implementation using
numeric gradient checking:

    \begin{tcolorbox}[breakable, size=fbox, boxrule=1pt, pad at break*=1mm,colback=cellbackground, colframe=cellborder]
\prompt{In}{incolor}{ }{\boxspacing}
\begin{Verbatim}[commandchars=\\\{\}]
\PY{n}{np}\PY{o}{.}\PY{n}{random}\PY{o}{.}\PY{n}{seed}\PY{p}{(}\PY{l+m+mi}{231}\PY{p}{)}
\PY{n}{x} \PY{o}{=} \PY{n}{np}\PY{o}{.}\PY{n}{random}\PY{o}{.}\PY{n}{randn}\PY{p}{(}\PY{l+m+mi}{10}\PY{p}{,} \PY{l+m+mi}{10}\PY{p}{)}
\PY{n}{dout} \PY{o}{=} \PY{n}{np}\PY{o}{.}\PY{n}{random}\PY{o}{.}\PY{n}{randn}\PY{p}{(}\PY{o}{*}\PY{n}{x}\PY{o}{.}\PY{n}{shape}\PY{p}{)}

\PY{n}{dx\PYZus{}num} \PY{o}{=} \PY{n}{eval\PYZus{}numerical\PYZus{}gradient\PYZus{}array}\PY{p}{(}\PY{k}{lambda} \PY{n}{x}\PY{p}{:} \PY{n}{relu\PYZus{}forward}\PY{p}{(}\PY{n}{x}\PY{p}{)}\PY{p}{[}\PY{l+m+mi}{0}\PY{p}{]}\PY{p}{,} \PY{n}{x}\PY{p}{,} \PY{n}{dout}\PY{p}{)}

\PY{n}{\PYZus{}}\PY{p}{,} \PY{n}{cache} \PY{o}{=} \PY{n}{relu\PYZus{}forward}\PY{p}{(}\PY{n}{x}\PY{p}{)}
\PY{n}{dx} \PY{o}{=} \PY{n}{relu\PYZus{}backward}\PY{p}{(}\PY{n}{dout}\PY{p}{,} \PY{n}{cache}\PY{p}{)}

\PY{c+c1}{\PYZsh{} The error should be on the order of e\PYZhy{}12}
\PY{n+nb}{print}\PY{p}{(}\PY{l+s+s1}{\PYZsq{}}\PY{l+s+s1}{Testing relu\PYZus{}backward function:}\PY{l+s+s1}{\PYZsq{}}\PY{p}{)}
\PY{n+nb}{print}\PY{p}{(}\PY{l+s+s1}{\PYZsq{}}\PY{l+s+s1}{dx error: }\PY{l+s+s1}{\PYZsq{}}\PY{p}{,} \PY{n}{rel\PYZus{}error}\PY{p}{(}\PY{n}{dx\PYZus{}num}\PY{p}{,} \PY{n}{dx}\PY{p}{)}\PY{p}{)}
\end{Verbatim}
\end{tcolorbox}

    \begin{Verbatim}[commandchars=\\\{\}]
Testing relu\_backward function:
dx error:  3.2756349136310288e-12
    \end{Verbatim}

    \hypertarget{inline-question-1}{%
\subsection{Inline Question 1:}\label{inline-question-1}}

We've only asked you to implement ReLU, but there are a number of
different activation functions that one could use in neural networks,
each with its pros and cons. In particular, an issue commonly seen with
activation functions is getting zero (or close to zero) gradient flow
during backpropagation. Which of the following activation functions have
this problem? If you consider these functions in the one dimensional
case, what types of input would lead to this behaviour? 1. Sigmoid 2.
ReLU 3. Leaky ReLU

\hypertarget{answer}{%
\subsection{Answer:}\label{answer}}

\begin{enumerate}
\def\labelenumi{\arabic{enumi}.}
\item
  在\({x\to\infty}\)的时候, gradient会趋近于0.
\item
  因为Relu形式为\(\max(0,x)\),当\(x<0\)的时候就会出现gradient为0的情况.
\item
  \(x<0\)的时候会趋近于0.
\end{enumerate}

    \hypertarget{sandwich-layers}{%
\section{``Sandwich'' layers}\label{sandwich-layers}}

There are some common patterns of layers that are frequently used in
neural nets. For example, affine layers are frequently followed by a
ReLU nonlinearity. To make these common patterns easy, we define several
convenience layers in the file \texttt{cs231n/layer\_utils.py}.

For now take a look at the \texttt{affine\_relu\_forward} and
\texttt{affine\_relu\_backward} functions, and run the following to
numerically gradient check the backward pass:

    \begin{tcolorbox}[breakable, size=fbox, boxrule=1pt, pad at break*=1mm,colback=cellbackground, colframe=cellborder]
\prompt{In}{incolor}{ }{\boxspacing}
\begin{Verbatim}[commandchars=\\\{\}]
\PY{k+kn}{from} \PY{n+nn}{cs231n}\PY{n+nn}{.}\PY{n+nn}{layer\PYZus{}utils} \PY{k+kn}{import} \PY{n}{affine\PYZus{}relu\PYZus{}forward}\PY{p}{,} \PY{n}{affine\PYZus{}relu\PYZus{}backward}
\PY{n}{np}\PY{o}{.}\PY{n}{random}\PY{o}{.}\PY{n}{seed}\PY{p}{(}\PY{l+m+mi}{231}\PY{p}{)}
\PY{n}{x} \PY{o}{=} \PY{n}{np}\PY{o}{.}\PY{n}{random}\PY{o}{.}\PY{n}{randn}\PY{p}{(}\PY{l+m+mi}{2}\PY{p}{,} \PY{l+m+mi}{3}\PY{p}{,} \PY{l+m+mi}{4}\PY{p}{)}
\PY{n}{w} \PY{o}{=} \PY{n}{np}\PY{o}{.}\PY{n}{random}\PY{o}{.}\PY{n}{randn}\PY{p}{(}\PY{l+m+mi}{12}\PY{p}{,} \PY{l+m+mi}{10}\PY{p}{)}
\PY{n}{b} \PY{o}{=} \PY{n}{np}\PY{o}{.}\PY{n}{random}\PY{o}{.}\PY{n}{randn}\PY{p}{(}\PY{l+m+mi}{10}\PY{p}{)}
\PY{n}{dout} \PY{o}{=} \PY{n}{np}\PY{o}{.}\PY{n}{random}\PY{o}{.}\PY{n}{randn}\PY{p}{(}\PY{l+m+mi}{2}\PY{p}{,} \PY{l+m+mi}{10}\PY{p}{)}

\PY{n}{out}\PY{p}{,} \PY{n}{cache} \PY{o}{=} \PY{n}{affine\PYZus{}relu\PYZus{}forward}\PY{p}{(}\PY{n}{x}\PY{p}{,} \PY{n}{w}\PY{p}{,} \PY{n}{b}\PY{p}{)}
\PY{n}{dx}\PY{p}{,} \PY{n}{dw}\PY{p}{,} \PY{n}{db} \PY{o}{=} \PY{n}{affine\PYZus{}relu\PYZus{}backward}\PY{p}{(}\PY{n}{dout}\PY{p}{,} \PY{n}{cache}\PY{p}{)}

\PY{n}{dx\PYZus{}num} \PY{o}{=} \PY{n}{eval\PYZus{}numerical\PYZus{}gradient\PYZus{}array}\PY{p}{(}\PY{k}{lambda} \PY{n}{x}\PY{p}{:} \PY{n}{affine\PYZus{}relu\PYZus{}forward}\PY{p}{(}\PY{n}{x}\PY{p}{,} \PY{n}{w}\PY{p}{,} \PY{n}{b}\PY{p}{)}\PY{p}{[}\PY{l+m+mi}{0}\PY{p}{]}\PY{p}{,} \PY{n}{x}\PY{p}{,} \PY{n}{dout}\PY{p}{)}
\PY{n}{dw\PYZus{}num} \PY{o}{=} \PY{n}{eval\PYZus{}numerical\PYZus{}gradient\PYZus{}array}\PY{p}{(}\PY{k}{lambda} \PY{n}{w}\PY{p}{:} \PY{n}{affine\PYZus{}relu\PYZus{}forward}\PY{p}{(}\PY{n}{x}\PY{p}{,} \PY{n}{w}\PY{p}{,} \PY{n}{b}\PY{p}{)}\PY{p}{[}\PY{l+m+mi}{0}\PY{p}{]}\PY{p}{,} \PY{n}{w}\PY{p}{,} \PY{n}{dout}\PY{p}{)}
\PY{n}{db\PYZus{}num} \PY{o}{=} \PY{n}{eval\PYZus{}numerical\PYZus{}gradient\PYZus{}array}\PY{p}{(}\PY{k}{lambda} \PY{n}{b}\PY{p}{:} \PY{n}{affine\PYZus{}relu\PYZus{}forward}\PY{p}{(}\PY{n}{x}\PY{p}{,} \PY{n}{w}\PY{p}{,} \PY{n}{b}\PY{p}{)}\PY{p}{[}\PY{l+m+mi}{0}\PY{p}{]}\PY{p}{,} \PY{n}{b}\PY{p}{,} \PY{n}{dout}\PY{p}{)}

\PY{c+c1}{\PYZsh{} Relative error should be around e\PYZhy{}10 or less}
\PY{n+nb}{print}\PY{p}{(}\PY{l+s+s1}{\PYZsq{}}\PY{l+s+s1}{Testing affine\PYZus{}relu\PYZus{}forward and affine\PYZus{}relu\PYZus{}backward:}\PY{l+s+s1}{\PYZsq{}}\PY{p}{)}
\PY{n+nb}{print}\PY{p}{(}\PY{l+s+s1}{\PYZsq{}}\PY{l+s+s1}{dx error: }\PY{l+s+s1}{\PYZsq{}}\PY{p}{,} \PY{n}{rel\PYZus{}error}\PY{p}{(}\PY{n}{dx\PYZus{}num}\PY{p}{,} \PY{n}{dx}\PY{p}{)}\PY{p}{)}
\PY{n+nb}{print}\PY{p}{(}\PY{l+s+s1}{\PYZsq{}}\PY{l+s+s1}{dw error: }\PY{l+s+s1}{\PYZsq{}}\PY{p}{,} \PY{n}{rel\PYZus{}error}\PY{p}{(}\PY{n}{dw\PYZus{}num}\PY{p}{,} \PY{n}{dw}\PY{p}{)}\PY{p}{)}
\PY{n+nb}{print}\PY{p}{(}\PY{l+s+s1}{\PYZsq{}}\PY{l+s+s1}{db error: }\PY{l+s+s1}{\PYZsq{}}\PY{p}{,} \PY{n}{rel\PYZus{}error}\PY{p}{(}\PY{n}{db\PYZus{}num}\PY{p}{,} \PY{n}{db}\PY{p}{)}\PY{p}{)}
\end{Verbatim}
\end{tcolorbox}

    \begin{Verbatim}[commandchars=\\\{\}]
Testing affine\_relu\_forward and affine\_relu\_backward:
dx error:  2.299579177309368e-11
dw error:  8.162011105764925e-11
db error:  7.826724021458994e-12
    \end{Verbatim}

    \hypertarget{loss-layers-softmax-and-svm}{%
\section{Loss layers: Softmax and
SVM}\label{loss-layers-softmax-and-svm}}

Now implement the loss and gradient for softmax and SVM in the
\texttt{softmax\_loss} and \texttt{svm\_loss} function in
\texttt{cs231n/layers.py}. These should be similar to what you
implemented in \texttt{cs231n/classifiers/softmax.py} and
\texttt{cs231n/classifiers/linear\_svm.py}.

You can make sure that the implementations are correct by running the
following:

    \begin{tcolorbox}[breakable, size=fbox, boxrule=1pt, pad at break*=1mm,colback=cellbackground, colframe=cellborder]
\prompt{In}{incolor}{ }{\boxspacing}
\begin{Verbatim}[commandchars=\\\{\}]
\PY{n}{np}\PY{o}{.}\PY{n}{random}\PY{o}{.}\PY{n}{seed}\PY{p}{(}\PY{l+m+mi}{231}\PY{p}{)}
\PY{n}{num\PYZus{}classes}\PY{p}{,} \PY{n}{num\PYZus{}inputs} \PY{o}{=} \PY{l+m+mi}{10}\PY{p}{,} \PY{l+m+mi}{50}
\PY{n}{x} \PY{o}{=} \PY{l+m+mf}{0.001} \PY{o}{*} \PY{n}{np}\PY{o}{.}\PY{n}{random}\PY{o}{.}\PY{n}{randn}\PY{p}{(}\PY{n}{num\PYZus{}inputs}\PY{p}{,} \PY{n}{num\PYZus{}classes}\PY{p}{)}
\PY{n}{y} \PY{o}{=} \PY{n}{np}\PY{o}{.}\PY{n}{random}\PY{o}{.}\PY{n}{randint}\PY{p}{(}\PY{n}{num\PYZus{}classes}\PY{p}{,} \PY{n}{size}\PY{o}{=}\PY{n}{num\PYZus{}inputs}\PY{p}{)}

\PY{n}{dx\PYZus{}num} \PY{o}{=} \PY{n}{eval\PYZus{}numerical\PYZus{}gradient}\PY{p}{(}\PY{k}{lambda} \PY{n}{x}\PY{p}{:} \PY{n}{svm\PYZus{}loss}\PY{p}{(}\PY{n}{x}\PY{p}{,} \PY{n}{y}\PY{p}{)}\PY{p}{[}\PY{l+m+mi}{0}\PY{p}{]}\PY{p}{,} \PY{n}{x}\PY{p}{,} \PY{n}{verbose}\PY{o}{=}\PY{k+kc}{False}\PY{p}{)}
\PY{n}{loss}\PY{p}{,} \PY{n}{dx} \PY{o}{=} \PY{n}{svm\PYZus{}loss}\PY{p}{(}\PY{n}{x}\PY{p}{,} \PY{n}{y}\PY{p}{)}

\PY{c+c1}{\PYZsh{} Test svm\PYZus{}loss function. Loss should be around 9 and dx error should be around the order of e\PYZhy{}9}
\PY{n+nb}{print}\PY{p}{(}\PY{l+s+s1}{\PYZsq{}}\PY{l+s+s1}{Testing svm\PYZus{}loss:}\PY{l+s+s1}{\PYZsq{}}\PY{p}{)}
\PY{n+nb}{print}\PY{p}{(}\PY{l+s+s1}{\PYZsq{}}\PY{l+s+s1}{loss: }\PY{l+s+s1}{\PYZsq{}}\PY{p}{,} \PY{n}{loss}\PY{p}{)}
\PY{n+nb}{print}\PY{p}{(}\PY{l+s+s1}{\PYZsq{}}\PY{l+s+s1}{dx error: }\PY{l+s+s1}{\PYZsq{}}\PY{p}{,} \PY{n}{rel\PYZus{}error}\PY{p}{(}\PY{n}{dx\PYZus{}num}\PY{p}{,} \PY{n}{dx}\PY{p}{)}\PY{p}{)}

\PY{n}{dx\PYZus{}num} \PY{o}{=} \PY{n}{eval\PYZus{}numerical\PYZus{}gradient}\PY{p}{(}\PY{k}{lambda} \PY{n}{x}\PY{p}{:} \PY{n}{softmax\PYZus{}loss}\PY{p}{(}\PY{n}{x}\PY{p}{,} \PY{n}{y}\PY{p}{)}\PY{p}{[}\PY{l+m+mi}{0}\PY{p}{]}\PY{p}{,} \PY{n}{x}\PY{p}{,} \PY{n}{verbose}\PY{o}{=}\PY{k+kc}{False}\PY{p}{)}
\PY{n}{loss}\PY{p}{,} \PY{n}{dx} \PY{o}{=} \PY{n}{softmax\PYZus{}loss}\PY{p}{(}\PY{n}{x}\PY{p}{,} \PY{n}{y}\PY{p}{)}

\PY{c+c1}{\PYZsh{} Test softmax\PYZus{}loss function. Loss should be close to 2.3 and dx error should be around e\PYZhy{}8}
\PY{n+nb}{print}\PY{p}{(}\PY{l+s+s1}{\PYZsq{}}\PY{l+s+se}{\PYZbs{}n}\PY{l+s+s1}{Testing softmax\PYZus{}loss:}\PY{l+s+s1}{\PYZsq{}}\PY{p}{)}
\PY{n+nb}{print}\PY{p}{(}\PY{l+s+s1}{\PYZsq{}}\PY{l+s+s1}{loss: }\PY{l+s+s1}{\PYZsq{}}\PY{p}{,} \PY{n}{loss}\PY{p}{)}
\PY{n+nb}{print}\PY{p}{(}\PY{l+s+s1}{\PYZsq{}}\PY{l+s+s1}{dx error: }\PY{l+s+s1}{\PYZsq{}}\PY{p}{,} \PY{n}{rel\PYZus{}error}\PY{p}{(}\PY{n}{dx\PYZus{}num}\PY{p}{,} \PY{n}{dx}\PY{p}{)}\PY{p}{)}
\end{Verbatim}
\end{tcolorbox}

    \begin{Verbatim}[commandchars=\\\{\}]
Testing svm\_loss:
loss:  8.999602749096233
dx error:  1.4021566006651672e-09

Testing softmax\_loss:
loss:  2.302545844500738
dx error:  9.341692129091487e-09
    \end{Verbatim}

    \hypertarget{two-layer-network}{%
\section{Two-layer network}\label{two-layer-network}}

Open the file \texttt{cs231n/classifiers/fc\_net.py} and complete the
implementation of the \texttt{TwoLayerNet} class. Read through it to
make sure you understand the API. You can run the cell below to test
your implementation.

    \begin{tcolorbox}[breakable, size=fbox, boxrule=1pt, pad at break*=1mm,colback=cellbackground, colframe=cellborder]
\prompt{In}{incolor}{ }{\boxspacing}
\begin{Verbatim}[commandchars=\\\{\}]
\PY{n}{np}\PY{o}{.}\PY{n}{random}\PY{o}{.}\PY{n}{seed}\PY{p}{(}\PY{l+m+mi}{231}\PY{p}{)}
\PY{n}{N}\PY{p}{,} \PY{n}{D}\PY{p}{,} \PY{n}{H}\PY{p}{,} \PY{n}{C} \PY{o}{=} \PY{l+m+mi}{3}\PY{p}{,} \PY{l+m+mi}{5}\PY{p}{,} \PY{l+m+mi}{50}\PY{p}{,} \PY{l+m+mi}{7}
\PY{n}{X} \PY{o}{=} \PY{n}{np}\PY{o}{.}\PY{n}{random}\PY{o}{.}\PY{n}{randn}\PY{p}{(}\PY{n}{N}\PY{p}{,} \PY{n}{D}\PY{p}{)}
\PY{n}{y} \PY{o}{=} \PY{n}{np}\PY{o}{.}\PY{n}{random}\PY{o}{.}\PY{n}{randint}\PY{p}{(}\PY{n}{C}\PY{p}{,} \PY{n}{size}\PY{o}{=}\PY{n}{N}\PY{p}{)}

\PY{n}{std} \PY{o}{=} \PY{l+m+mf}{1e\PYZhy{}3}
\PY{n}{model} \PY{o}{=} \PY{n}{TwoLayerNet}\PY{p}{(}\PY{n}{input\PYZus{}dim}\PY{o}{=}\PY{n}{D}\PY{p}{,} \PY{n}{hidden\PYZus{}dim}\PY{o}{=}\PY{n}{H}\PY{p}{,} \PY{n}{num\PYZus{}classes}\PY{o}{=}\PY{n}{C}\PY{p}{,} \PY{n}{weight\PYZus{}scale}\PY{o}{=}\PY{n}{std}\PY{p}{)}

\PY{n+nb}{print}\PY{p}{(}\PY{l+s+s1}{\PYZsq{}}\PY{l+s+s1}{Testing initialization ... }\PY{l+s+s1}{\PYZsq{}}\PY{p}{)}
\PY{n}{W1\PYZus{}std} \PY{o}{=} \PY{n+nb}{abs}\PY{p}{(}\PY{n}{model}\PY{o}{.}\PY{n}{params}\PY{p}{[}\PY{l+s+s1}{\PYZsq{}}\PY{l+s+s1}{W1}\PY{l+s+s1}{\PYZsq{}}\PY{p}{]}\PY{o}{.}\PY{n}{std}\PY{p}{(}\PY{p}{)} \PY{o}{\PYZhy{}} \PY{n}{std}\PY{p}{)}
\PY{n}{b1} \PY{o}{=} \PY{n}{model}\PY{o}{.}\PY{n}{params}\PY{p}{[}\PY{l+s+s1}{\PYZsq{}}\PY{l+s+s1}{b1}\PY{l+s+s1}{\PYZsq{}}\PY{p}{]}
\PY{n}{W2\PYZus{}std} \PY{o}{=} \PY{n+nb}{abs}\PY{p}{(}\PY{n}{model}\PY{o}{.}\PY{n}{params}\PY{p}{[}\PY{l+s+s1}{\PYZsq{}}\PY{l+s+s1}{W2}\PY{l+s+s1}{\PYZsq{}}\PY{p}{]}\PY{o}{.}\PY{n}{std}\PY{p}{(}\PY{p}{)} \PY{o}{\PYZhy{}} \PY{n}{std}\PY{p}{)}
\PY{n}{b2} \PY{o}{=} \PY{n}{model}\PY{o}{.}\PY{n}{params}\PY{p}{[}\PY{l+s+s1}{\PYZsq{}}\PY{l+s+s1}{b2}\PY{l+s+s1}{\PYZsq{}}\PY{p}{]}
\PY{k}{assert} \PY{n}{W1\PYZus{}std} \PY{o}{\PYZlt{}} \PY{n}{std} \PY{o}{/} \PY{l+m+mi}{10}\PY{p}{,} \PY{l+s+s1}{\PYZsq{}}\PY{l+s+s1}{First layer weights do not seem right}\PY{l+s+s1}{\PYZsq{}}
\PY{k}{assert} \PY{n}{np}\PY{o}{.}\PY{n}{all}\PY{p}{(}\PY{n}{b1} \PY{o}{==} \PY{l+m+mi}{0}\PY{p}{)}\PY{p}{,} \PY{l+s+s1}{\PYZsq{}}\PY{l+s+s1}{First layer biases do not seem right}\PY{l+s+s1}{\PYZsq{}}
\PY{k}{assert} \PY{n}{W2\PYZus{}std} \PY{o}{\PYZlt{}} \PY{n}{std} \PY{o}{/} \PY{l+m+mi}{10}\PY{p}{,} \PY{l+s+s1}{\PYZsq{}}\PY{l+s+s1}{Second layer weights do not seem right}\PY{l+s+s1}{\PYZsq{}}
\PY{k}{assert} \PY{n}{np}\PY{o}{.}\PY{n}{all}\PY{p}{(}\PY{n}{b2} \PY{o}{==} \PY{l+m+mi}{0}\PY{p}{)}\PY{p}{,} \PY{l+s+s1}{\PYZsq{}}\PY{l+s+s1}{Second layer biases do not seem right}\PY{l+s+s1}{\PYZsq{}}

\PY{n+nb}{print}\PY{p}{(}\PY{l+s+s1}{\PYZsq{}}\PY{l+s+s1}{Testing test\PYZhy{}time forward pass ... }\PY{l+s+s1}{\PYZsq{}}\PY{p}{)}
\PY{n}{model}\PY{o}{.}\PY{n}{params}\PY{p}{[}\PY{l+s+s1}{\PYZsq{}}\PY{l+s+s1}{W1}\PY{l+s+s1}{\PYZsq{}}\PY{p}{]} \PY{o}{=} \PY{n}{np}\PY{o}{.}\PY{n}{linspace}\PY{p}{(}\PY{o}{\PYZhy{}}\PY{l+m+mf}{0.7}\PY{p}{,} \PY{l+m+mf}{0.3}\PY{p}{,} \PY{n}{num}\PY{o}{=}\PY{n}{D}\PY{o}{*}\PY{n}{H}\PY{p}{)}\PY{o}{.}\PY{n}{reshape}\PY{p}{(}\PY{n}{D}\PY{p}{,} \PY{n}{H}\PY{p}{)}
\PY{n}{model}\PY{o}{.}\PY{n}{params}\PY{p}{[}\PY{l+s+s1}{\PYZsq{}}\PY{l+s+s1}{b1}\PY{l+s+s1}{\PYZsq{}}\PY{p}{]} \PY{o}{=} \PY{n}{np}\PY{o}{.}\PY{n}{linspace}\PY{p}{(}\PY{o}{\PYZhy{}}\PY{l+m+mf}{0.1}\PY{p}{,} \PY{l+m+mf}{0.9}\PY{p}{,} \PY{n}{num}\PY{o}{=}\PY{n}{H}\PY{p}{)}
\PY{n}{model}\PY{o}{.}\PY{n}{params}\PY{p}{[}\PY{l+s+s1}{\PYZsq{}}\PY{l+s+s1}{W2}\PY{l+s+s1}{\PYZsq{}}\PY{p}{]} \PY{o}{=} \PY{n}{np}\PY{o}{.}\PY{n}{linspace}\PY{p}{(}\PY{o}{\PYZhy{}}\PY{l+m+mf}{0.3}\PY{p}{,} \PY{l+m+mf}{0.4}\PY{p}{,} \PY{n}{num}\PY{o}{=}\PY{n}{H}\PY{o}{*}\PY{n}{C}\PY{p}{)}\PY{o}{.}\PY{n}{reshape}\PY{p}{(}\PY{n}{H}\PY{p}{,} \PY{n}{C}\PY{p}{)}
\PY{n}{model}\PY{o}{.}\PY{n}{params}\PY{p}{[}\PY{l+s+s1}{\PYZsq{}}\PY{l+s+s1}{b2}\PY{l+s+s1}{\PYZsq{}}\PY{p}{]} \PY{o}{=} \PY{n}{np}\PY{o}{.}\PY{n}{linspace}\PY{p}{(}\PY{o}{\PYZhy{}}\PY{l+m+mf}{0.9}\PY{p}{,} \PY{l+m+mf}{0.1}\PY{p}{,} \PY{n}{num}\PY{o}{=}\PY{n}{C}\PY{p}{)}
\PY{n}{X} \PY{o}{=} \PY{n}{np}\PY{o}{.}\PY{n}{linspace}\PY{p}{(}\PY{o}{\PYZhy{}}\PY{l+m+mf}{5.5}\PY{p}{,} \PY{l+m+mf}{4.5}\PY{p}{,} \PY{n}{num}\PY{o}{=}\PY{n}{N}\PY{o}{*}\PY{n}{D}\PY{p}{)}\PY{o}{.}\PY{n}{reshape}\PY{p}{(}\PY{n}{D}\PY{p}{,} \PY{n}{N}\PY{p}{)}\PY{o}{.}\PY{n}{T}
\PY{n}{scores} \PY{o}{=} \PY{n}{model}\PY{o}{.}\PY{n}{loss}\PY{p}{(}\PY{n}{X}\PY{p}{)}
\PY{n}{correct\PYZus{}scores} \PY{o}{=} \PY{n}{np}\PY{o}{.}\PY{n}{asarray}\PY{p}{(}
  \PY{p}{[}\PY{p}{[}\PY{l+m+mf}{11.53165108}\PY{p}{,}  \PY{l+m+mf}{12.2917344}\PY{p}{,}   \PY{l+m+mf}{13.05181771}\PY{p}{,}  \PY{l+m+mf}{13.81190102}\PY{p}{,}  \PY{l+m+mf}{14.57198434}\PY{p}{,} \PY{l+m+mf}{15.33206765}\PY{p}{,}  \PY{l+m+mf}{16.09215096}\PY{p}{]}\PY{p}{,}
   \PY{p}{[}\PY{l+m+mf}{12.05769098}\PY{p}{,}  \PY{l+m+mf}{12.74614105}\PY{p}{,}  \PY{l+m+mf}{13.43459113}\PY{p}{,}  \PY{l+m+mf}{14.1230412}\PY{p}{,}   \PY{l+m+mf}{14.81149128}\PY{p}{,} \PY{l+m+mf}{15.49994135}\PY{p}{,}  \PY{l+m+mf}{16.18839143}\PY{p}{]}\PY{p}{,}
   \PY{p}{[}\PY{l+m+mf}{12.58373087}\PY{p}{,}  \PY{l+m+mf}{13.20054771}\PY{p}{,}  \PY{l+m+mf}{13.81736455}\PY{p}{,}  \PY{l+m+mf}{14.43418138}\PY{p}{,}  \PY{l+m+mf}{15.05099822}\PY{p}{,} \PY{l+m+mf}{15.66781506}\PY{p}{,}  \PY{l+m+mf}{16.2846319} \PY{p}{]}\PY{p}{]}\PY{p}{)}
\PY{n}{scores\PYZus{}diff} \PY{o}{=} \PY{n}{np}\PY{o}{.}\PY{n}{abs}\PY{p}{(}\PY{n}{scores} \PY{o}{\PYZhy{}} \PY{n}{correct\PYZus{}scores}\PY{p}{)}\PY{o}{.}\PY{n}{sum}\PY{p}{(}\PY{p}{)}
\PY{k}{assert} \PY{n}{scores\PYZus{}diff} \PY{o}{\PYZlt{}} \PY{l+m+mf}{1e\PYZhy{}6}\PY{p}{,} \PY{l+s+s1}{\PYZsq{}}\PY{l+s+s1}{Problem with test\PYZhy{}time forward pass}\PY{l+s+s1}{\PYZsq{}}

\PY{n+nb}{print}\PY{p}{(}\PY{l+s+s1}{\PYZsq{}}\PY{l+s+s1}{Testing training loss (no regularization)}\PY{l+s+s1}{\PYZsq{}}\PY{p}{)}
\PY{n}{y} \PY{o}{=} \PY{n}{np}\PY{o}{.}\PY{n}{asarray}\PY{p}{(}\PY{p}{[}\PY{l+m+mi}{0}\PY{p}{,} \PY{l+m+mi}{5}\PY{p}{,} \PY{l+m+mi}{1}\PY{p}{]}\PY{p}{)}
\PY{n}{loss}\PY{p}{,} \PY{n}{grads} \PY{o}{=} \PY{n}{model}\PY{o}{.}\PY{n}{loss}\PY{p}{(}\PY{n}{X}\PY{p}{,} \PY{n}{y}\PY{p}{)}
\PY{n}{correct\PYZus{}loss} \PY{o}{=} \PY{l+m+mf}{3.4702243556}
\PY{k}{assert} \PY{n+nb}{abs}\PY{p}{(}\PY{n}{loss} \PY{o}{\PYZhy{}} \PY{n}{correct\PYZus{}loss}\PY{p}{)} \PY{o}{\PYZlt{}} \PY{l+m+mf}{1e\PYZhy{}10}\PY{p}{,} \PY{l+s+s1}{\PYZsq{}}\PY{l+s+s1}{Problem with training\PYZhy{}time loss}\PY{l+s+s1}{\PYZsq{}}

\PY{n}{model}\PY{o}{.}\PY{n}{reg} \PY{o}{=} \PY{l+m+mf}{1.0}
\PY{n}{loss}\PY{p}{,} \PY{n}{grads} \PY{o}{=} \PY{n}{model}\PY{o}{.}\PY{n}{loss}\PY{p}{(}\PY{n}{X}\PY{p}{,} \PY{n}{y}\PY{p}{)}
\PY{n}{correct\PYZus{}loss} \PY{o}{=} \PY{l+m+mf}{26.5948426952}
\PY{k}{assert} \PY{n+nb}{abs}\PY{p}{(}\PY{n}{loss} \PY{o}{\PYZhy{}} \PY{n}{correct\PYZus{}loss}\PY{p}{)} \PY{o}{\PYZlt{}} \PY{l+m+mf}{1e\PYZhy{}10}\PY{p}{,} \PY{l+s+s1}{\PYZsq{}}\PY{l+s+s1}{Problem with regularization loss}\PY{l+s+s1}{\PYZsq{}}

\PY{c+c1}{\PYZsh{} Errors should be around e\PYZhy{}7 or less}
\PY{k}{for} \PY{n}{reg} \PY{o+ow}{in} \PY{p}{[}\PY{l+m+mf}{0.0}\PY{p}{,} \PY{l+m+mf}{0.7}\PY{p}{]}\PY{p}{:}
  \PY{n+nb}{print}\PY{p}{(}\PY{l+s+s1}{\PYZsq{}}\PY{l+s+s1}{Running numeric gradient check with reg = }\PY{l+s+s1}{\PYZsq{}}\PY{p}{,} \PY{n}{reg}\PY{p}{)}
  \PY{n}{model}\PY{o}{.}\PY{n}{reg} \PY{o}{=} \PY{n}{reg}
  \PY{n}{loss}\PY{p}{,} \PY{n}{grads} \PY{o}{=} \PY{n}{model}\PY{o}{.}\PY{n}{loss}\PY{p}{(}\PY{n}{X}\PY{p}{,} \PY{n}{y}\PY{p}{)}

  \PY{k}{for} \PY{n}{name} \PY{o+ow}{in} \PY{n+nb}{sorted}\PY{p}{(}\PY{n}{grads}\PY{p}{)}\PY{p}{:}
    \PY{n}{f} \PY{o}{=} \PY{k}{lambda} \PY{n}{\PYZus{}}\PY{p}{:} \PY{n}{model}\PY{o}{.}\PY{n}{loss}\PY{p}{(}\PY{n}{X}\PY{p}{,} \PY{n}{y}\PY{p}{)}\PY{p}{[}\PY{l+m+mi}{0}\PY{p}{]}
    \PY{n}{grad\PYZus{}num} \PY{o}{=} \PY{n}{eval\PYZus{}numerical\PYZus{}gradient}\PY{p}{(}\PY{n}{f}\PY{p}{,} \PY{n}{model}\PY{o}{.}\PY{n}{params}\PY{p}{[}\PY{n}{name}\PY{p}{]}\PY{p}{,} \PY{n}{verbose}\PY{o}{=}\PY{k+kc}{False}\PY{p}{)}
    \PY{n+nb}{print}\PY{p}{(}\PY{l+s+s1}{\PYZsq{}}\PY{l+s+si}{\PYZpc{}s}\PY{l+s+s1}{ relative error: }\PY{l+s+si}{\PYZpc{}.2e}\PY{l+s+s1}{\PYZsq{}} \PY{o}{\PYZpc{}} \PY{p}{(}\PY{n}{name}\PY{p}{,} \PY{n}{rel\PYZus{}error}\PY{p}{(}\PY{n}{grad\PYZus{}num}\PY{p}{,} \PY{n}{grads}\PY{p}{[}\PY{n}{name}\PY{p}{]}\PY{p}{)}\PY{p}{)}\PY{p}{)}
\end{Verbatim}
\end{tcolorbox}

    \begin{Verbatim}[commandchars=\\\{\}]
Testing initialization {\ldots}
Testing test-time forward pass {\ldots}
Testing training loss (no regularization)
Running numeric gradient check with reg =  0.0
W1 relative error: 3.97e-08
W2 relative error: 8.97e-10
b1 relative error: 2.26e-08
b2 relative error: 9.39e-10
Running numeric gradient check with reg =  0.7
W1 relative error: 3.12e-07
W2 relative error: 7.98e-08
b1 relative error: 2.80e-08
b2 relative error: 1.97e-09
    \end{Verbatim}

    \hypertarget{solver} accuracy on the validation set.

    \begin{tcolorbox}[breakable, size=fbox, boxrule=1pt, pad at break*=1mm,colback=cellbackground, colframe=cellborder]
\prompt{In}{incolor}{ }{\boxspacing}
\begin{Verbatim}[commandchars=\\\{\}]
\PY{n}{input\PYZus{}size} \PY{o}{=} \PY{l+m+mi}{32} \PY{o}{*} \PY{l+m+mi}{32} \PY{o}{*} \PY{l+m+mi}{3}
\PY{n}{hidden\PYZus{}size} \PY{o}{=} \PY{l+m+mi}{50}
\PY{n}{num\PYZus{}classes} \PY{o}{=} \PY{l+m+mi}{10}
\PY{n}{model} \PY{o}{=} \PY{n}{TwoLayerNet}\PY{p}{(}\PY{n}{input\PYZus{}size}\PY{p}{,} \PY{n}{hidden\PYZus{}size}\PY{p}{,} \PY{n}{num\PYZus{}classes}\PY{p}{)}
\PY{n}{solver} \PY{o}{=} \PY{k+kc}{None}

\PY{c+c1}{\PYZsh{}\PYZsh{}\PYZsh{}\PYZsh{}\PYZsh{}\PYZsh{}\PYZsh{}\PYZsh{}\PYZsh{}\PYZsh{}\PYZsh{}\PYZsh{}\PYZsh{}\PYZsh{}\PYZsh{}\PYZsh{}\PYZsh{}\PYZsh{}\PYZsh{}\PYZsh{}\PYZsh{}\PYZsh{}\PYZsh{}\PYZsh{}\PYZsh{}\PYZsh{}\PYZsh{}\PYZsh{}\PYZsh{}\PYZsh{}\PYZsh{}\PYZsh{}\PYZsh{}\PYZsh{}\PYZsh{}\PYZsh{}\PYZsh{}\PYZsh{}\PYZsh{}\PYZsh{}\PYZsh{}\PYZsh{}\PYZsh{}\PYZsh{}\PYZsh{}\PYZsh{}\PYZsh{}\PYZsh{}\PYZsh{}\PYZsh{}\PYZsh{}\PYZsh{}\PYZsh{}\PYZsh{}\PYZsh{}\PYZsh{}\PYZsh{}\PYZsh{}\PYZsh{}\PYZsh{}\PYZsh{}\PYZsh{}\PYZsh{}\PYZsh{}\PYZsh{}\PYZsh{}\PYZsh{}\PYZsh{}\PYZsh{}\PYZsh{}\PYZsh{}\PYZsh{}\PYZsh{}\PYZsh{}\PYZsh{}\PYZsh{}\PYZsh{}\PYZsh{}}
\PY{c+c1}{\PYZsh{} TODO: Use a Solver instance to train a TwoLayerNet that achieves about 36\PYZpc{} \PYZsh{}}
\PY{c+c1}{\PYZsh{} accuracy on the validation set.                                            \PYZsh{}}
\PY{c+c1}{\PYZsh{}\PYZsh{}\PYZsh{}\PYZsh{}\PYZsh{}\PYZsh{}\PYZsh{}\PYZsh{}\PYZsh{}\PYZsh{}\PYZsh{}\PYZsh{}\PYZsh{}\PYZsh{}\PYZsh{}\PYZsh{}\PYZsh{}\PYZsh{}\PYZsh{}\PYZsh{}\PYZsh{}\PYZsh{}\PYZsh{}\PYZsh{}\PYZsh{}\PYZsh{}\PYZsh{}\PYZsh{}\PYZsh{}\PYZsh{}\PYZsh{}\PYZsh{}\PYZsh{}\PYZsh{}\PYZsh{}\PYZsh{}\PYZsh{}\PYZsh{}\PYZsh{}\PYZsh{}\PYZsh{}\PYZsh{}\PYZsh{}\PYZsh{}\PYZsh{}\PYZsh{}\PYZsh{}\PYZsh{}\PYZsh{}\PYZsh{}\PYZsh{}\PYZsh{}\PYZsh{}\PYZsh{}\PYZsh{}\PYZsh{}\PYZsh{}\PYZsh{}\PYZsh{}\PYZsh{}\PYZsh{}\PYZsh{}\PYZsh{}\PYZsh{}\PYZsh{}\PYZsh{}\PYZsh{}\PYZsh{}\PYZsh{}\PYZsh{}\PYZsh{}\PYZsh{}\PYZsh{}\PYZsh{}\PYZsh{}\PYZsh{}\PYZsh{}\PYZsh{}}
\PY{c+c1}{\PYZsh{} *****START OF YOUR CODE (DO NOT DELETE/MODIFY THIS LINE)*****}

\PY{n}{solver} \PY{o}{=} \PY{n}{Solver}\PY{p}{(}\PY{n}{model}\PY{p}{,} \PY{n}{data}\PY{p}{,}
          \PY{n}{update\PYZus{}rule}\PY{o}{=}\PY{l+s+s1}{\PYZsq{}}\PY{l+s+s1}{sgd}\PY{l+s+s1}{\PYZsq{}}\PY{p}{,}
          \PY{n}{optim\PYZus{}config}\PY{o}{=}\PY{p}{\PYZob{}}\PY{l+s+s1}{\PYZsq{}}\PY{l+s+s1}{learning\PYZus{}rate}\PY{l+s+s1}{\PYZsq{}}\PY{p}{:} \PY{l+m+mf}{1e\PYZhy{}4}\PY{p}{\PYZcb{}}\PY{p}{,}
          \PY{n}{lr\PYZus{}decay}\PY{o}{=}\PY{l+m+mf}{0.95}\PY{p}{,} \PY{n}{num\PYZus{}epochs}\PY{o}{=}\PY{l+m+mi}{5}\PY{p}{,}
          \PY{n}{batch\PYZus{}size}\PY{o}{=}\PY{l+m+mi}{200}\PY{p}{,} \PY{n}{print\PYZus{}every}\PY{o}{=}\PY{l+m+mi}{100}\PY{p}{)}
\PY{n}{solver}\PY{o}{.}\PY{n}{train}\PY{p}{(}\PY{p}{)}

\PY{c+c1}{\PYZsh{} *****END OF YOUR CODE (DO NOT DELETE/MODIFY THIS LINE)*****}
\PY{c+c1}{\PYZsh{}\PYZsh{}\PYZsh{}\PYZsh{}\PYZsh{}\PYZsh{}\PYZsh{}\PYZsh{}\PYZsh{}\PYZsh{}\PYZsh{}\PYZsh{}\PYZsh{}\PYZsh{}\PYZsh{}\PYZsh{}\PYZsh{}\PYZsh{}\PYZsh{}\PYZsh{}\PYZsh{}\PYZsh{}\PYZsh{}\PYZsh{}\PYZsh{}\PYZsh{}\PYZsh{}\PYZsh{}\PYZsh{}\PYZsh{}\PYZsh{}\PYZsh{}\PYZsh{}\PYZsh{}\PYZsh{}\PYZsh{}\PYZsh{}\PYZsh{}\PYZsh{}\PYZsh{}\PYZsh{}\PYZsh{}\PYZsh{}\PYZsh{}\PYZsh{}\PYZsh{}\PYZsh{}\PYZsh{}\PYZsh{}\PYZsh{}\PYZsh{}\PYZsh{}\PYZsh{}\PYZsh{}\PYZsh{}\PYZsh{}\PYZsh{}\PYZsh{}\PYZsh{}\PYZsh{}\PYZsh{}\PYZsh{}\PYZsh{}\PYZsh{}\PYZsh{}\PYZsh{}\PYZsh{}\PYZsh{}\PYZsh{}\PYZsh{}\PYZsh{}\PYZsh{}\PYZsh{}\PYZsh{}\PYZsh{}\PYZsh{}\PYZsh{}\PYZsh{}}
\PY{c+c1}{\PYZsh{}                             END OF YOUR CODE                               \PYZsh{}}
\PY{c+c1}{\PYZsh{}\PYZsh{}\PYZsh{}\PYZsh{}\PYZsh{}\PYZsh{}\PYZsh{}\PYZsh{}\PYZsh{}\PYZsh{}\PYZsh{}\PYZsh{}\PYZsh{}\PYZsh{}\PYZsh{}\PYZsh{}\PYZsh{}\PYZsh{}\PYZsh{}\PYZsh{}\PYZsh{}\PYZsh{}\PYZsh{}\PYZsh{}\PYZsh{}\PYZsh{}\PYZsh{}\PYZsh{}\PYZsh{}\PYZsh{}\PYZsh{}\PYZsh{}\PYZsh{}\PYZsh{}\PYZsh{}\PYZsh{}\PYZsh{}\PYZsh{}\PYZsh{}\PYZsh{}\PYZsh{}\PYZsh{}\PYZsh{}\PYZsh{}\PYZsh{}\PYZsh{}\PYZsh{}\PYZsh{}\PYZsh{}\PYZsh{}\PYZsh{}\PYZsh{}\PYZsh{}\PYZsh{}\PYZsh{}\PYZsh{}\PYZsh{}\PYZsh{}\PYZsh{}\PYZsh{}\PYZsh{}\PYZsh{}\PYZsh{}\PYZsh{}\PYZsh{}\PYZsh{}\PYZsh{}\PYZsh{}\PYZsh{}\PYZsh{}\PYZsh{}\PYZsh{}\PYZsh{}\PYZsh{}\PYZsh{}\PYZsh{}\PYZsh{}\PYZsh{}}
\end{Verbatim}
\end{tcolorbox}

    \begin{Verbatim}[commandchars=\\\{\}]
(Iteration 1 / 1225) loss: 2.303854
(Epoch 0 / 5) train acc: 0.118000; val\_acc: 0.128000
(Iteration 101 / 1225) loss: 2.270676
(Iteration 201 / 1225) loss: 2.155035
(Epoch 1 / 5) train acc: 0.218000; val\_acc: 0.253000
(Iteration 301 / 1225) loss: 2.049379
(Iteration 401 / 1225) loss: 2.028418
(Epoch 2 / 5) train acc: 0.286000; val\_acc: 0.279000
(Iteration 501 / 1225) loss: 1.921624
(Iteration 601 / 1225) loss: 1.939255
(Iteration 701 / 1225) loss: 1.832826
(Epoch 3 / 5) train acc: 0.354000; val\_acc: 0.315000
(Iteration 801 / 1225) loss: 1.887060
(Iteration 901 / 1225) loss: 1.768584
(Epoch 4 / 5) train acc: 0.355000; val\_acc: 0.346000
(Iteration 1001 / 1225) loss: 1.804191
(Iteration 1101 / 1225) loss: 1.725366
(Iteration 1201 / 1225) loss: 1.833903
(Epoch 5 / 5) train acc: 0.364000; val\_acc: 0.371000
    \end{Verbatim}

    \hypertarget{debug-the-training}{%
\section{Debug the training}\label{debug-the-training}}

With the default parameters we provided above, you should get a
validation accuracy of about 0.36 on the validation set. This isn't very
good.

One strategy for getting insight into what's wrong is to plot the loss
function and the accuracies on the training and validation sets during
optimization.

Another strategy is to visualize the weights that were learned in the
first layer of the network. In most neural networks trained on visual
data, the first layer weights typically show some visible structure when
visualized.

    \begin{tcolorbox}[breakable, size=fbox, boxrule=1pt, pad at break*=1mm,colback=cellbackground, colframe=cellborder]
\prompt{In}{incolor}{ }{\boxspacing}
\begin{Verbatim}[commandchars=\\\{\}]
\PY{c+c1}{\PYZsh{} Run this cell to visualize training loss and train / val accuracy}

\PY{n}{plt}\PY{o}{.}\PY{n}{subplot}\PY{p}{(}\PY{l+m+mi}{2}\PY{p}{,} \PY{l+m+mi}{1}\PY{p}{,} \PY{l+m+mi}{1}\PY{p}{)}
\PY{n}{plt}\PY{o}{.}\PY{n}{title}\PY{p}{(}\PY{l+s+s1}{\PYZsq{}}\PY{l+s+s1}{Training loss}\PY{l+s+s1}{\PYZsq{}}\PY{p}{)}
\PY{n}{plt}\PY{o}{.}\PY{n}{plot}\PY{p}{(}\PY{n}{solver}\PY{o}{.}\PY{n}{loss\PYZus{}history}\PY{p}{,} \PY{l+s+s1}{\PYZsq{}}\PY{l+s+s1}{o}\PY{l+s+s1}{\PYZsq{}}\PY{p}{)}
\PY{n}{plt}\PY{o}{.}\PY{n}{xlabel}\PY{p}{(}\PY{l+s+s1}{\PYZsq{}}\PY{l+s+s1}{Iteration}\PY{l+s+s1}{\PYZsq{}}\PY{p}{)}

\PY{n}{plt}\PY{o}{.}\PY{n}{subplot}\PY{p}{(}\PY{l+m+mi}{2}\PY{p}{,} \PY{l+m+mi}{1}\PY{p}{,} \PY{l+m+mi}{2}\PY{p}{)}
\PY{n}{plt}\PY{o}{.}\PY{n}{title}\PY{p}{(}\PY{l+s+s1}{\PYZsq{}}\PY{l+s+s1}{Accuracy}\PY{l+s+s1}{\PYZsq{}}\PY{p}{)}
\PY{n}{plt}\PY{o}{.}\PY{n}{plot}\PY{p}{(}\PY{n}{solver}\PY{o}{.}\PY{n}{train\PYZus{}acc\PYZus{}history}\PY{p}{,} \PY{l+s+s1}{\PYZsq{}}\PY{l+s+s1}{\PYZhy{}o}\PY{l+s+s1}{\PYZsq{}}\PY{p}{,} \PY{n}{label}\PY{o}{=}\PY{l+s+s1}{\PYZsq{}}\PY{l+s+s1}{train}\PY{l+s+s1}{\PYZsq{}}\PY{p}{)}
\PY{n}{plt}\PY{o}{.}\PY{n}{plot}\PY{p}{(}\PY{n}{solver}\PY{o}{.}\PY{n}{val\PYZus{}acc\PYZus{}history}\PY{p}{,} \PY{l+s+s1}{\PYZsq{}}\PY{l+s+s1}{\PYZhy{}o}\PY{l+s+s1}{\PYZsq{}}\PY{p}{,} \PY{n}{label}\PY{o}{=}\PY{l+s+s1}{\PYZsq{}}\PY{l+s+s1}{val}\PY{l+s+s1}{\PYZsq{}}\PY{p}{)}
\PY{n}{plt}\PY{o}{.}\PY{n}{plot}\PY{p}{(}\PY{p}{[}\PY{l+m+mf}{0.5}\PY{p}{]} \PY{o}{*} \PY{n+nb}{len}\PY{p}{(}\PY{n}{solver}\PY{o}{.}\PY{n}{val\PYZus{}acc\PYZus{}history}\PY{p}{)}\PY{p}{,} \PY{l+s+s1}{\PYZsq{}}\PY{l+s+s1}{k\PYZhy{}\PYZhy{}}\PY{l+s+s1}{\PYZsq{}}\PY{p}{)}
\PY{n}{plt}\PY{o}{.}\PY{n}{xlabel}\PY{p}{(}\PY{l+s+s1}{\PYZsq{}}\PY{l+s+s1}{Epoch}\PY{l+s+s1}{\PYZsq{}}\PY{p}{)}
\PY{n}{plt}\PY{o}{.}\PY{n}{legend}\PY{p}{(}\PY{n}{loc}\PY{o}{=}\PY{l+s+s1}{\PYZsq{}}\PY{l+s+s1}{lower right}\PY{l+s+s1}{\PYZsq{}}\PY{p}{)}
\PY{n}{plt}\PY{o}{.}\PY{n}{gcf}\PY{p}{(}\PY{p}{)}\PY{o}{.}\PY{n}{set\PYZus{}size\PYZus{}inches}\PY{p}{(}\PY{l+m+mi}{15}\PY{p}{,} \PY{l+m+mi}{12}\PY{p}{)}
\PY{n}{plt}\PY{o}{.}\PY{n}{show}\PY{p}{(}\PY{p}{)}
\end{Verbatim}
\end{tcolorbox}

    \begin{center}
    \adjustimage{max size={0.9\linewidth}{0.9\paperheight}}{two_layer_net_files/two_layer_net_22_0.png}
    \end{center}
    { \hspace*{\fill} \\}
    
    \begin{tcolorbox}[breakable, size=fbox, boxrule=1pt, pad at break*=1mm,colback=cellbackground, colframe=cellborder]
\prompt{In}{incolor}{ }{\boxspacing}
\begin{Verbatim}[commandchars=\\\{\}]
\PY{k+kn}{from} \PY{n+nn}{cs231n}\PY{n+nn}{.}\PY{n+nn}{vis\PYZus{}utils} \PY{k+kn}{import} \PY{n}{visualize\PYZus{}grid}

\PY{c+c1}{\PYZsh{} Visualize the weights of the network}

\PY{k}{def} \PY{n+nf}{show\PYZus{}net\PYZus{}weights}\PY{p}{(}\PY{n}{net}\PY{p}{)}\PY{p}{:}
    \PY{n}{W1} \PY{o}{=} \PY{n}{net}\PY{o}{.}\PY{n}{params}\PY{p}{[}\PY{l+s+s1}{\PYZsq{}}\PY{l+s+s1}{W1}\PY{l+s+s1}{\PYZsq{}}\PY{p}{]}
    \PY{n}{W1} \PY{o}{=} \PY{n}{W1}\PY{o}{.}\PY{n}{reshape}\PY{p}{(}\PY{l+m+mi}{3}\PY{p}{,} \PY{l+m+mi}{32}\PY{p}{,} \PY{l+m+mi}{32}\PY{p}{,} \PY{o}{\PYZhy{}}\PY{l+m+mi}{1}\PY{p}{)}\PY{o}{.}\PY{n}{transpose}\PY{p}{(}\PY{l+m+mi}{3}\PY{p}{,} \PY{l+m+mi}{1}\PY{p}{,} \PY{l+m+mi}{2}\PY{p}{,} \PY{l+m+mi}{0}\PY{p}{)}
    \PY{n}{plt}\PY{o}{.}\PY{n}{imshow}\PY{p}{(}\PY{n}{visualize\PYZus{}grid}\PY{p}{(}\PY{n}{W1}\PY{p}{,} \PY{n}{padding}\PY{o}{=}\PY{l+m+mi}{3}\PY{p}{)}\PY{o}{.}\PY{n}{astype}\PY{p}{(}\PY{l+s+s1}{\PYZsq{}}\PY{l+s+s1}{uint8}\PY{l+s+s1}{\PYZsq{}}\PY{p}{)}\PY{p}{)}
    \PY{n}{plt}\PY{o}{.}\PY{n}{gca}\PY{p}{(}\PY{p}{)}\PY{o}{.}\PY{n}{axis}\PY{p}{(}\PY{l+s+s1}{\PYZsq{}}\PY{l+s+s1}{off}\PY{l+s+s1}{\PYZsq{}}\PY{p}{)}
    \PY{n}{plt}\PY{o}{.}\PY{n}{show}\PY{p}{(}\PY{p}{)}

\PY{n}{show\PYZus{}net\PYZus{}weights}\PY{p}{(}\PY{n}{model}\PY{p}{)}
\end{Verbatim}
\end{tcolorbox}

    \begin{center}
    \adjustimage{max size={0.9\linewidth}{0.9\paperheight}}{two_layer_net_files/two_layer_net_23_0.png}
    \end{center}
    { \hspace*{\fill} \\}
    
    \hypertarget{tune-your-hyperparameters}{%
\section{Tune your hyperparameters}\label{tune-your-hyperparameters}}

\textbf{What's wrong?}. Looking at the visualizations above, we see that
the loss is decreasing more or less linearly, which seems to suggest
that the learning rate may be too low. Moreover, there is no gap between
the training and validation accuracy, suggesting that the model we used
has low capacity, and that we should increase its size. On the other
hand, with a very large model we would expect to see more overfitting,
which would manifest itself as a very large gap between the training and
validation accuracy.

\textbf{Tuning}. Tuning the hyperparameters and developing intuition for
how they affect the final performance is a large part of using Neural
Networks, so we want you to get a lot of practice. Below, you should
experiment with different values of the various hyperparameters,
including hidden layer size, learning rate, numer of training epochs,
and regularization strength. You might also consider tuning the learning
rate decay, but you should be able to get good performance using the
default value.

\textbf{Approximate results}. You should be aim to achieve a
classification accuracy of greater than 48\% on the validation set. Our
best network gets over 52\% on the validation set.

\textbf{Experiment}: You goal in this exercise is to get as good of a
result on CIFAR-10 as you can (52\% could serve as a reference), with a
fully-connected Neural Network. Feel free implement your own techniques
(e.g.~PCA to reduce dimensionality, or adding dropout, or adding
features to the solver, etc.).

    \begin{tcolorbox}[breakable, size=fbox, boxrule=1pt, pad at break*=1mm,colback=cellbackground, colframe=cellborder]
\prompt{In}{incolor}{ }{\boxspacing}
\begin{Verbatim}[commandchars=\\\{\}]
\PY{n}{best\PYZus{}model} \PY{o}{=} \PY{k+kc}{None}

\PY{c+c1}{\PYZsh{}\PYZsh{}\PYZsh{}\PYZsh{}\PYZsh{}\PYZsh{}\PYZsh{}\PYZsh{}\PYZsh{}\PYZsh{}\PYZsh{}\PYZsh{}\PYZsh{}\PYZsh{}\PYZsh{}\PYZsh{}\PYZsh{}\PYZsh{}\PYZsh{}\PYZsh{}\PYZsh{}\PYZsh{}\PYZsh{}\PYZsh{}\PYZsh{}\PYZsh{}\PYZsh{}\PYZsh{}\PYZsh{}\PYZsh{}\PYZsh{}\PYZsh{}\PYZsh{}\PYZsh{}\PYZsh{}\PYZsh{}\PYZsh{}\PYZsh{}\PYZsh{}\PYZsh{}\PYZsh{}\PYZsh{}\PYZsh{}\PYZsh{}\PYZsh{}\PYZsh{}\PYZsh{}\PYZsh{}\PYZsh{}\PYZsh{}\PYZsh{}\PYZsh{}\PYZsh{}\PYZsh{}\PYZsh{}\PYZsh{}\PYZsh{}\PYZsh{}\PYZsh{}\PYZsh{}\PYZsh{}\PYZsh{}\PYZsh{}\PYZsh{}\PYZsh{}\PYZsh{}\PYZsh{}\PYZsh{}\PYZsh{}\PYZsh{}\PYZsh{}\PYZsh{}\PYZsh{}\PYZsh{}\PYZsh{}\PYZsh{}\PYZsh{}\PYZsh{}\PYZsh{}\PYZsh{}\PYZsh{}}
\PY{c+c1}{\PYZsh{} TODO: Tune hyperparameters using the validation set. Store your best trained  \PYZsh{}}
\PY{c+c1}{\PYZsh{} model in best\PYZus{}model.                                                          \PYZsh{}}
\PY{c+c1}{\PYZsh{}                                                                               \PYZsh{}}
\PY{c+c1}{\PYZsh{} To help debug your network, it may help to use visualizations similar to the  \PYZsh{}}
\PY{c+c1}{\PYZsh{} ones we used above; these visualizations will have significant qualitative    \PYZsh{}}
\PY{c+c1}{\PYZsh{} differences from the ones we saw above for the poorly tuned network.          \PYZsh{}}
\PY{c+c1}{\PYZsh{}                                                                               \PYZsh{}}
\PY{c+c1}{\PYZsh{} Tweaking hyperparameters by hand can be fun, but you might find it useful to  \PYZsh{}}
\PY{c+c1}{\PYZsh{} write code to sweep through possible combinations of hyperparameters          \PYZsh{}}
\PY{c+c1}{\PYZsh{} automatically like we did on thexs previous exercises.                          \PYZsh{}}
\PY{c+c1}{\PYZsh{}\PYZsh{}\PYZsh{}\PYZsh{}\PYZsh{}\PYZsh{}\PYZsh{}\PYZsh{}\PYZsh{}\PYZsh{}\PYZsh{}\PYZsh{}\PYZsh{}\PYZsh{}\PYZsh{}\PYZsh{}\PYZsh{}\PYZsh{}\PYZsh{}\PYZsh{}\PYZsh{}\PYZsh{}\PYZsh{}\PYZsh{}\PYZsh{}\PYZsh{}\PYZsh{}\PYZsh{}\PYZsh{}\PYZsh{}\PYZsh{}\PYZsh{}\PYZsh{}\PYZsh{}\PYZsh{}\PYZsh{}\PYZsh{}\PYZsh{}\PYZsh{}\PYZsh{}\PYZsh{}\PYZsh{}\PYZsh{}\PYZsh{}\PYZsh{}\PYZsh{}\PYZsh{}\PYZsh{}\PYZsh{}\PYZsh{}\PYZsh{}\PYZsh{}\PYZsh{}\PYZsh{}\PYZsh{}\PYZsh{}\PYZsh{}\PYZsh{}\PYZsh{}\PYZsh{}\PYZsh{}\PYZsh{}\PYZsh{}\PYZsh{}\PYZsh{}\PYZsh{}\PYZsh{}\PYZsh{}\PYZsh{}\PYZsh{}\PYZsh{}\PYZsh{}\PYZsh{}\PYZsh{}\PYZsh{}\PYZsh{}\PYZsh{}\PYZsh{}\PYZsh{}\PYZsh{}\PYZsh{}}
\PY{c+c1}{\PYZsh{} *****START OF YOUR CODE (DO NOT DELETE/MODIFY THIS LINE)*****}

\PY{n}{best\PYZus{}acc} \PY{o}{=} \PY{o}{\PYZhy{}}\PY{l+m+mi}{1}
\PY{n}{model} \PY{o}{=} \PY{k+kc}{None}
\PY{n}{solver} \PY{o}{=} \PY{k+kc}{None}

\PY{n}{input\PYZus{}size} \PY{o}{=} \PY{l+m+mi}{32} \PY{o}{*} \PY{l+m+mi}{32} \PY{o}{*} \PY{l+m+mi}{3}
\PY{n}{hidden\PYZus{}size} \PY{o}{=} \PY{l+m+mi}{50}
\PY{n}{num\PYZus{}classes} \PY{o}{=} \PY{l+m+mi}{10}

\PY{n}{num\PYZus{}epoches} \PY{o}{=} \PY{l+m+mi}{25}

\PY{n}{model} \PY{o}{=} \PY{n}{TwoLayerNet}\PY{p}{(}\PY{n}{input\PYZus{}size}\PY{p}{,} \PY{n}{hidden\PYZus{}size}\PY{p}{,} \PY{n}{num\PYZus{}classes}\PY{p}{)}

\PY{n}{solver} \PY{o}{=} \PY{n}{Solver}\PY{p}{(}\PY{n}{model}\PY{p}{,} \PY{n}{data}\PY{p}{,}
          \PY{n}{update\PYZus{}rule}\PY{o}{=}\PY{l+s+s1}{\PYZsq{}}\PY{l+s+s1}{sgd}\PY{l+s+s1}{\PYZsq{}}\PY{p}{,}
          \PY{n}{optim\PYZus{}config}\PY{o}{=}\PY{p}{\PYZob{}}\PY{l+s+s1}{\PYZsq{}}\PY{l+s+s1}{learning\PYZus{}rate}\PY{l+s+s1}{\PYZsq{}}\PY{p}{:} \PY{l+m+mf}{5e\PYZhy{}4}\PY{p}{\PYZcb{}}\PY{p}{,}
          \PY{n}{lr\PYZus{}decay}\PY{o}{=}\PY{l+m+mf}{0.95}\PY{p}{,} \PY{n}{num\PYZus{}epochs}\PY{o}{=}\PY{l+m+mi}{30}\PY{p}{,}
          \PY{n}{batch\PYZus{}size}\PY{o}{=}\PY{l+m+mi}{200}\PY{p}{,} \PY{n}{print\PYZus{}every}\PY{o}{=}\PY{l+m+mi}{100}\PY{p}{)}
\PY{n}{solver}\PY{o}{.}\PY{n}{train}\PY{p}{(}\PY{p}{)}

\PY{c+c1}{\PYZsh{} y\PYZus{}val\PYZus{}pred = np.argmax(model.loss(data[\PYZsq{}X\PYZus{}val\PYZsq{}]), axis=1)}
\PY{c+c1}{\PYZsh{} accuracy = (y\PYZus{}val\PYZus{}pred == data[\PYZsq{}y\PYZus{}val\PYZsq{}]).mean()}
\PY{n}{accuracy} \PY{o}{=} \PY{n}{solver}\PY{o}{.}\PY{n}{best\PYZus{}val\PYZus{}acc}
\PY{k}{if} \PY{n}{accuracy} \PY{o}{\PYZgt{}} \PY{n}{best\PYZus{}acc}\PY{p}{:}
  \PY{n}{best\PYZus{}model} \PY{o}{=} \PY{n}{model}
  \PY{n}{best\PYZus{}acc} \PY{o}{=} \PY{n}{accuracy}
  \PY{n+nb}{print}\PY{p}{(}\PY{l+s+s1}{\PYZsq{}}\PY{l+s+s1}{new best accuracy: }\PY{l+s+s1}{\PYZsq{}}\PY{p}{,} \PY{n}{accuracy}\PY{p}{)}

\PY{c+c1}{\PYZsh{} *****END OF YOUR CODE (DO NOT DELETE/MODIFY THIS LINE)*****}
\PY{c+c1}{\PYZsh{}\PYZsh{}\PYZsh{}\PYZsh{}\PYZsh{}\PYZsh{}\PYZsh{}\PYZsh{}\PYZsh{}\PYZsh{}\PYZsh{}\PYZsh{}\PYZsh{}\PYZsh{}\PYZsh{}\PYZsh{}\PYZsh{}\PYZsh{}\PYZsh{}\PYZsh{}\PYZsh{}\PYZsh{}\PYZsh{}\PYZsh{}\PYZsh{}\PYZsh{}\PYZsh{}\PYZsh{}\PYZsh{}\PYZsh{}\PYZsh{}\PYZsh{}\PYZsh{}\PYZsh{}\PYZsh{}\PYZsh{}\PYZsh{}\PYZsh{}\PYZsh{}\PYZsh{}\PYZsh{}\PYZsh{}\PYZsh{}\PYZsh{}\PYZsh{}\PYZsh{}\PYZsh{}\PYZsh{}\PYZsh{}\PYZsh{}\PYZsh{}\PYZsh{}\PYZsh{}\PYZsh{}\PYZsh{}\PYZsh{}\PYZsh{}\PYZsh{}\PYZsh{}\PYZsh{}\PYZsh{}\PYZsh{}\PYZsh{}\PYZsh{}\PYZsh{}\PYZsh{}\PYZsh{}\PYZsh{}\PYZsh{}\PYZsh{}\PYZsh{}\PYZsh{}\PYZsh{}\PYZsh{}\PYZsh{}\PYZsh{}\PYZsh{}\PYZsh{}\PYZsh{}\PYZsh{}}
\PY{c+c1}{\PYZsh{}                              END OF YOUR CODE                                \PYZsh{}}
\PY{c+c1}{\PYZsh{}\PYZsh{}\PYZsh{}\PYZsh{}\PYZsh{}\PYZsh{}\PYZsh{}\PYZsh{}\PYZsh{}\PYZsh{}\PYZsh{}\PYZsh{}\PYZsh{}\PYZsh{}\PYZsh{}\PYZsh{}\PYZsh{}\PYZsh{}\PYZsh{}\PYZsh{}\PYZsh{}\PYZsh{}\PYZsh{}\PYZsh{}\PYZsh{}\PYZsh{}\PYZsh{}\PYZsh{}\PYZsh{}\PYZsh{}\PYZsh{}\PYZsh{}\PYZsh{}\PYZsh{}\PYZsh{}\PYZsh{}\PYZsh{}\PYZsh{}\PYZsh{}\PYZsh{}\PYZsh{}\PYZsh{}\PYZsh{}\PYZsh{}\PYZsh{}\PYZsh{}\PYZsh{}\PYZsh{}\PYZsh{}\PYZsh{}\PYZsh{}\PYZsh{}\PYZsh{}\PYZsh{}\PYZsh{}\PYZsh{}\PYZsh{}\PYZsh{}\PYZsh{}\PYZsh{}\PYZsh{}\PYZsh{}\PYZsh{}\PYZsh{}\PYZsh{}\PYZsh{}\PYZsh{}\PYZsh{}\PYZsh{}\PYZsh{}\PYZsh{}\PYZsh{}\PYZsh{}\PYZsh{}\PYZsh{}\PYZsh{}\PYZsh{}\PYZsh{}\PYZsh{}\PYZsh{}}
\end{Verbatim}
\end{tcolorbox}

    \begin{Verbatim}[commandchars=\\\{\}]
(Iteration 1 / 7350) loss: 2.300190
(Epoch 0 / 30) train acc: 0.145000; val\_acc: 0.165000
(Iteration 101 / 7350) loss: 2.015234
(Iteration 201 / 7350) loss: 1.790591
(Epoch 1 / 30) train acc: 0.390000; val\_acc: 0.380000
(Iteration 301 / 7350) loss: 1.758738
(Iteration 401 / 7350) loss: 1.666508
(Epoch 2 / 30) train acc: 0.431000; val\_acc: 0.438000
(Iteration 501 / 7350) loss: 1.620063
(Iteration 601 / 7350) loss: 1.648616
(Iteration 701 / 7350) loss: 1.682793
(Epoch 3 / 30) train acc: 0.480000; val\_acc: 0.454000
(Iteration 801 / 7350) loss: 1.548954
(Iteration 901 / 7350) loss: 1.393497
(Epoch 4 / 30) train acc: 0.424000; val\_acc: 0.473000
(Iteration 1001 / 7350) loss: 1.566226
(Iteration 1101 / 7350) loss: 1.517484
(Iteration 1201 / 7350) loss: 1.500529
(Epoch 5 / 30) train acc: 0.466000; val\_acc: 0.469000
(Iteration 1301 / 7350) loss: 1.296426
(Iteration 1401 / 7350) loss: 1.359932
(Epoch 6 / 30) train acc: 0.491000; val\_acc: 0.476000
(Iteration 1501 / 7350) loss: 1.484785
(Iteration 1601 / 7350) loss: 1.516682
(Iteration 1701 / 7350) loss: 1.460739
(Epoch 7 / 30) train acc: 0.481000; val\_acc: 0.489000
(Iteration 1801 / 7350) loss: 1.543901
(Iteration 1901 / 7350) loss: 1.464250
(Epoch 8 / 30) train acc: 0.497000; val\_acc: 0.485000
(Iteration 2001 / 7350) loss: 1.312434
(Iteration 2101 / 7350) loss: 1.412041
(Iteration 2201 / 7350) loss: 1.529823
(Epoch 9 / 30) train acc: 0.532000; val\_acc: 0.494000
(Iteration 2301 / 7350) loss: 1.226684
(Iteration 2401 / 7350) loss: 1.323381
(Epoch 10 / 30) train acc: 0.535000; val\_acc: 0.497000
(Iteration 2501 / 7350) loss: 1.352839
(Iteration 2601 / 7350) loss: 1.337114
(Epoch 11 / 30) train acc: 0.531000; val\_acc: 0.501000
(Iteration 2701 / 7350) loss: 1.267915
(Iteration 2801 / 7350) loss: 1.367660
(Iteration 2901 / 7350) loss: 1.356661
(Epoch 12 / 30) train acc: 0.549000; val\_acc: 0.521000
(Iteration 3001 / 7350) loss: 1.306190
(Iteration 3101 / 7350) loss: 1.378280
(Epoch 13 / 30) train acc: 0.511000; val\_acc: 0.518000
(Iteration 3201 / 7350) loss: 1.388568
(Iteration 3301 / 7350) loss: 1.323884
(Iteration 3401 / 7350) loss: 1.354564
(Epoch 14 / 30) train acc: 0.547000; val\_acc: 0.522000
(Iteration 3501 / 7350) loss: 1.233557
(Iteration 3601 / 7350) loss: 1.211154
(Epoch 15 / 30) train acc: 0.575000; val\_acc: 0.515000
(Iteration 3701 / 7350) loss: 1.273599
(Iteration 3801 / 7350) loss: 1.163792
(Iteration 3901 / 7350) loss: 1.298349
(Epoch 16 / 30) train acc: 0.543000; val\_acc: 0.513000
(Iteration 4001 / 7350) loss: 1.163151
(Iteration 4101 / 7350) loss: 1.333338
(Epoch 17 / 30) train acc: 0.523000; val\_acc: 0.522000
(Iteration 4201 / 7350) loss: 1.220220
(Iteration 4301 / 7350) loss: 1.396700
(Iteration 4401 / 7350) loss: 1.153267
(Epoch 18 / 30) train acc: 0.555000; val\_acc: 0.515000
(Iteration 4501 / 7350) loss: 1.140811
(Iteration 4601 / 7350) loss: 1.295988
(Epoch 19 / 30) train acc: 0.533000; val\_acc: 0.511000
(Iteration 4701 / 7350) loss: 1.244502
(Iteration 4801 / 7350) loss: 1.220347
(Epoch 20 / 30) train acc: 0.559000; val\_acc: 0.523000
(Iteration 4901 / 7350) loss: 1.241858
(Iteration 5001 / 7350) loss: 1.200796
(Iteration 5101 / 7350) loss: 1.394251
(Epoch 21 / 30) train acc: 0.577000; val\_acc: 0.525000
(Iteration 5201 / 7350) loss: 1.206179
(Iteration 5301 / 7350) loss: 1.098918
(Epoch 22 / 30) train acc: 0.565000; val\_acc: 0.520000
(Iteration 5401 / 7350) loss: 1.171365
(Iteration 5501 / 7350) loss: 1.228942
(Iteration 5601 / 7350) loss: 1.163735
(Epoch 23 / 30) train acc: 0.555000; val\_acc: 0.528000
(Iteration 5701 / 7350) loss: 1.308055
(Iteration 5801 / 7350) loss: 1.228080
(Epoch 24 / 30) train acc: 0.589000; val\_acc: 0.519000
(Iteration 5901 / 7350) loss: 1.227241
(Iteration 6001 / 7350) loss: 1.333534
(Iteration 6101 / 7350) loss: 1.144016
(Epoch 25 / 30) train acc: 0.556000; val\_acc: 0.527000
(Iteration 6201 / 7350) loss: 1.231183
(Iteration 6301 / 7350) loss: 1.239231
(Epoch 26 / 30) train acc: 0.573000; val\_acc: 0.521000
(Iteration 6401 / 7350) loss: 1.195566
(Iteration 6501 / 7350) loss: 1.243926
(Iteration 6601 / 7350) loss: 1.287579
(Epoch 27 / 30) train acc: 0.570000; val\_acc: 0.528000
(Iteration 6701 / 7350) loss: 1.144590
(Iteration 6801 / 7350) loss: 1.183924
(Epoch 28 / 30) train acc: 0.564000; val\_acc: 0.525000
(Iteration 6901 / 7350) loss: 1.211566
(Iteration 7001 / 7350) loss: 1.274509
(Iteration 7101 / 7350) loss: 1.162291
(Epoch 29 / 30) train acc: 0.578000; val\_acc: 0.520000
(Iteration 7201 / 7350) loss: 1.156013
(Iteration 7301 / 7350) loss: 1.281307
(Epoch 30 / 30) train acc: 0.561000; val\_acc: 0.519000
new best accuracy:  0.528
    \end{Verbatim}

    \hypertarget{test-your-model}{%
\section{Test your model!}\label{test-your-model}}

Run your best model on the validation and test sets. You should achieve
above 48\% accuracy on the validation set and the test set.

    \begin{tcolorbox}[breakable, size=fbox, boxrule=1pt, pad at break*=1mm,colback=cellbackground, colframe=cellborder]
\prompt{In}{incolor}{ }{\boxspacing}
\begin{Verbatim}[commandchars=\\\{\}]
\PY{n}{y\PYZus{}val\PYZus{}pred} \PY{o}{=} \PY{n}{np}\PY{o}{.}\PY{n}{argmax}\PY{p}{(}\PY{n}{best\PYZus{}model}\PY{o}{.}\PY{n}{loss}\PY{p}{(}\PY{n}{data}\PY{p}{[}\PY{l+s+s1}{\PYZsq{}}\PY{l+s+s1}{X\PYZus{}val}\PY{l+s+s1}{\PYZsq{}}\PY{p}{]}\PY{p}{)}\PY{p}{,} \PY{n}{axis}\PY{o}{=}\PY{l+m+mi}{1}\PY{p}{)}
\PY{n+nb}{print}\PY{p}{(}\PY{l+s+s1}{\PYZsq{}}\PY{l+s+s1}{Validation set accuracy: }\PY{l+s+s1}{\PYZsq{}}\PY{p}{,} \PY{p}{(}\PY{n}{y\PYZus{}val\PYZus{}pred} \PY{o}{==} \PY{n}{data}\PY{p}{[}\PY{l+s+s1}{\PYZsq{}}\PY{l+s+s1}{y\PYZus{}val}\PY{l+s+s1}{\PYZsq{}}\PY{p}{]}\PY{p}{)}\PY{o}{.}\PY{n}{mean}\PY{p}{(}\PY{p}{)}\PY{p}{)}
\end{Verbatim}
\end{tcolorbox}

    \begin{Verbatim}[commandchars=\\\{\}]
Validation set accuracy:  0.528
    \end{Verbatim}

    \begin{tcolorbox}[breakable, size=fbox, boxrule=1pt, pad at break*=1mm,colback=cellbackground, colframe=cellborder]
\prompt{In}{incolor}{ }{\boxspacing}
\begin{Verbatim}[commandchars=\\\{\}]
\PY{n}{y\PYZus{}test\PYZus{}pred} \PY{o}{=} \PY{n}{np}\PY{o}{.}\PY{n}{argmax}\PY{p}{(}\PY{n}{best\PYZus{}model}\PY{o}{.}\PY{n}{loss}\PY{p}{(}\PY{n}{data}\PY{p}{[}\PY{l+s+s1}{\PYZsq{}}\PY{l+s+s1}{X\PYZus{}test}\PY{l+s+s1}{\PYZsq{}}\PY{p}{]}\PY{p}{)}\PY{p}{,} \PY{n}{axis}\PY{o}{=}\PY{l+m+mi}{1}\PY{p}{)}
\PY{n+nb}{print}\PY{p}{(}\PY{l+s+s1}{\PYZsq{}}\PY{l+s+s1}{Test set accuracy: }\PY{l+s+s1}{\PYZsq{}}\PY{p}{,} \PY{p}{(}\PY{n}{y\PYZus{}test\PYZus{}pred} \PY{o}{==} \PY{n}{data}\PY{p}{[}\PY{l+s+s1}{\PYZsq{}}\PY{l+s+s1}{y\PYZus{}test}\PY{l+s+s1}{\PYZsq{}}\PY{p}{]}\PY{p}{)}\PY{o}{.}\PY{n}{mean}\PY{p}{(}\PY{p}{)}\PY{p}{)}
\end{Verbatim}
\end{tcolorbox}

    \begin{Verbatim}[commandchars=\\\{\}]
Test set accuracy:  0.519
    \end{Verbatim}

    \hypertarget{inline-question-2}{%
\subsection{Inline Question 2:}\label{inline-question-2}}

Now that you have trained a Neural Network classifier, you may find that
your testing accuracy is much lower than the training accuracy. In what
ways can we decrease this gap? Select all that apply.

\begin{enumerate}
\def\labelenumi{\arabic{enumi}.}
\tightlist
\item
  Train on a larger dataset.
\item
  Add more hidden units.
\item
  Increase the regularization strength.
\item
  None of the above.
\end{enumerate}

\(\color{blue}{\textit Your Answer:}\)

1., 3.

\(\color{blue}{\textit Your Explanation:}\)

对训练集拟合效果好而对测试集拟合效果差得多, 说明对训练集存在过拟合现象,
可以通过增大训练集或增大regularization程度来避免过拟合现象的发生.

    \begin{tcolorbox}[breakable, size=fbox, boxrule=1pt, pad at break*=1mm,colback=cellbackground, colframe=cellborder]
\prompt{In}{incolor}{ }{\boxspacing}
\begin{Verbatim}[commandchars=\\\{\}]

\end{Verbatim}
\end{tcolorbox}


    % Add a bibliography block to the postdoc
    
    
    
\end{document}
